%%%%%%%%%%%%%%%%%%%%%%%%%%%%%%%%%%%%%%%%%%%%%%%%%%%%%%%%%%%%%%%%%%%%%%%%
% vim:enc=utf-8:ts=5:sw=5:et:ff=unix:
%%%%%%%%%%%%%%%%%%%%%%%%%%%%%%%%%%%%%%%%%%%%%%%%%%%%%%%%%%%%%%%%%%%%%%%%

\chapter{Introdução}
%
A edição de texto é uma das tarefas mais frequentemente executadas por seres
humanos em ambientes computacionais, em qualquer nível. Usuários finais,
administradores de sistemas, programadores de software, desenvolvedores {\em
web}, e tantas outras categorias, todos eles, constantemente, necessitam
editar textos. 

Usuários finais editam texto para criar documentos, enviar e-mails, atualizar
o blog, escrever recados ou simplesmente trocar mensagens instantâneas pela
internet. Administradores de sistemas editam arquivos de configuração, criam
regras de segurança, editam {\em scripts} e manipulam saídas de comandos
armazenados em arquivos de texto. Programadores desenvolvem códigos-fonte e a
documentação de programas essencialmente em editores de texto.  Desenvolvedores
{\em web} interagem com editores de texto para criarem {\em layout} e dinâmica 
de sites.

Tamanha é a frequência e onipresença da tarefa de edição de texto que a
eficiência, flexibilidade e o repertório de ferramentas de editores de texto
tornam-se quesitos críticos para se atingir {\em produtividade} e {\em
conforto} na edição de textos.

% falar da não trivialidade do aprendizado (curva de aprendizado), mas também
% da eficiência e produtividade a médio/longo prazo
% Mitre - adicionando essa sugestão acima... 
%         requer revisão, quem o fizer, elimine esses comentários.
Qualquer tarefa de aprendizado requer um certo esforço. Todo programa 
introduz novos conceitos, opções e configurações que transformam o \textit{
modus operanti} do usuário. Em princípio, quanto maior o esforço, maior o
benefício. Quem quer apenas escrever textos, pode-se contentar com um editor
básico, cuja as únicas opções são digitar o texto, abrir e salvar o documento 
ou pode utilizar um editor que permita pré-configurar ações, formatar o 
conteúdo, revisar a ortografia, etc, além da ação básica que é escrever textos.

Qualquer usuário de computador pode abrir o primeiro tipo de editor e
imediatamente começar a escrever, a curto prazo, sua ação terá consequências
imediatas e não requer conhecimentos adicionais. Por outro lado, esse usuário
terá que fazer esforço para digitar o mesmos cabeçalho todos os dias. 

O outro tipo de editor permite que o usuário pré-configure o cabeçalho do
documento e todos os dias esse trecho já estará digitado. Em contrapartida, o
usuário deve aprender como pré-configurar o editor. O que requer esforço para
aprender a utilizar o programa escolhido. O benefício somente será observado a
médio/longo prazo, quando o tempo ganho ao utilizar a configuração será
superior ao tempo consumido aprendendo sobre o programa.
% ---------------------

O ``\href{http://www.vim.org}{Vim}''\footnote{Vim - \url{http://www.vim.org}}
\index{vim} é um editor de texto extremamente configurável, criado para
permitir a edição de forma eficiente, tornando-a produtiva e confortável. 
Também é uma aprimoração do editor ``Vi'', um tradicional programa dos
sistemas Unix. Possui uma série de mudanças em relação a este último. O
próprio slogan do Vim é {\em Vi IMproved}, ou seja, {\em Vi Melhorado}.  O Vim
é tão conhecido e respeitado entre programadores, e tão útil para programação,
que muitos o consideram uma verdadeira ``IDE\index{ide} (\textit{Integrated 
Development Environment}, em português, Ambiente Integrado de 
Desenvolvimento''.

Ele é capaz de reconhecer mais de 500 sintaxes de linguagens de programação e
marcação, possui mapeamento para teclas, macros, abreviações, busca por
{\em{Expressões
Regulares}}\index{expressões regulares}\footnote{Expressões Regulares - 
\url{http://guia-er.sourceforge.net/guia-er.html}}, entre outras facilidades.

% NOTA: não estou convencido sobre a relevância desse conteúdo, fica pelo menos
% como um o primeiro exemplo.
% referenciando a figura. O til serve para evitar que o número fique 
% em um linha e o nome em outro.
% toda referência é feita antes de inserir a figura.
A figura~\ref{fig:vimedittex} mostra o vim sendo usando para editar o arquivo
o desse livro sobre vim.

% Inserindo apenas uma figura
% A opção [htp] é o melhor conjunto para livros, pois evita que a figura
% se afaste muito do texto onde é citado.
\begin{figure}[htp]
  % centralizando a figura
  \centering 
  % especficiando a figura
  % a figura não precisa ter extensão
  % width : define o tamanho lateral
  %         use a unidade que conhecer, px, m, mm, cm, in, etc.
  %         usar apenas essa opção assegura que a proporção seja mantida.
  %         é possível especificar a outra dimensão ou uma fator global, 
  %         exemplos estão comentados, para height (altura) e scale (escala)
  % tudo que está entre colchetes é opcional, e existem muitas opções para 
  % esse campo, mas o que está abaixo é suficiente em 90 % dos casos
  \includegraphics[width=9cm]{img/vimedittex} % 9 cm de lado
  %\includegraphics[height=7cm]{img/vimedittex} % 7 cm de altura
  %\includegraphics[scale=0.4]{img/vimedittex} % 40 % do tamanho real
  % legenda da figura, a legenda vem depois da figura.
  \caption{Usando o vim para editar o código em \LaTeX}
  % rotulado essa figura, é usando por \ref{} para citar a figura.
  \label{fig:vimedittex}
\end{figure}

O Vim conta com uma comunidade bastante atuante e é, ao lado do
Emacs\footnote{Emacs - \url{http://www.gnu.org/software/emacs/}}, um dos 
editores mais usados nos sistemas GNU/Linux\footnote{O kernel Linux sem os 
programas GNU não serviria para muita coisa.}, embora esteja também disponível
em outros sistemas, como o Windows e o Macintosh. 
%O site oficial do Vim é \url{http://www.vim.org}.

\section{Instalação do Vim}\index{vim!instalar}
\vimhelp{install}
%
\subsection{Instalação no Windows}
%
Há uma versão gráfica do Vim disponível para vários sistemas operacionais, 
incluindo o Windows; esta versão pode ser encontrada no 
\href{http://www.vim.org/download.php}{site oficial}~\cite{SiteOficialDownloads}. 
Para instalá-lo basta baixar o instalador no link indicado e dispará-lo com um
duplo clique (este procedimento requer privilégios de administrador).

\subsection{Instalação no GNU/Linux}
%
A maioria das distribuições GNU/Linux traz o Vim em seus repositórios, sendo
que é bastante comum o Vim já vir incluído na instalação típica da distribuição.
A forma de instalação preferível depende do Vim:
\begin{itemize}
\item Já vir instalado por {\em default} -- neste caso nada precisa ser feito.

\item Estar disponível no repositório, mas não instalado -- em distribuições
derivadas da Debian GNU/Linux\footnote{Debian GNU/Linux - \url{http://www.debian.org/index.pt.html}},
a instalação do Vim através dos repositórios é usualmente executada
digitando-se {\tt `apt-get install vim'}\footnote{Recomenda-se também instalar
a documentação em HTML do Vim: {\tt `apt-get install vim-doc'}} em um {\em terminal} (este procedimento
requer privilégios de administrador e, tipicamente, conexão com a internet).
 
Algumas distribuições GNU/Linux dividem o programa vim em vários pacotes. 
Pacotes adicionais como \texttt{gvim} e \texttt{vim-enhanced}, entre outros,
representam diferentes versões do mesmo aplicativo. O  \texttt{gvim} é a versão
gráfica do Vim e o \texttt{vim-enhanced} é uma versão do vim compilada com um
suporte interno ao Python\footnote{O Python (\url{http://www.python.org}) é uma
linguagem de programação orientada a objetos muito comum no meio profissional e
acadêmico}.
A alternativa para resolver esse problema é buscar na documentação da 
distribuição o que significa cada pacote.

\item Não estar disponível no repositório da distribuição -- cenário {\em muito}
improvável, mas nas sua ocorrência o Vim pode ser instalado através da compilação do
código-fonte; basta seguir as instruções do 
\href{http://www.vim.org/download.php}{site oficial}~\cite{SiteOficialDownloads}.

\end{itemize}

\section{Dicas iniciais}\label{Dicas iniciais}
%
Ao longo do livro alguns comandos ou dicas podem estar duplicados, o que
é útil devido ao contexto e também porque o aprendizado por saturação
é um ótimo recurso. Ao perceber uma dica duplicada, antes de
reclamar veja se já sabe o que está sendo passado. % Tem certeza que esse
% comentário é necessário, parece ofensivo...
Contudo dicas e sugestões serão bem vindas! 

Para abrir um arquivo\index{iniciar} com Vim digite num terminal:
%
\begin{verbatim}
     vim texto.txt
\end{verbatim}
onde {\tt texto.txt} é o nome do arquivo que deseja-se criar ou editar.

Em algumas distribuições, pode-se usar o comando {\tt vi} ao invés de {\tt vim}.

\section{Ajuda integrada}
%
O Vim possui uma ajuda\index{ajuda}\index{manual} integrada muito completa, são mais
de 100 arquivos somando milhares de linhas. O único inconveniente é não haver ainda
tradução para o português, sendo o inglês seu idioma oficial; entretanto, as explicações
costumam ser sintéticas e diretas, de forma que noções em inglês seriam
suficientes para a compreensão de grande parte do conteúdo da ajuda integrada.

Obs: No Vim quase todos os comandos podem ser abreviados, no caso
``\verb+help+'' pode ser chamado por ``\verb+h+'' e assim por diante. Um
comando só pode ser abreviado até o ponto em que este nome mais curto não
coincida com o nome de algum outro comando existente.  Para chamar a ajuda do
Vim pressione `\texttt{Esc}' e em seguida:
%Vim pressione \verb|<Esc>| e em seguida:
%
\begin{verbatim}
     :help .... versão longa, ou
     :h ....... versão abreviada
\end{verbatim}
%
ou simplesmente `\texttt{F1}'.

Siga os links usando o atalho `\verb|ctrl+]|', em modo gráfico o clique com o
mouse também funciona, e para voltar use `\verb|ctrl+o|' ou
`\verb|ctrl+t|' Para as situações de desespero pode-se digitar:

\begin{verbatim}
     :help!
\end{verbatim}

{\Large {\ding{45}}} Quando um comando puder ser abreviado poderá aparecer
desta forma: `\texttt{:so[urce]}'. Deste modo se está indicando que o comando 
`\texttt{:source}' pode ser usado de forma abreviada, no caso `\texttt{:so}'.

\section{Em caso de erros }\label{Em caso de erros }
%
Recarregue\index{em caso de erros} o arquivo que está sendo editado pressionando
`\verb|Esc|' e em seguida usando o comando `\texttt{:e}'.
ou simplesmente inicie outro arquivo ignorando o atual, com o comando `\texttt{:enew!}',
ou saia do arquivo sem modifica-lo, com `\texttt{:q!}'. Pode-se ainda tentar gravar
forçado com o comando `\texttt{:wq!}'


\section{Como interpretar atalhos e comandos}\label{Como interpretar atalhos e comandos}
%
A tecla ``\verb|<Ctrl>|''\index{tecla!\texttt{<ctrl>}} é representada na maioria dos manuais e na ajuda
pelo caractere ``\verb|^|'' circunflexo, ou seja, o atalho \verb|Ctrl-L| aparecerá assim:
\begin{verbatim}
     ^L
\end{verbatim} % aqui pula-se linha, porque abaixo começa um novo parágrafo

No arquivo de configuração do Vim, um ``\verb|<Enter>|'' pode aparecer como:
\begin{verbatim}
     <cr>
\end{verbatim}

Para saber mais sobre como usar atalhos\index{atalhos}\index{mapeamento} no Vim
veja a seção \ref{Mapeamentos} na página~\pageref{Mapeamentos} e para ler sobre
o arquivo de configuração veja o capítulo \ref{cha:Como editar preferências no
Vim} na página \pageref{cha:Como editar preferências no Vim}.

\section{Modos de operação}\label{Modos de operação}

A tabela abaixo mostra uma referência rápida para os modos de operação\index{modos de operação}do Vim,
a seguir mais detalhes sobre cada um dos modos. 

\begin{tabular}{|l|l|l|}
\hline
\textbf{Modo} & \textbf{Descrição} & \textbf{Atalho} \tabularnewline
\hline \hline
Normal\index{modo normal} & Para deletar, copiar, formatar, etc & 
                            {\tt <Esc>}\tabularnewline \hline
                            %
Inserção\index{modo de inserção} & Prioritariamente, digitação de texto &
                            {}{\tt i,a,I,A,o,O}\tabularnewline \hline
                            %
Visual\index{modo visual} & Seleção de blocos verticais e linhas inteiras &
                            {}{\tt V, v, Ctrl-v} \tabularnewline \hline
                            %
Comando\index{modo de comando} & Uma verdadeira linguagem de programação &
                            {}{\tt <Esc>:}\tabularnewline \hline
                            %
\end{tabular}

Em oposição à esmagadora maioria dos editores o Vim é um editor que trabalha
com ``modos de operação (modo de inserção, modo normal, modo visual e etc)'', o
que a princípio dificulta a vida do iniciante, mas abre um universo de
possibilidades, pois ao trabalhar com modos distintos uma tecla de atalho pode
ter vários significados, exemplificando\index{modos de operação!exemplos}: Em
modo normal pressionar `{\tt dd}' apaga a linha atual, já em modo de inserção
ele irá se comportar como se você estivesse usando qualquer outro editor, ou
seja, irá inserir duas vezes a letra `{\tt d}'.  Em modo normal pressionar a
tecla `{\tt v}' inicia uma seleção visual (use as setas de direção).  Para sair
do novo visual \verb|<Esc>|.  Como um dos princípios do vim é agilidade pode-se
usar ao invés de setas (em modo normal) as letras {\tt h,l,k,j} como se fossem
setas:

\begin{verbatim}
         k
     h       l
         j
\end{verbatim}

Imagine as letras acima como teclas de direção, a letra `{\tt k}' é uma seta acima
a letra `{\tt j}' é uma seta abaixo e assim por diante.

\section{Entrando em modo de edição}\label{Entrando em modo de edição}
\index{modo de inserção}
Estando no modo normal, digita-se:
\begin{verbatim}
     a .... inicia inserção de texto após o atual
     i .... inicia inserção de texto antes do caractere atual
     A .... inicia inserção de texto no final da linha
     I .... inicia inserção de texto no começo da linha
     o .... inicia inserção de texto na linha abaixo
     O .... inicia inserção de texto na linha acima
\end{verbatim}

Outra possibilidade é utilizar a tecla \verb|<Insert>| para entrar no modo de inserção de
texto antes do caractere atual, ou seja, o mesmo que a tecla \verb|i|. Uma vez no modo de 
inserção, a tecla \verb|<Insert>| permite alternar o modo de digitação de inserção de 
simples de caracteres para substituição de caracteres.

Agora começamos a sentir o gostinho de usar o Vim, uma tecla seja
maiúscula ou minúscula, faz muita diferença se você não estiver em
modo de inserção, e para sair do modo de inserção e voltar ao modo normal sempre 
use \verb|<Esc>|.

\section{Erros comuns}\label{sec:Erros comuns}
\index{modos de operação!errors comuns}
\begin{itemize}

\item Estando em {\em{modo de inserção}} pressionar `{\tt j}' na intenção
de rolar o documento, neste caso estaremos inserindo simplesmente a letra `{\tt j}'. 

\item Estando em {\em{modo normal}} acionar acidentalmente o ``\verb+<Caps Lock>+'' 
e tentar rolar o documento usando a letra ``\verb+J+'', o efeito é a
junção das linhas, aliás um ótimo recurso quando a intenção é de fato esta.

\item Em {\em{modo normal}} tentar digitar {\em{um número seguido de uma palavra}} e ao perceber que 
nada está sendo digitado, iniciar o modo de inserção, digitando por fim o que se queria, 
o resultado é que o número que foi digitado inicialmente vira um quantificador para o que 
se digitou ao entrar no modo de inserção. A palavra aparecerá repetida na quantidade do 
número digitado. Assim, se você quiser digitar 10 vezes ``\verb+isto é um teste+''
 faça assim:

\begin{verbatim}
     <Esc> ........... se assegure de estar em modo normal
     10 .............. quantificador
     i ............... entra no modo de inserção
     isto é um teste <Enter> <Esc>  
\end{verbatim}

\end{itemize}

{\Large {\ding{45}}} Alguns atalhos úteis\dots
\index{atalhos}
\begin{verbatim}
     Ctrl-O ..... comando do modo normal no modo insert
     i Ctrl-a ... repetir a última inserção
     @: ......... repetir o último comando
     Shift-insert colar texto da área de transferência
     gi ......... modo de inserção no mesmo ponto da última vez
     gv ......... repete seleção visual
\end{verbatim}

Para saber mais sobre repetição de comandos veja o capítulo \ref{Repetição de comandos},
na página \pageref{Repetição de comandos}.

No Vim, cada arquivo aberto é chamado de \verb|buffer|, ou seja, dados
carregados na memória. Você pode acessar o mesmo {\em buffer} em mais de uma
janela, bem como dividir a janela em vários {\em buffers} distintos o que veremos
mais adiante.


