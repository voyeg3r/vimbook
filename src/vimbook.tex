%%%%%%%%%%%%%%%%%%%%%%%%%%%%%%%%%%%%%%%%%%%%%%%%%%%%%%%%%%%%%%%%%%%%%%%%%%%%%%%%
% Permission is granted to copy, distribute and/or modify this document
% under the terms of the GNU Free Documentation License, Version 1.3 or
% any later version published by the Free Software Foundation.
%
% A copy of the license is included in the file called "COPYING".
%%%%%%%%%%%%%%%%%%%%%%%%%%%%%%%%%%%%%%%%%%%%%%%%%%%%%%%%%%%%%%%%%%%%%%%%%%%%%%%%

\documentclass[10pt,a4paper,openany]{book}
\usepackage[utf8]{inputenc} %se for usar no windows mude esta linha
\usepackage[portuges]{babel}
\usepackage[pdftex]{color,graphicx}
\usepackage[T1]{fontenc} %fonte computer modern
\usepackage{graphicx}
\usepackage{url}
\RequirePackage{enumerate} %extensão para o pacote enumerate
\usepackage{marvosym} % gera alguns símbolos
\usepackage{hyperref} % gera os links no pdf
\usepackage{pifont} % simbolos de setas
\usepackage{hyperref}
\usepackage{fancyhdr}

\hypersetup{backref=true,pdfpagemode=UseOutlines,colorlinks=true,a5paper,
breaklinks=true,hyperindex,linkcolor=blue,anchorcolor=black,citecolor=green,
filecolor=magenta,menucolor=red,pagecolor=red,urlcolor=cyan,bookmarks=true,
bookmarksopen=false,pdfpagelayout=SinglePage,pdfpagetransition=Dissolve}

\pagestyle{fancy}
\renewcommand{\chaptermark}[1]{\markboth{#1}{}}
\renewcommand{\sectionmark}[1]{\markright{\thesection\ #1}{}}
\fancyhf{}
\fancyhead[LE,RO]{\bfseries\thepage}
\fancyhead[LO]{\bfseries\rightmark}
\fancyhead[RE]{\bfseries\leftmark}
\renewcommand{\headrulewidth}{0.5pt}
\renewcommand{\footrulewidth}{0.0pt}
\addtolength{\headheight}{2.5pt}    % 0.5pt
\fancypagestyle{plain}{
     \fancyhead{}
     \renewcommand{\headrulewidth}{0pt}
     }

\newcommand{\titulo}{\Huge \sc Pequeno guia do Vim}

\newcommand{\dica}[1]{{{\large \ding{42}}}}

\setlength{\parindent}{0em}
\begin{document}
%\maketitle
\DeclareGraphicsExtensions{.jpg,.pdf,.mps,.png,}

%%%%%%%%%%%%%%%%%%%% Título do documento %%%%%%%%%%%%%%%%%%%%%%%
\begin{titlepage}
 \begin{flushleft}
   \vspace{2mm}
   \vspace{2mm}
   \vspace{2mm}
 \end{flushleft}
   %\vspace{2cm}
\begin{center}
   \rule{12cm}{1mm} \\ \vspace{2mm}
   {\LARGE{\bf{\titulo}}} \\
   \vspace{-1mm}
   \rule{12cm}{1mm} \\

% logo do Vim
  \vspace{2cm}
  \begin{figure}[h]
    \center
    \includegraphics{img/vimlogo.png}
    \label{logodovim}
    %\caption{Logo do Vim}
\end{figure}
% fim da imagem

   \vspace{3cm}
   \begin{flushright}
   \begin{minipage}[t]{8cm}
          ``Um guia simples, com exemplos de comandos
          para uso no dia-a-dia. A idéia é que este
          material cresça e torne-se uma referência confiável
          e prática. Use este livro nos termos da {\em Licença de Documentação Livre GNU} (GFDL).'' \\
          \par Este trabalho está em constante aprimoramento, e é fruto da
          colaboração de voluntários. Participe do desenvolvimento enviando sugestões e
          melhorias; acesse o site do projeto no endereço: \\
\begin{center}
          \url{http://code.google.com/p/vimbook}
\end{center}
        
   \end{minipage} \\
   \end{flushright}

   \vspace{3cm}

   {\small Versão gerada em \\ \bf \today}
\end{center}
\end{titlepage}
%%%%%%%%%%%%%%%%%%%%%%%%%%%%%%%%%%%%%%%%%%%%%%%%%%%%%%%

\newpage

\begin{center}
{\Huge \bf Autores}

\vspace{2cm}

\begin{tabular}{cc}

\bf Sérgio Luiz Araújo Silva & \tt <voyeg3r@gmail.com> \\
\bf Douglas Adriano Augusto & \tt <daaugusto@gmail.com>\\
$\vdots$ & $\vdots$ \\

\end{tabular}

\end{center}

\newpage
\tableofcontents

\chapter{Introdução}

O ``Vim'' é um editor extremamente configurável, criado para permitir a edição
de textos de forma eficiente. Também é um melhoramento do editor ``Vi'', um
tradicional programa dos sistemas Unix. Possui uma série de mudanças em
relação a este último. O próprio slogan do Vim é {\em Vi IMproved}, ou seja,
{\em Vi Melhorado}.  O Vim é tão conhecido e respeitado entre programadores, e tão
útil para programação, que muitos o consideram uma verdadeira
``IDE\footnote{Ambiente Integrado de desenvolvimento.}''.

Ele é capaz de reconhecer mais de 500 sintaxes de linguagens de programação e
marcação, possui mapeamento para teclas, macros, abreviações, busca por
{\em{Expressões
Regulares}}\footnote{http://guia-er.sourceforge.net/guia-er.html}, entre
outras facilidades. Conta com uma comunidade bastante atuante e é, ao lado do
Emacs\footnote{http://www.gnu.org/software/emacs/} um dos editores mais usados
nos sistemas GNU/Linux\footnote{O kernel Linux sem os programas GNU não
serviria para muita coisa.}, mas pode ser também instalado em outros sistemas,
como o Windows e o Macintosh.  O site oficial do Vim é \url{http://www.vim.org}.

\section{Instalação do Vim}
\subsection{Instalação no Windows}

Há uma versão gráfica do vim instalável para vários sistemas, incluindo o Windows,
podemos encontra-la em \url{http://www.vim.org/download.php#pc}, para instalar basta
baixar o instalador no link indicado e disparar o instalador com um duplo clique, lembre-se 
de que você tem que estar {\em logado} como administrador da máquina para rodar o instalador.

\subsection{Instalação no Linux}

Nos sistemas GNU/Linux baseados em Debian\footnote{http://www.debian.org/index.pt.html} 
você precisa: 

\begin{itemize}
\item Estar logado com privilégios administrativos
\item Ter conexão com internet devidamente configurada ou CDs de instalação
\item executar no terminal os seguintes comandos

\begin{verbatim}
     apt-get update
     apt-get install vim-gnome vim-doc
\end{verbatim}

\end{itemize}

\section{Dicas iniciais}\label{Dicas iniciais}

Ao longo do livro alguns comandos ou dicas podem estar duplicados, o que
é útil devido ao contexto e também porque o aprendizado por saturação
é um ótimo recurso. Portanto se ver uma dica duplicada, antes de
reclamar veja se já sabe o que está sendo passado! \\

Para chamar o Vim digite num terminal:

\begin{verbatim}
     vim meu_texto.txt
\end{verbatim}

\section{Ajuda integrada}

O Vim possui uma ajuda integrada muito completa, são mais de 100 arquivos
somando milhares de linhas. O único inconveniente é o idioma ``inglês'' se bem
que muito do meu aprendizado de inglês se deve a este detalhe.
Obs: no Vim quase todos os comandos podem ser abreviados, no caso
``\verb+help+'' pode ser chamado por ``\verb+h+'' e assim por diante.
Para chamar a ajuda do Vim:  \\

Pressione \verb|<Esc>| e em seguida\dots

\begin{verbatim}
     :help .... versão longa
     :h ....... versão abreviada
\end{verbatim}

Ou simplesmente:

\begin{verbatim}
     <F1>
\end{verbatim}

Siga os links usando o atalho ``\verb|Ctrl-]|'', e para voltar use
  ``\verb|Ctrl-O|''.

Se você estiver realmente desesperado, digite:

\begin{verbatim}
     :help!
\end{verbatim}

\section{Em caso de erros }\label{Em caso de erros }
Recarregue o arquivo que está sendo editado assim:

\begin{verbatim}
     <Esc> .. para sair do modo de edição
     :e! .... recarrega o arquivo sem qualquer edição
\end{verbatim}

Ou simplesmente inicie outro arquivo ignorando o atual

\begin{verbatim}
     :enew!
\end{verbatim}

ou saia do arquivo sem modifica-lo

\begin{verbatim}
     :q! .... saída forçada, nada é alterado
     :wq! ... tenta gravar e sair forçado
\end{verbatim}

\section{Como interpretar atalhos e comandos}\label{Como interpretar atalhos e comandos}

A tecla ``\verb|<Ctrl>|'' é representada na maioria dos manuais e na ajuda
pelo caractere ``\verb|^|'' circunflexo, ou seja, o atalho \verb|Ctrl-L| aparecerá assim:

\begin{verbatim}
     ^L
\end{verbatim}

No arquivo de configuração do Vim, um ``\verb|<Enter>|'' pode aparecer como:

\begin{verbatim}
     <cr>
\end{verbatim}

Para saber mais sobre como usar atalhos no Vim veja a seção \ref{Notas sobre
mapeamentos} na página \pageref{Notas sobre mapeamentos} e para ler sobre o
arquivo de configuração veja o capítulo \ref{cha:Como editar preferências no Vim} na página
\pageref{cha:Como editar preferências no Vim}.


\section{Modos de operação}\label{Modos de operação}

Em oposição à esmagadora maioria dos editores o Vim é um editor modal, o que a
princípio dificulta a vida do iniciante, mas abre um universo de
possibilidades, pois ao trabalhar com modos distintos uma tecla de
atalho pode ter vários significados, senão vejamos:
Em modo normal pressionar duas vezes a letra ``d''
\begin{verbatim}
     dd
\end{verbatim}
apaga a linha atual, já em modo de inserção ele irá se comportar como se você estivesse
usando qualquer outro editor, ou seja, irá inserir duas vezes a letra ``d''.

Em modo normal pressionar a tecla ``v'' inicia uma seleção visual (use as setas de direção).
Para sair do novo visual \verb|<Esc>|, mas o Vim tem, em modo normal teclas de direção mais práticas

\begin{verbatim}
         k
     h       l
         j
\end{verbatim}

Imagine as letras acima como teclas de direção, a letra ``k'' é uma seta acima
a letra ``j'' é uma seta abaixo e assim por diante.

\section{Entrando em modo de edição}
\label{Entrando em modo de edição}

\begin{verbatim}
     a .... inicia inserção de texto após o atual
     i .... inicia inserção de texto antes do caractere atual
     A .... inicia inserção de texto no final da linha
     I .... inicia inserção de texto no começo da linha
     o .... inicia inserção de texto na linha abaixo
     O .... inicia inserção de texto na linha acima
\end{verbatim}

Agora começamos a sentir o gostinho de usar o Vim, uma tecla seja
maiúscula ou minúscula, faz muita diferença se você não estiver em
modo de inserção, e para sair do modo de inserção sempre use \verb|<Esc>|.

A tabela abaixo mostra uma referência rápida para os modos de operação do Vim,
a seguir mais detalhes sobre cada um dos modos.

\begin{description}

\item [Normal] Neste modo podemos colar o que está no
 ``\verb|buffer|\footnote{No Vim a memória é chamada de buffer, assim como
 arquivos carregados.}'', uma espécie de área de transferência. Podemos ter um
 ``\verb|buffer|'' para cada letra do alfabeto, também é possível apagar linhas, e
 colocar trechos no ``\verb|buffer|''. Quando se inicia o Vim já estamos neste modo;
 caso esteja em outro modo basta pressionar ``\verb|<Esc>|''.  Para acessar:

\begin{verbatim}
     <Esc> ....... sai do modo de inserção
     ^[ .......... Ctrl-[ também sai do modo de inserção
\end{verbatim}

Para ler mais sobre \verb|buffers| veja o capítulo \ref{Registros}.

\item [Inserção]
Neste modo é feita a inserção de texto. Para entrar neste modo basta
pressionar a tecla ``\verb|i|'' ({\em insert}) ou ``\verb|c|'' ({\em change})
ou tecla ``\verb|a|'' ({\em append}).  Para substituir um único caractere você
pode usar:

\begin{verbatim}
     r<char> ...... onde char pode ser qualquer caractere
\end{verbatim}

Para trocar caracteres de lugar faça:

\begin{verbatim}
     xp ........... troca letras de lugar
\end{verbatim}


Para acessar este modo:  \verb+i,a,I,A,o,O+

\item [Visual] Neste modo podemos selecionar blocos verticais de texto. É
exibido um destaque visual. É uma das melhores formas de se copiar
conteúdo no Vim.

Para acessar (a partir do modo normal):

\begin{verbatim}
     v ...... seleção de caracteres
     v5j .... seleção visual para as pŕoximas 5 linhas
     V ...... (maiúsculo) - seleção de linhas inteiras
     Ctrl-v . Seleciona blocos de texto (use setas)
\end{verbatim}

\item [Comando] Neste modo digitamos comandos como o de salvar

\begin{verbatim}
     :w
\end{verbatim}

ou para ir para uma linha qualquer:

\begin{verbatim}
     :100 <Enter>
\end{verbatim}

para acessar
\begin{verbatim}
     :
\end{verbatim}

\end{description}

\section{Erros comuns}\label{sec:Erros comuns}

\begin{itemize}

\item Estando em {\em{modo de inserção}} pressionar ``\verb+j+'' na intenção
de rolar o documento, neste caso estaremos inserindo simplesmente a letra ``j''. 

\item Estando em {\em{modo normal}} acionar acidentalmente o ``\verb+<Caps Lock>+'' 
e tentar rolar o documento usando a letra ``\verb+J+'', o efeito é a
junção das linhas, aliás um ótimo recurso quando a intenção é de fato esta.

\item Em {\em{modo normal}} tentar digitar {\em{um número seguido de uma palavra}} e ao perceber que 
nada está sendo digitado, iniciar o modo de inserção, digitando por fim o que se queria, 
o resultado é que o número que foi digitado inicialmente vira um quantificador par o que 
se digitou ao entrar no modo de inserção. A palavra aparecerá repetida na quantidade do 
número digitado. Assim, se você quiser digitar 10 vezes ``\verb+isto é um teste+''
 faça assim:

\begin{verbatim}
     <Esc> ........... se assegure de estar em modo normal
     10 .............. quantificador
     i ............... entra no modo de inserção
     isto é um teste <Enter> <Esc>  
\end{verbatim}

\end{itemize}


\section{Dicas}
\label{Dicas}
Para usar um comando do modo normal no modo de inserção faça:

\begin{verbatim}
     Ctrl-O (comando)
\end{verbatim}

Para repetir o último trecho do modo de inserção faça:

\begin{verbatim}
     i Ctrl-a
\end{verbatim}

Para repetir o último comando \verb+:+ faça:

\begin{verbatim}
     @:
\end{verbatim}

Para inserir texto da área de transferência (caso esteja em modo de inserção) faça:

\begin{verbatim}
     Shift-insert
\end{verbatim}

Para entrar em modo de edição no mesmo ponto da última edição

\begin{verbatim}
     gi
\end{verbatim}

Para repetir uma seleção (visual)

\begin{verbatim}
     gv
\end{verbatim}

Para saber mais sobre repetição de comandos veja o capítulo \ref{Repetição de comandos},
na página \pageref{Repetição de comandos}.

No Vim cada arquivo aberto é chamado de \verb|buffer| ou seja, dados
carregados na memória. Você pode acessar o mesmo buffer em mais de uma
janela, bem como dividir a janela em vários buffers distintos o que veremos
mais adiante.

\chapter{Movendo-se no documento}\label{cha:Movendo-se no documento}

Antes de mergulharmos mais a fundo nas teclas e atalhos de
movimentação vamos recapitular um pouco do que foi visto: \\



Estando em modo normal

\begin{verbatim}
     i ..... entra no modo de inserção antes do caractere atual
     I ..... entra no modo de inserção no começo da linha
     a ..... entra no modo de inserção após o caractere atual
     A ..... entra no modo de inserção no final da linha
     o ..... entra no modo de inserção uma linha abaixo
     O ..... entra em modo de inserção uma linha cima
     <Esc> . sai do modo de inserção
\end{verbatim}

Uma vez no modo de inserção todas as letras são, assim como nos outros
editores simples letras, mas lebre-se a tecla mágica para sair do modo
de inserção é: \verb+<Esc>+.

\begin{verbatim}
     <Esc> .... lhe leva para o modo normal
\end{verbatim}

As letras h, k, l, j funcionam como setas:

\begin{verbatim}
        k
     h     l
        j
\end{verbatim}

Ou seja, a letra ``k'' é usada para subir no texto, a letra ``j'' para
descer, a letra ``h'' para mover-se para a esquerda e a letra ``l''
para mover-se para a direita. A idéia é que se consiga ir para
qualquer lugar do texto sem tirar as mãos do teclado.



\section{Big words}
\label{Big word's}

Para o Vim ``{\em{palavras-separadas-por-hífen}}'' são consideradas em separado, portanto se você usar,
em modo normal ``\verb+w+'' avançar entre as palavras ele pulará uma de
cada vez, no entanto se usar ``\verb+W+''
em maiúsculo (como visto) ele pulará a ``a-palavra-inteira'' :)

\begin{verbatim}
     E .... pula para o final de palavras com hifen
     B .... pula palavras com hifen (retrocede)
     W .... pula palavras hifenizadas (começo)
\end{verbatim}



Para ir para linhas específicas digite

\begin{verbatim}
     :n<Enter>  ..... vai para linha ``n''
     ngg ............ vai para linha ``n''
\end{verbatim}

onde ``\verb|n|'' corresponde ao número da linha.

Para retornar ao modo normal pressione \verb|<Esc>| ou use \verb|Ctrl-[|
(\verb|^[|).

\section{Os saltos}\label{Os saltos}

\begin{verbatim}
     gg .... vai para o início do arquivo
     G ..... vai para o final do arquivo
     0 ..... vai para o início da linha
     $ ..... vai para o final da linha
     yG .... copia da linha atual até o final do arquivo
     25gg .. salta para a linha 25
     '' .... salta para a linha da última posição em que o cursor estava
     fx .... para primeria ocorrência de x
     tx .... Para ir para uma letra antes de x
     Fx .... Para ir para ocorrência anterior de x
     Tx .... Para ir para uma letra após o último x
     * ..... Próxima ocorrência de palavra sob o cursor
     % ..... localiza parênteses correspondente
     `' .... salta exatamente para a posição em que o cursor estava
     d$ .... deleta do ponto atual até o final da linha
     gi .... entra em modo de inserção no ponto da última edição
     gv .... repete a última seleção visual e posiciona o cursor neste local
     gf .... abre o arquivo sob o cursor
     gd .... salta para declaração de variável sob o cursor
     gD .... salta para declaraçao (global) de variável sob o cursor
     w ...... move para o início da próxima palavra
     W ...... pula para próxima palavra (desconsidera hífens)
     E ...... pula para o final da próxima palavra (desconsidera hifens)
     e ...... move o cursor para o final da próxima palavra
     zt ..... movo o cursor para o topo da página
     zm ..... move o cursor para o meio da página
\end{verbatim}

\section{Copiar e Deletar}\label{sec:Copiar e Deletar}

Deletar está associado à letra ``\verb|d|''.

\begin{verbatim}
     dd .... deleta linha atual
     D ..... deleta restante da linha
\end{verbatim}


``Dica'': Você pode combinar o comando de deleção ``\verb+d+'' com o
comando de movimento (considere o modo normal) para apagar até a
próxima vírgula use: ``\verb+df,+''. \\


Copiar está associado à letra ``\verb|y|''.

\begin{verbatim}
     yy .... copia a linha atual
     Y ..... copia a linha atual
\end{verbatim}

A maioria dos comandos do Vim pode ser precedida por um quantificador:

\begin{verbatim}
     5j ..... desce 5 linhas
     d5j .... deleta as próximas 5 linhas
     k ...... em modo normal sobe uma linha
     5k ..... sobe 5 linhas
     y5k .... copia 5 linhas (para cima)
     w ...... pula uma palavra para frente
     5w ..... pula 5 palavras
     d5w .... deleta 5 palavras
     b ...... retrocede uma palavra
     5b ..... retrocede 5 palavras
     fx ..... posiciona o cursor em ``x''
     dfx .... deleta até o próximo ``x''
     d5j .... deleta 5 linhas
     dgg .... deleta da linha atual até o começo do arquivo
     dG ..... deleta até o final do arquivo
     yG ..... copia até o final do arquivo
     yfx .... copia até o próximo ``x''
     y5j .... copia 5 linhas
\end{verbatim}

Podemos pular sentenças:

\begin{verbatim}
     ) .... pula uma sentença para frente
     ( .... pula uma sentença para tráz
     } .... pula um parágrafo para frente
     { .... pula um parágrafo para tráz
     y) ... copia uma sentença para frente
     d} ... deleta um parágrafo para frente
\end{verbatim}

O que foi deletado ou copiado pode ser colado:

\begin{verbatim}
     p .... cola o que foi copiado ou deletado abaixo
     P .... cola o que foi copiado ou deletado acima
\end{verbatim}

Caso tenha uma estrutura como abaixo:

\begin{verbatim}
     def pot(x):
        return x**2
\end{verbatim}

E tiver uma referência qualquer para a função \verb+pot+ e desejar
mover-se até sua definição basta posicionar o cursor sobre a palavra
\verb+pot+ e pressionar (em modo normal)

\begin{verbatim}
     gd
\end{verbatim}

Se a variável for global, ou seja, estive fora do documento
(provavelmente em outro) use:

\begin{verbatim}
     gD
\end{verbatim}

Quando definimos uma variável tipo

\begin{verbatim}
     var = `teste'
\end{verbatim}

e em algum ponto do documento houver referência a esta variável e
quiser-mos ver seu conteúdo fazemos

\begin{verbatim}
     [i
\end{verbatim}
Na verdade o atalho acima lhe mostrará o último ponto onde foi feita
a atribuição àquela variável que está sob o cursor, uma mão na roda
para os programadores de plantão!

Obs: observe a  barra de status do Vim se o tipo de arquivo está certo,
tipo. Para detalhes sobre como personalizar a barra de status na seção
\ref{Funçã para barra de status}.

\begin{verbatim}
     ft=python
\end{verbatim}

a busca por definições de função só funciona se o tipo de arquivo
estiver correto

\begin{verbatim}
     :set ft=python
\end{verbatim}

outro detalhe para voltar ao último ponto em que você estava

\begin{verbatim}
     ''
\end{verbatim}

\section{Paginando}
\label{Paginando}

Para rolar uma página de cada vez (em modo normal)

\begin{verbatim}
     Ctrl-f
     Ctrl-b
\end{verbatim}


\begin{verbatim}
     :h jumps .... ajuda sobre a lista de saltos
     :jumps ...... exibe a lista de saltos
     Ctrl-i ... salta para a posição mais recente
     Ctrl-o ... salta para a posição mais antiga
\end{verbatim}

Observação: lembre-se

\begin{verbatim}
     ^ .... equivale a Ctrl
     ^I ... equivale a Ctrl-I
\end{verbatim}


Retroceder na lista de saltos, incluindo outros arquivos,

\begin{verbatim}
     ^o
\end{verbatim}

Avançar na lista de saltos

\begin{verbatim}
     ^i
\end{verbatim}

Abrir o último arquivo editado:

\begin{verbatim}
     '0
\end{verbatim}

Abrir o penúltimo arquivo editado

\begin{verbatim}
     '1
\end{verbatim}

Para pular para uma definição de função (para mais detalhes veja :h gd)

\begin{verbatim}
     gd
\end{verbatim}

Para pular para o fim do parágrafo faça

\begin{verbatim}
     }
\end{verbatim}

Para pular para a coluna 10 da linha atual

\begin{verbatim}
     10|
\end{verbatim}

Pular para definição de uma variável

\begin{verbatim}
     [i   ........ Mostra a definição mais próxima de uma variável
\end{verbatim}

O atalho acima é útil quando se está programando, se estiver num
trecho de código pode visualizar o conteúdo das variáveis que foram
definidas acima

Você pode abrir vários arquivos tipo \verb|*.txt| e fazer
algo como gravar e ir para o próximo arquivo com o comando a
seguir:

\begin{verbatim}
     :wn
\end{verbatim}

Ou gravar um arquivo e voltar ao anterior

\begin{verbatim}
     :wp
\end{verbatim}

Pode ainda ``rebobinar'' sua lista de arquivos :)

\begin{verbatim}
     :rew[wind]
\end{verbatim}

Ou ir para o primeiro

\begin{verbatim}
     :fir[ist]
\end{verbatim}

\section{Usando marcadores}
\label{Usando marcadores}

No Vim podemos marcar o ponto em que o cursor está, você deve estar em
modo normal, portanto pressione

\begin{verbatim}
     <Esc>
\end{verbatim}

você estará em modo normal, assim podem pressionar a tecla ``\verb+m+''
seguida de uma das letras do alfabeto

\begin{verbatim}
     ma ....... cria uma marca 'a'
     `a ....... move o cursor para a marca 'a'
\end{verbatim}

\section{Marcas globais}
\label{Marcas globais}
Marcas globais são marcas que permitem pular de um arquivo a outro.
Para criar uma marca global use a letra que designa a marca em
maiúsculo.

\begin{verbatim}
     mA ....... cria uma marca global A
\end{verbatim}

\chapter{Editando}
\label{Editando}

Que tal abrir um arquivo já na linha 10 por exemplo?

\begin{verbatim}
     vim +10 /caminho/para/o/arquivo
\end{verbatim}

Ou ainda abrir na linha que contém um determinado padrão?

\begin{verbatim}
     vim +/padrão arquivo
\end{verbatim}

Obs: caso o padrão tenha espaços no nome coloque entre parênteses ou
use escape ``$\backslash$'' a fim de não obter erro.

\section{Deletando uma parte do texto}\label{Deletando uma parte do texto}

O comando ``d'' remove o conteúdo para a memória.

\begin{verbatim}
     x .... apaga o caractere sob o cursor
     d5x .. apaga os próximos 5 caracteres
     dd  .. apaga a linha atual
     5dd .. apaga 5 linhas (também pode ser: d5d)
     dw  .. apaga uma palavra
     5dw .. apaga 5 palavras (também pode ser: d5w)
     dl  .. apaga uma letra (sinônimo: x)
     5dl .. apaga 5 letras (também pode ser: d5l ou 5x)
     d0  .. apaga até o início da linha
     d^  .. apaga até o primeiro caractere da linha
     d$  .. apaga até o final da linha (sinônimo: D)
     dgg .. apaga até o início do arquivo
     dG  .. apaga até o final do arquivo
     D .... apaga o resto da linha
\end{verbatim}

Depois do texto ter sido colocado na memória, digite `p' para `inserir' o
texto em uma outra posição. Outros comandos:

\begin{verbatim}
     diw .. apaga palavra mesmo que não esteja posicionado no início
     dip .. apaga o parágrafo atual
     d4b .. apaga as quatro palavras anteriores
     dfx .. apaga até o próximo ``x''
     d/casa/+1 - deleta até a linha após a palava casa
\end{verbatim}

Se você trocar a letra `d' nos comandos acima por `c' de {\em change}
(``mudança'') ao invés de deletar será feita uma mudança de conteúdo.  Por
exemplo:

\begin{verbatim}
     ciw .............. modifica uma palavra
     cip .............. modifica um parágrafo
     cis .............. modifica uma sentença
     C ................ modifica até o final da linha
\end{verbatim}

\section{Copiando sem deletar}\label{Copiando sem deletar}

O comando ``y'' ({\em yank}) permite copiar uma parte do texto para a memória sem deletar.
Existe uma semelhança muito grande entre os comandos ``y'' e os comandos ``d'':

\begin{verbatim}
     yy  .... copia a linha atual (sinônimo: Y)
     5yy .... copia 5 linhas (também pode ser: y5y ou 5Y)
     y/pat .. copia até `pat'
     yw  .... copia uma palavra
     5yw .... copia 5 palavras (também pode ser: y5w)
     yl  .... copia uma letra
     5yl .... copia 5 letras (também pode ser: y5l)
     y^  .... copia da posição atual até o início da linha (sinônimo: y0)
     y$  .... copia da posição atual até o final da linha
     ygg .... copia da posição atual até o início do arquivo
     yG  .... copia da posição atual até o final do arquivo
\end{verbatim}

Digite ``P'' (p maiúsculo) para colar o texto recém copiado na posição onde
encontra-se o cursor, ou ``p'' para colar o texto na posição imediatamente
após o cursor.

\begin{verbatim}
     vip .... adiciona seleção visual ao parágrafo atual 'inner paragraph'
     yip .... copia o parágrafo atual
     yit .... copia a tag agual `inner tag' útil para arquivos html xml
\end{verbatim}

\section{Lista de alterações}
O Vim mantém uma lista de alterações, para avançar nas alterações use

\begin{verbatim}
     g,
\end{verbatim}

Para recuar nas alterações

\begin{verbatim}
     g;
\end{verbatim}

Para visualizar a lista de alterações

\begin{verbatim}
     :changes
\end{verbatim}

Para mais detalhes

\begin{verbatim}
     :h changes
\end{verbatim}


\section{Forçando a edição de um novo arquivo}\label{sec:Forçando a edição de um novo arquivo}

O Vim como qualquer outro editor é muito exigente no que se refere a alterações
de arquivo.  Se você estiver editando um arquivo e desejar abandona-lo, o Vim
perguntará se quer salvar alterações, se você estiver certo de que não quer
salvar o arquivo atual e deseja imediatamente começar a editar um novo arquivo
faça:

\begin{verbatim}
     :enew!
\end{verbatim}

O comando acima é uma abreviação de {\em edit new} De modo similar você pode
desejar ignorar todas as alterações feitas desde a abertura do arquivo

\begin{verbatim}
     :e!
\end{verbatim}


\section{Ordenando}

O Vim 7 passa a ter um comando de ordenação que também retira linhas
duplicadas

\begin{verbatim}
     :sort u ... ordena e retira linhas duplicadas
     :sort n ... ordena numericamente
\end{verbatim}

Obs: a ordenação numérica é diferente da ordenação alfabética se em um
trecho contendo algo como:

\begin{verbatim}
     8
     9
     10
     11
     12
\end{verbatim}

Você tentar fazer:

\begin{verbatim}
     :sort
\end{verbatim}

O Vim colocará nas três primeiras linhas

\begin{verbatim}
     10
     11
     12
\end{verbatim}

Portanto lembre-se que se a ordenação envolver números use:

\begin{verbatim}
     :sort n
\end{verbatim}

Você pode fazer a ordenação em um intervalo assim:

\begin{verbatim}
     :1,15 sort n
\end{verbatim}

O comando acima diz: Ordene numericamente da linha 1 até a linha 15

\section{Removendo linhas duplicadas}

\begin{verbatim}
     :sort u
\end{verbatim}


\section{Editando em modo de comando}\label{sec:Editando em modo de comando}

Para mover um trecho usando o modo de comandos faça:

\begin{verbatim}
     :10,20m $
\end{verbatim}

O comando acima move (`m') da linha 10 até a linha 20 para o final \verb|$|.

\begin{verbatim}
     :g /palavra/ m 0
\end{verbatim}

Move as linhas contendo 'palavra' para o começo (linha zero)


\begin{verbatim}
     :10,20y a
\end{verbatim}

Copia da linha `10' até a linha `20' para o registro `a'

\begin{verbatim}
     :56pu a
\end{verbatim}

Cola o registro `a' na linha 56

\begin{verbatim}
     :g/padrão/d
\end{verbatim}

O comando acima deleta todas as linhas contendo a palavra `padrão'

Podemos inverter a lógica do comando global \verb+g+:

\begin{verbatim}
     :g!/padrão/d
\end{verbatim}


Não delete as linhas contendo padrão, ou seja, delete tudo menos as linhas
contendo a palavra `padrão'. Para ler mais sobre o comando ``global'' utilizado 
nesta seção veja o capítulo \ref{sec:O comando global ``g''}.

\begin{verbatim}
     :7,10copy $
\end{verbatim}

Da linha 7 até a linha 10 copie para o final
Veja mais sobre edição no modo de comando na seção ``\ref{Buscas e
substituições} Buscas e substituições''.

\section{O arquivo alternativo}
\label{O arquivo alternativo}

É muito comum um usuário concluir a edição em um arquivo no Vim e
inocentemente imaginar que não vai mais modificar qualquer coisa nele, então
este usuário abre um novo arquivo:

\begin{verbatim}
     :e novo-arquivo.txt
\end{verbatim}

Mas de repente o usuário lembra que seria necessário adicionar uma linha no
arquivo recém editado, neste caso usa-se o atalho

\begin{verbatim}
     Ctrl-6
\end{verbatim}

cuja função é alternar entre o arquivo atual e o último editado. Para retornar
ao outro arquivo basta portanto pressionar \verb|Ctrl-6| novamente.

\section{Incrementando números em modo normal}\label{Incrementando números em modo normal}
Posicione o cursor sobre um número e pressione

\begin{verbatim}
     Ctrl-a ..... incrementa o número
     Ctrl-x ..... decrementa o número
\end{verbatim}

\section{Repetindo a digitação de linhas}
\label{Repetindo a digitação de linhas}

\begin{verbatim}
     Ctrl-y ......... repete linha acima
     Ctrl-e ......... repete linha abaixo
     Ctrl-x Ctrl-l .. repete linhas inteiras
     Ctrl-a ......... repete a última inserção
\end{verbatim}

Para saber mais sobre repetição de comandos veja o capítulo \ref{Repetição de comandos},
na página \pageref{Repetição de comandos}.

\section{Movendo um trecho de forma inusitada}
\label{Movendo um trecho de forma inusitada}

\begin{verbatim}
     :20,30m 0 ..... move da linha `20' até `30' para o começo
     :20,/pat/m 5 .. move da linha `20' até `pat' para a linha 5
\end{verbatim}


\section{Uma calculadora diferente}
\label{Uma calculadora diferente}
Sempre que desejar inserir um cálculo você pode usar o atalho

\begin{verbatim}
     Ctrl-r=
     Ctrl-r=5*850
\end{verbatim}


\section{Desfazendo}
\label{Desfazendo}

Se você cometer um erro, não se preocupe! Use o comando ``u'':

\begin{verbatim}
     u ............ desfazer
     U ............ desfaz mudanças na última linha editada
     Ctrl-r  ...... refazer
\end{verbatim}

Para mais ajuda sobre ``desfazer'':

\begin{verbatim}
     :help undo
\end{verbatim}

\subsection{Undo tree}
\label{Undo tree}

Um novo recurso muito interessante foi adicionado ao Vim ``a partir da
versão 7''  é a chamada árvore do desfazer.  Se
você desfaz alguma coisa, fez uma alteração um novo {\em branch} ou
galho, derivação de alteração é criado.  Basicamente, os {\em branches}
nos permitem acessar quaisquer alterações ocorridas no arquivo.

\subsubsection{Um exemplo didático}
\label{Um exemplo didático}

Siga estes passos (para cada passo \verb|<Esc>|, ou seja, saia do modo
de inserção)


\begin{description}
\item [Passo 1] - digite na linha 1 o seguinte texto
\begin{verbatim}
     # controle de fluxo <Esc>
\end{verbatim}

\item [Passo 2] - digite na linha 2 o seguinte texto
\begin{verbatim}
     # um laço for <Esc>
\end{verbatim}

\item [Passo 3] - Nas linhas 3 e 4 digite...

\begin{verbatim}
     for i in range(10):
         print i  <Esc>
\end{verbatim}

\item [Passo 4] - pressione ``u'' duas vezes (você voltará ao passo 1)
\item [Passo 5] - Na linha 2 digite

\begin{verbatim}
     # operador ternário <Esc>
\end{verbatim}

\item [Passo 6] - na linha 3 digite

\begin{verbatim}
     var = (1 if teste == 0 else 2)  <Esc>
\end{verbatim}

\end{description}

Obs: A necessidade do {\tt Esc} é para demarcar as ações, pois o Vim
considera cada inserção uma ação.  Agora usando o atalho de desfazer
tradicional ``u'' e de refazer {\tt Ctrl-r} observe que não é mais possível
acessar todas as alterações efetuadas. Em resumo, se você fizer uma
nova alteração após um desfazer (alteração derivada) o comando refazer
não mais vai ser possível para aquele momento. \\

Agora volte até a alteração 1 e use seguidas vezes:

\begin{verbatim}
     g+
\end{verbatim}

e / ou

\begin{verbatim}
     g-
\end{verbatim}

Dessa forma você acessará todas as alterações ocorridas no texto!

\section{Salvando}

A maneira mais simples de salvar um arquivo, é usar o comando

\begin{verbatim}
     :w
\end{verbatim}


Para especificar um novo nome para o arquivo, simplesmente digite

\begin{verbatim}
     :w! >> ``file''
\end{verbatim}

O conteúdo será gravado no arquivo ``file'' e você continuará no arquivo original.

Também existe o comando

\begin{verbatim}
     :saveas nome
\end{verbatim}

salva o arquivo com um novo nome e muda para esse novo arquivo (o arquivo original não é apagado).
Para sair do editor, salvando o arquivo atual, digite :x (ou :wq).

\begin{verbatim}
     :w ............................ salva
     :w 'novonome' ................. salvar como
     :wq  .......................... salva e sai'
     :saveas nome .................. salvar como
     :x ............................ salva se existirem modificações
     :10,20 w! ~/Desktop/teste.txt . sava um trecho para outro arquivo
     :w! ........................... salvamento forçado
     :e! ........................... reinicia a edição ignorando alterações
\end{verbatim}

Para mais informações, digite:

\begin{verbatim}
     :help writing
\end{verbatim}

\section{Usando marcas}
\label{sec:Usando marcas}

As marcas são um meio eficiente de se pular para um local no arquivo. Para
criar uma,  estando em modo normal faça:

\begin{verbatim}
     ma
\end{verbatim}

Onde ``m'' indica a criação de uma marca e ``a'' é o nome da marca. Para pular para a marca ``a'' faça:

\begin{verbatim}
     `a
\end{verbatim}

Para voltar ao ponto do último salto

\begin{verbatim}
     ''
\end{verbatim}

Para deletar de até a marca ``a'' (em modo normal)

\begin{verbatim}
     d'a
\end{verbatim}

\section{Marcas globais}
Durante a edição de vários arquivos defina uma marca global com o comando

\begin{verbatim}
     mA
\end{verbatim}

Onde ``m'' cria a marca e ``A'' (maiúsculo) define uma marca ``A'' acessível a qualquer momento com o comando

\begin{verbatim}
     'A
\end{verbatim}

Isto fará o Vim dar um salto até a marca A mesmo que esteja em outro
arquivo, mesmo que você tenha acabado de fecha-lo. Para abrir vários
arquivos uma solução seria:

\begin{verbatim}
     vim *.txt
\end{verbatim}

Para ir para o próximo arquivo:

\begin{verbatim}
     :bn
\end{verbatim}

Para ir para o arquivo anterior

\begin{verbatim}
     :bp
\end{verbatim}

Caso existam modificações no arquivo você terá que salvar antes, veja: 

\begin{verbatim}
     :wn
\end{verbatim}

O comando acima diz: grave e vá para o próximo!


\subsection{Abrindo o último arquivo rapidamente}
O Vim guarda um registro para cada arquivo editado veja
mais no capítulo \ref{Registros} na página \pageref{Registros}.

\begin{verbatim}
     '0 ........ abre o último arquivo editado
     '1 ........ abre o penúltimo arquivo editado
     Ctrl-6 .... abre o arquivo alternativo (booleano)
\end{verbatim}

Bom, já que abrimos o nosso último arquivo editado com o comando

\begin{verbatim}
     `0
\end{verbatim}

podemos, e provavelmente o faremos, editar no mesmo ponto em que estávamos
editando da última vez

\begin{verbatim}
     gi
\end{verbatim}

Na seção \ref{Buscas e substituições} você encontra mais dicas de edição!


\subsection{Modelines}\label{sec:Modelines}

São um modo de guardar preferências no próprio arquivo, suas
preferências viajam literalmente junto com o arquivo, basta usar em
uma das 5 primeiras linhas ou na última linha do arquivo algo
como:

\begin{verbatim}
     # vim:ft=sh:
\end{verbatim}

OBS: Você deve colocar um espaço entre a palavra `vim' e a primeira
coluna, ou seja, a palavra `vim' deve vir precedida de um espaço, daí
em diante cada opção fica assim:

\begin{verbatim}
     :opção:
\end{verbatim}

Por exemplo: posso salvar um arquivo com extensão \verb|.sh| e dentro do
mesmo indicar no {\em modeline} algo como:

\begin{verbatim}
     # vim:ft=txt:nu:
\end{verbatim}

Apesar de usar a extensão `sh' o Vim reconhecerá este arquivo como `txt', e
caso eu não tenha habilitado a numeração, ainda assim o Vim usará por causa da
opção `nu'.  Portanto o uso de {\em modelines} pode ser um grande recurso para o seu
dia-a-dia pois você pode coloca-las dentro dos comentários!

\section{Edição avançada de linhas}

Seja o seguinte texto:

\begin{verbatim}
1  este é um texto novo
2  este é um texto novo
3  este é um texto novo
4  este é um texto novo
5  este é um texto novo
6  este é um texto novo
7  este é um texto novo
8  este é um texto novo
9  este é um texto novo
10 este é um texto novo
\end{verbatim}

Suponha que queira-se apagar ``é um texto'' da linha 5 até o fim (linha 10). Isto pode ser feito
assim: 

\begin{verbatim}
     :5,$ normal 0wd3w
\end{verbatim}

Explicando o comando acima:

\begin{verbatim}
     :5,$ .... indica o intervalo que é da linha 5 até o fim ``$''
     normal .. executa em modo normal
     0 ....... move o cursor para o começo da linha
     w ....... pula uma palavra
     d3w ..... apaga 3 palavras ``w''
\end{verbatim}

Obs: É claro que um comando de substituição simples

\begin{verbatim}
     :5,$s/é um texto//g
\end{verbatim}

Resolveria neste caso, mas a vantagem do método anterior é que
é válido para três palavras, sejam quais forem.\\

Também é possível empregar comandos de inserção (como {\em i} ou {\em a}) e
retornar ao modo normal, bastando para isso usar o recurso \verb|Ctrl-v Esc|,
de forma a simular o acionamento da tecla \verb|Esc| (saída do modo de
inserção). Por exemplo, suponha agora que deseja-se mudar a frase ``{\em este
é um texto novo}'' para ``{\em este não é um texto velho}''; pode ser feito
assim:

\begin{verbatim}
     :5,$ normal 02winão ^[$ciwvelho
\end{verbatim}

Decompondo o comando acima temos:

\begin{verbatim}
     :5,$ .... indica o intervalo que é da linha 5 até o fim ``$''
     normal .. executa em modo normal
     0 ....... move o cursor para o começo da linha
     2w ...... pula duas palavras (vai para a palavra "é")
     i ....... entra no modo de inserção
     não  .... insere a palavra "não" seguida de espaço " "
     ^[ ...... sai do modo de inserção (através de Ctrl-v seguido de Esc)
     $ ....... vai para o fim da linha
     ciw ..... apaga a última palavra ("novo") e entra em modo de inserção
     velho ... insere a palavra "velho" no lugar de "novo"
\end{verbatim}

A combinação \verb|Ctrl-v| é utilizada para inserir caracteres de controle na
sua forma literal, prevenindo-se assim a interpretação destes neste exato
momento.
\chapter{Folders}\label{cha:Folders}
{\em Folders} são como dobras nas quais o Vim esconde partes do texto,
algo assim:

\begin{verbatim}
     +-- 10 linhas ---------------------------
\end{verbatim}

Deste ponto em diante chamaremos os {\em folders} descritos no manual do
Vim como dobras!  Quando tiver que manipular grandes quantidades de
texto tente usar dobras, isto permite uma visualização completa do
texto.  Um modo de entender rapidamente como funcionam as dobras no
Vim seria criando uma ``dobra'' para as próximas 10 (dez) linhas com o
comando abaixo:

\begin{verbatim}
     zf10j
\end{verbatim}

Você pode ainda criar uma seleção visual

\begin{verbatim}
     Shift-v ............ seleção por linha
     j .................. desce linha
     zf ................. cria o folder
     zo ................. abre o folder
\end{verbatim}

\section{Métodos de dobras }
\label{Métodos de dobras }
O Vim tem seis modos {\em fold}, são eles:

\begin{itemize}
\item Sintaxe ({\em syntax})
\item Indentação ({\em indent})
\item Marcas ({\em marker})
\item Manual
\item Diferenças ({\em diff})
\item expresões ({\em Expressões Regulares})
\end{itemize}

Para determinar o tipo de dobra faça

\begin{verbatim}
     :set foldmethod=tipo
\end{verbatim}

onde o tipo pode ser um dos tipos listados acima, exemplo:

\begin{verbatim}
     :set foldmethod=marker
\end{verbatim}

Outro modo para determinar o método de dobra seria colocando na última
linha do seu arquivo algo assim:

\begin{verbatim}
     vim:fdm=marker:fdl=0:
\end{verbatim}

Obs: \verb|fdm| significa {\em foldmethod}, e \verb|fdl| significa
{\em foldlevel}. Deve haver um espaço entre a palavra inicial ``vim'' e o
começo da linha este recurso chama-se {\em modeline}, leia mais na seção
``\ref{sec:Modelines} modelines'' na página \pageref{sec:Modelines}.

\section{Manipulando dobras }\label{Manipulando dobras }

\begin{verbatim}
     zo .......... abre a dobra
     zO .......... abre a dobra, recursivamente
     za .......... abre/fecha (alterna) a dobra
     zA .......... abre/fecha (alterna) a dobra, recursivamente
     zR .......... abre todas as dobras do arquivo atual
     zc .......... fecha uma dobra
     zC .......... fecha a dobra abaixo do cursor, recursivamente
     zfap ........ cria uma dobra para o parágrafo `ap' atual
     zf/casa ..... cria uma dobra até a palavra casa
     zf'a ........ cria uma dobra até a marca `a'
     zd .......... apaga a dobra (não o seu conteúdo)
     zj .......... move para o início da próxima dobra
     zk .......... move para o final da dobra anterior
     [z .......... move o cursor para início da dobra aberta
     ]z .......... move o cursor para o fim da dobra aberta
     zi .......... desabilita ou habilita as dobras
     zm, zr ...... diminui/aumenta nível da dobra 'fdl'
     :set fdl=0 .. nível da dobra 0 (foldlevel)
\end{verbatim}

Para abrir e fechar as dobras usando a barra de
espaços coloque o trecho abaixo no seu arquivo de configuração do Vim
\verb|.vimrc| - veja \ref{cha:Como editar preferências no Vim}.

\begin{verbatim}
     nnoremap <space> @=((foldclosed(line(".")) < 0) ? 'zc' : 'zo')<CR>
\end{verbatim}

\section{Criando dobras usando o modo visual}
\label{Criando folders usando o modo visual}
Para iniciar a seleção visual

\begin{verbatim}
     Esc ........ vai para o modo normal
     shift-v .... inicia seleção visual
     j .......... aumenta a seleção visual (desce)
     zf ......... cria a dobra na seleção ativa
\end{verbatim}

Um modo inusitado de se criar dobras é:

\begin{verbatim}
     Shift-v ..... inicia seleção visual
     /chapter/-2 . extende a seleção até /chapter -2 linhas
     zf .......... cria a dobra
\end{verbatim}

\chapter{Registros}
\label{Registros}

O Vim possui nove tipos de registros, cada tipo tem uma utilidade
específica, por exemplo você pode usar um registro que guarda o último
comando digitado, pode ainda imprimir dentro do texto o nome do
próprio arquivo, vamos aos detalhes.

\begin{itemize}
   \item O registro sem nome ``''
   \item 10 registros nomeados de ``9''
   \item O registro de pequenas deleções "-
   \item 26 registros nomeados de ``z'' ou de ``Z''
   \item 4 registros somente leitura
   \item O registro de expressões "=
   \item Os registro de seleção e  "*, "+ and "~
   \item O registro ``o''
   \item Registro do último padrão de busca "/
\end{itemize}

\section{O registro sem nome ``''}
\label{O registro sem nome ``''}

Armazena o conteúdo de ações como:

\begin{verbatim}
     d ....... deleção
     s ....... substituição
     c ....... um outro tipo de modificação
     x ....... apaga um caractere
     yy ...... copia uma linha inteira
\end{verbatim}

Para acessar o conteúdo deste registro basta usar as letras ``{\tt p}'' ou ``{\tt P}'' que
na verdade são comandos para colar abaixo da linha atual e acima da
linha atual (em modo normal)

\section{Registros nomeados de 0 a 9}
\label{Registros nomeados de 0 a 9}

O registro zero armazena o conteúdo da última cópia `yy', à partir do
registro 1 vão sendo armazenadas as deleções sucessivas de modo que a
mais recente deleção será armazenada no registro 1 e os registros vão
sendo incrementados em direção ao nono.  Deleção menores que uma linha
não são armazenadas nestes registros, caso em que o Vim usa o registro
de pequenas deleções ou que se tenha especificado algum outro
registro.

\begin{verbatim}
     :help registers
\end{verbatim}

\section{Registro de pequenas deleções}
\label{Registro de pequenas deleções}
Quando você {\em deleta} algo menor que uma linha o Vim armazena os dados deletados neste registro.

\section{Registros nomeados de ``a até z'' ou ``A até Z''}
\label{Registros nomeados de ``a até z'' ou ``A até Z''}
Você pode armazenar uma linha em modo normal assim:

\begin{verbatim}
     "ayy
\end{verbatim}

Desse modo você guardou o conteúdo da linha no registro ``a'' caso
queira armazenar mais uma linha no registro ``a'' use este comando

\begin{verbatim}
     "Add
\end{verbatim}

Neste outro caso apaguei a linha corrente adicionando-a ao final do registro ``a''.

\begin{verbatim}
     "ayip .. copia o parágrafo atual para o registro ``a''
     "a ..... registro a
     y ...... yank (copia)
     ip ..... inner paragraph (este parágrafo)
\end{verbatim}

\section{Registros somente leitura ``: . \% \#''}
\label{Registros somente leitura}

\begin{verbatim}
     ": ..... armazena o último comando
     ". ..... armazena uma cópia do último texto inserido
     "% ..... contém o nome do arquivo corrente
     "# ..... contém o nome do arquivo alternativo
\end{verbatim}

Uma forma prática de usar registros em modo de inserção é usando
\verb|Ctrl-r|


\begin{verbatim}
     Ctrl-r % .... insere o nome do arquivo atual
     Ctrl-r : .... insere o último comando digitado
     Ctrl-r / .... insere a última busca efetuada
     Ctrl-r a .... insere o registro `a'
\end{verbatim}

Em modo de inserção você pode repetir a última inserção de texto
simplesmente pressionando

\begin{verbatim}
     Ctrl-a
\end{verbatim}

\section{Registro de expressões}
\label{Registro de expressões}

\begin{verbatim}
     "=
\end{verbatim}

Este registro na verdade é usado em algumas funções avançadas.

\section{Registros de arrastar e mover}
\label{Registros de arrastar e mover}

O registro 
\begin{verbatim}
     "*
\end{verbatim}
 é responsável por armazenar o último texto selecionado (p.e., através do
mouse). Já o registro 
\begin{verbatim}
     "+
\end{verbatim}
é o denominado ``área de transferência'', normalmente utilizado para se
transferir conteúdos entre aplicações---este registro é preenchido, por
exemplo, usando-se a típica combinação {\tt Ctrl-v} encontrada em muitas
aplicações. Finalmente, o registro 
\begin{verbatim}
     "~
\end{verbatim}
armazena o texto colado pela operação mais recente de ``arrastar-e-soltar''
({\em drag-and-drop}). 

\section{Registro buraco negro "\_}
\label{Registro buraco negro}
Use este registro quando não quiser alterar os demais registros, por exemplo: se você deletar a linha atual,

\begin{verbatim}
     dd
\end{verbatim}

Esta ação irá colocar a linha atual no registro numerado 1, caso não
queira alterar o conteúdo do registro 1 apague para o buraco negro
assim:

\begin{verbatim}
     "_dd
\end{verbatim}

\section{Registros de buscas ``/''}
\label{Registros de buscas ``/''}

Se desejar inserir em uma substituição uma busca prévia, você poderia
fazer assim em modo de comandos:

\begin{verbatim}
     :%s,<Ctrl-r>/,novo-texto,g
\end{verbatim}

Observação: veja que estou trocando o delimitador da busca para deixar
claro o uso do registro de buscas ``/''

\section{Manipulando registros}
\label{Manipulando registros}

\begin{verbatim}
     :let @a=@_     : limpa o registro a
     :let @a=``''   : limpa o registro a
     :let @a=@"     : salva registro sem nome *N*
     :let @*=@a     : copia o registro para o buffer de colagem
     :let @*=@:     : copia o ultimo comando para o buffer de colagem
     :let @*=@/     : copia a última busca para o buffer de colagem
     :let @*=@%     : copia o nome do arquivo para o buffer de colagem
     :reg           : mostra o conteúdo de todos os registros
\end{verbatim}

Em modo de inserção

\begin{verbatim}
     <C-R>-         : Insere o registro de pequenas deleções
     <C-R>[0-9a-z]  : Insere registros 0-9 e a-z
     <C-R>%         : Insere o nome do arquivo
     <C-R>=somevar  : Insere o conteúdo de uma variável
\end{verbatim}


Um exemplo: pré-carregando o nome do arquivo no registro \verb+n+.

coloque em seu \verb+~/.vimrc+

\begin{verbatim}
     let @n=@%
\end{verbatim}

Como foi atribuído ao registro \verb+n+ o conteúdo de @\%, ou seja, o nome
do arquivo, você pode fazer algo assim em modo de inserção:

\begin{verbatim}
     Ctrl-r n
\end{verbatim}

E o nome do arquivo será inserido

\section{Listando os registros atuais}
\label{Listando os registros atuais}
Digitando o comando

\begin{verbatim}
     :reg
\end{verbatim}

ou ainda

\begin{verbatim}
     :ls
\end{verbatim}

O Vim mostrará os registros numerados e nomeados atualmente em uso

\section{Listando arquivos abertos}
\label{Listando arquivos abertos}
Suponha que você abriu vários arquivos txt assim:

\begin{verbatim}
     vim *.txt
\end{verbatim}

Para listar os arquivos aberto faça:

\begin{verbatim}
     :buffers
\end{verbatim}

Usando o comando acima o Vim exibirá a lista de todos os arquivos
abertos, após exibir a lista você pode escolher um dos arquivos da
lista, algo como:

\begin{verbatim}
     :buf 3
\end{verbatim}

Para editar arquivos em sequência faça as alterações no arquivo atual
e acesso o próximo assim:

\begin{verbatim}
     :wn
\end{verbatim}

O comando acima diz \verb|`grave' --> w|  e próximo \verb|`next' --> n|

\section{Dividindo a janela com o próximo arquivo da lista de buffers}
\label{Dividindo a janela com o próximo arquivo da lista de buffers}

\begin{verbatim}
     :sn
\end{verbatim}

O comando acima é uma abreviação de {\em split next}, ou seja, dividir e próximo.

\section{Como colocar um pedaço de texto em um registro?}
\label{Como colocar um pedaço de texto em um registro?}

\begin{verbatim}
     <Esc> ...... vai para o modo normal
     "a10j ...... coloca no registro `a' as próximas 10 linhas `10j'
\end{verbatim}

Para usar você pode:

\begin{verbatim}
     <Esc> ...... para ter certeza que está em modo normal
     "ap ........ registro a `paste', ou seja, cole
\end{verbatim}

Em modo de inserção faça:

\begin{verbatim}
     Ctrl-r a
\end{verbatim}

\section{Como criar um registro em modo visual?}
\label{Como criar um registro em modo visual?}
Inicie a seleção visual com o atalho

\begin{verbatim}
     Shift-v ..... seleciona linhas inteiras
\end{verbatim}

pressione a letra ``\verb|j|'' até chegar ao ponto desejado, agora faça

\begin{verbatim}
     "ay
\end{verbatim}

pressione ``\verb|v|'' para sair do modo visual

\section{Como definir um registro no vimrc?}
\label{Como definir um registro no vimrc?}

Se você não sabe ainda como editar preferências no Vim
leia antes o capítulo \ref{cha:Como editar preferências no Vim}. \\


Você pode criar uma variável no vimrc assim:

\begin{verbatim}
     let var="foo" ...... define foo para var
     echo var ........... mostra o valor de var
\end{verbatim}

Pode também dizer ao Vim algo como...

\begin{verbatim}
     :let @d=strftime("c")<Enter>
\end{verbatim}

Neste caso estou dizendo a ele que guarde na variável `d' at d,
o valor da data do sistema `strftime(``c'')' ou então cole isto no
vimrc:

\begin{verbatim}
     let @d=strftime("c")<cr>
\end{verbatim}

A diferença entre digitar diretamente um comando e adiciona-lo ao
vimrc é que uma vez no vimrc o registro em questão estará sempre
disponível, observe também as sutis diferenças, um {\tt Enter} inserido
manualmente é apenas uma indicação de uma ação que você fará
pressionando a tecla especificada, já o comando mapeado vira
``\verb|<cr>|'', veja ainda que no vimrc os dois pontos ``\verb|:|''
somem.

Pode mapear tudo isto

\begin{verbatim}
     let @d=strftime("c")<cr>
     imap ,d <cr-r>d
     nmap ,d "dp
\end{verbatim}

As atribuições acima correspondem a:

\begin{enumerate}
 \item  Guarda a data na variável ``d''
 \item  Mapeamento para o modo de inserção ``imap'' digite ,d
 \item  Mapeamento para o modo normal ``nmap'' digite ,d
\end{enumerate}

E digitar ,d normalmente

Desmistificando o strftime
\begin{verbatim}
     " d=dia m=mes Y=ano H=hora M=minuto c=data-completa
     :h strftime ........ ajuda completa sobre o comando
\end{verbatim}

e inserir em modo normal assim:

\begin{verbatim}
     "dp
\end{verbatim}

ou usar em modo de inserção assim

\begin{verbatim}
     Ctrl-r d
\end{verbatim}

\section{Como selecionar blocos verticais de texto?}
\label{Como selecionar blocos verticais de texto?}

\begin{verbatim}
     Ctrl-v
\end{verbatim}

agora use as letras h,l,k,j como setas de direção até finalizar
podendo guardar a seleção em um registro que vai de ``a'' a ``z'' exemplo:

\begin{verbatim}
     "ay
\end{verbatim}

Em modo normal você pode fazer assim para guardar um parágrafo inteiro em um registro

\begin{verbatim}
     "ayip
\end{verbatim}

O comando acima quer dizer

\begin{verbatim}
     para o registro ``a'' ......  "a
     copie ......................  ``y''
     o parágrafo atual .......... ``inner paragraph''
\end{verbatim}

\section{Referências}
\label{Referências}

\begin{itemize}
   \item \url{http://rayninfo.co.uk/vimtips.html}
   \item \url{http://aprendolatex.wordpress.com}
   \item \url{http://pt.wikibooks.org/wiki/Latex}
\end{itemize}

\chapter{Buscas e substituições}\label{Buscas e substituições}

Para fazer uma busca, certifique-se de que está em modo normal,
pressione ``/'' e digite a expressão a ser procurada. \\


Para encontrar a primeira ocorrência de ``foo'' no texto:

\begin{verbatim}
     /foo
\end{verbatim}

Para repetir a busca basta pressionar a tecla ``\verb+n+'' e para
repetir a busca em sentido oposto ``\verb+N+''.

\begin{verbatim}
     /teste/+3
\end{verbatim}

Posiciona o cursor três linhas após a ocorrência da palavra ``teste'' \\


\section{Usando ``Expressões Regulares'' em buscas}

\begin{verbatim}
     / ........... inicia uma busca (modo normal)
     \%x69 ....... código da letra `i'
     /\%x69 ...... localiza a letra `i' - hexadecimal 069
     
     \d .......... localiza números
     ^ ........... começo de linha
     $ ........... final de linha
     \+ .......... um ou mais
     /^\d\+$ ..... localiza somente dígitos
     
     \s .......... localiza espaços
     /\s\+$ ...... localiza espaços no final da linha 
\end{verbatim}

Para aprender mais sobre Expressões Regulares leia:

\begin{enumerate}
  \item \url{http://guia-er.sourceforge.net}
  \item :help regex
\end{enumerate}

Um meio mais rápido para encontrar a próxima ocorrência de uma palavra sob o
cursor é teclar \verb|*|. Para encontrar uma ocorrência anterior da palavra
sob o cursor, tecle \verb|#|---em ambos os casos o cursor deve estar
posicionado sobre a palavra que deseja procurar.

\section{Editando em nova janela}\label{Editando em nova janela}

Caso deseje manter o arquivo atual e editar 'simultaneamente' outro arquivo
pode dividir a janela assim:

\begin{verbatim}
     Ctrl-w-n
\end{verbatim}

Para mais detalhes sobre janelas acesse o capítulo
\ref{cha:Trabalhando com janelas} na página \pageref{cha:Trabalhando
com janelas}.


\section{Inserindo linha antes e depois}

Suponha que se queira um comando, considere \verb|,t|, que faça com que a
linha indentada corrente passe a ter uma linha em branco antes e depois; isto
pode ser obtido pelo seguinte mapeamento:

\begin{verbatim}
     :map ,t <Esc>:.s/^\(\s\+\)\(.*\)/\r\1\2\r/g<cr>
\end{verbatim}
     
Explicando:
     
\begin{verbatim}
     : ................ entra no modo de comando
     map ,t ........... mapeia ,t para a função desejada
     <Esc> ............ ao executar sai do modo de inserção
     s/isto/aquilo/g .. substitui isto por aquilo
     : ................ inicia o modo de comando
     . ................ na linha corrente
     s ................ substitua
     ^ ................ começo de linha
     \s\+ ............. um espaço ou mais (barras são escapes)
     .* ............... qualquer coisa depois
     \(grupo\) ........ agrupo para referenciar com \1
     \1 ............... repete na substituição o grupo 1
     \r ............... insere uma quebra de linha
     g ................ em todas as ocorrências da linha
     <cr> ............. Enter
\end{verbatim}

\section{Obtendo informações do arquivo}

\begin{verbatim}
     ga ............. mostra o código do caractere em decimal hexa e octal
     ^g ............. mostra o caminho e o nome do arquivo
\end{verbatim}

Obs: O código do caractere pode ser usado para substituições,
especialmente em se tratando de caracteres de controle como tabulações
``\verb|^I|'' ou final de linha DOS/Windows ``\verb|\%x0d|''. Você pode apagar os
caracteres de final de linha Dos/Windows usando uma simples
substituição, veja mais adiante:

\begin{verbatim}
     :%s/\%x0d//g
\end{verbatim}

Na seção \ref{cha:Como editar preferências no Vim} há um código para a barra de
status que faz com que a mesma exiba o código do caractere sob o cursor na
seção \ref{Funçã para barra de status}.

\section{Trabalhando com registradores}
\label{Trabalhando com registradores}

Você não precisa copiar e colar diferentes partes do texto para uma
mesma área de transferência.  Para isso, você pode usar os
``registros''.  Os registradores são indicados por aspas seguido por uma letra.
Exemplos: ``a'', ``b'', ``c'', etc.

Como copiar o texto para um registrador? É simples: basta especificar
o nome do registrador antes:

\begin{verbatim}
     "add ... apaga uma linha, copiando seu conteúdo para o registrador a
     "bdd ... apaga uma linha, copiando seu conteúdo para o registrador b
     "ap .... cola" o conteúdo do registrador a
     "ab .... cola" o conteúdo do registrador b
     "x3dd .. apaga 3 linhas, copiando o conteúdo para o registrador ``x''
     "ayy  .. copia uma linha, sem apagar, para o registrador a
     "a3yy .. copia 3 linhas, sem apagar, para o registrador a
     "ayw  .. copia uma palavra, sem apagar, para o registrador a
     "a3yw .. copia 3 palavras, sem apagar, para o registrador a
\end{verbatim}

No ``modo de inserção'', como visto anteriormente, você pode usar um atalho
para colar rapidamente o conteúdo de um registrador.

\begin{verbatim}
     Ctrl-r (registro)
\end{verbatim}

Para colar o conteúdo do registrador ``a''

\begin{verbatim}
     Ctrl-r a
\end{verbatim}

Para copiar a linha atual para a área de transferência

\begin{verbatim}
     "+yy
\end{verbatim}

Para colar da área de transferência

\begin{verbatim}
     "+p
\end{verbatim}

\section{Edições complexas }
\label{Edições complexas }

Trocando palavras de lugar: coloque o cursor no espaço antes da 1ª palavra e digite:

\begin{verbatim}
     deep
\end{verbatim}

Trocando letras de lugar:

\begin{verbatim}
     xp
\end{verbatim}

Trocando linhas de lugar:

\begin{verbatim}
     ddp
\end{verbatim}

Tornando todo o texto maiúsculo
 gggUG

\section{Indentando }

\begin{verbatim}
     >> ..... Indenta a linha atual
     ^T ..... Indenta a linha atual em modo de inserção
     ^D ..... Remove indentação em modo de inserção
     >ip .... indenta o parágrafo atual
\end{verbatim}

\section{Corrigindo a indentação de códigos}
\label{Corrigindo a indentação de códigos}
Selecione o bloco de código, por exemplo

\begin{verbatim}
     vip  ..... visual ``h'' (selecione este parágrafo)
     =  ....... corrija a indentação do que selecionei :)
\end{verbatim}

\section{Usando o file explorer}
\label{Usando o file explorer}
O Vim navega na árvore de diretórios com o comando

\begin{verbatim}
     vim .
\end{verbatim}

Use o ``j'' para descer e o ``k'' para subir ou Enter para editar o
arquivo selecionado. Outra dica é pressionar F1 ao abrir o
FileExplorer do Vim, você encontra dicas adicionais sobre este modo de
operação do Vim.

\section{Selecionando ou deletando conteúdo de tags html}
\label{Selecionando ou deletando conteúdo de tags html}

\begin{verbatim}
     <tag> conteúdo da tag </tag>
     basta usar (em modo normal) as teclas
     vit ............... visual ``inner tag | esta tag''
\end{verbatim}

Este recurso também funciona com parênteses

\begin{verbatim}
     vi( ..... visual select
     vi" ..... visual select
     di( ..... delete inner (, ou seja, seu conteúdo
\end{verbatim}


\section{Substituições }
\label{Substituições }

Para fazer uma busca, certifique-se de que está em modo normal, em
seguida digite use o comando ``s'', conforme será explicado.

Para substituir ``foo'' por ``bar'' na linha atual:

\begin{verbatim}
     :s/foo/bar
\end{verbatim}

Para substituir ``o'' por ``r'' da primeira à décima linha do arquivo:

\begin{verbatim}
     :1,10 s/foo/bar
\end{verbatim}

Para substituir ``foo'' por ``bar'' da primeira à última linha do arquivo:

\begin{verbatim}
     :1,$ s/foo/bar
\end{verbatim}

Ou simplesmente:

\begin{verbatim}
     :% s/foo/bar
\end{verbatim}

\begin{verbatim}
     $ ... significa para o Vim final do arquivo
     % ... representa o arquivo atual
\end{verbatim}

O comando ``s'' possui muitas opções que modificam seu comportamento.

\section{Exemplos }
\label{Exemplos }

Busca usando alternativas:

\begin{verbatim}
     /end\(if\|while\|for\)
\end{verbatim}

Buscará ``if'', ``while'' e ``for''.  Observe que é necessário `escapar' os
caracteres \verb|\(|, \verb|\|| e \verb|\)|, caso contrário eles serão
interpretados como caracteres comuns.

Quebra de linha

\begin{verbatim}
     /quebra\nde linha
\end{verbatim}

Ignorando maiúsculas e minúsculas

\begin{verbatim}
     /\cpalavra
\end{verbatim}

Usando \verb|\c| o Vim encontrará ``{\em{palavra}}'', ``{\em{Palavraa}}'' ou
até mesmo ``{\em{PALAVRA}}''. Uma dica é colocar no seu arquivo de
configuração ``vimrc'' veja o capítulo \ref{cha:Como editar preferências no Vim}.

\begin{verbatim}
     set ignorecase .. ignora maiúsculas e minúsculas na bucsca
     set smartcase ... se busca contiver maiúsculas ele passa a considerá-las
     set hlsearch .... mostra o que está sendo buscado em cores
     set incsearch ... ativa a busca incremental
\end{verbatim}

se você não sabe ainda como colocar estas preferências no arquivo de configuração pode
ativa-las em modo de comando precedendo-as com dois pontos, assim:

\begin{verbatim}
     :set ignorecase<Enter>
\end{verbatim}

Procurando palavras repetidas

\begin{verbatim}
     /\<\(\w*\) \1\>
\end{verbatim}

Multilinha

\begin{verbatim}
     /Hello\_s\+World
\end{verbatim}

Buscará `World', separado por qualquer número de espaços,
incluindo quebras de linha. Buscará as três seqüências:

\begin{verbatim}
     Hello World
     
     Hello    World
     
     Hello
     World
\end{verbatim}

Buscar linhas de até 30 caracteres de comprimento

\begin{verbatim}
     /^.\{,30\}$
\end{verbatim}

\begin{verbatim}
     ^ ...... representa começo de linha
\end{verbatim}

Apaga todas as tags html/xml de um arquivo

\begin{verbatim}
     :%s/<[^>]*>//g
\end{verbatim}

Apaga linhas vazias

\begin{verbatim}
     :%g/^$/d
\end{verbatim}

Ou

\begin{verbatim}
     :%s/^[\ \t]*\n//g
\end{verbatim}

Remover duas ou mais linhas vazias entre parágrafos diminuindo para
uma só linha vazia.

\begin{verbatim}
     :%s/\(^\n\{2,}\)/\r/g
\end{verbatim}

Você pode criar um mapeamento e colocar no seu ~/.vimrc

\begin{verbatim}
     map ,s <Esc>:%s/\(^\n\{2,}\)/\r/g<cr>
\end{verbatim}

No exemplo acima, ``,s'' é um mapeamento para reduzir linhas em branco
sucessivas para uma só  \\


Remove não dígitos (não pega números)

\begin{verbatim}
     :%s/^\D.*//g
\end{verbatim}

Remove final de linha DOS/Windows \verb|^M| que tem código hexadecimal igual a
``0d''

\begin{verbatim}
     :%s/\%x0d//g
\end{verbatim}

Troca palavras de lugar usando expressões regulares

\begin{verbatim}
     :%s/\(.\+\)\s\(.\+\)/\2 \1/
\end{verbatim}

Modificando todas as tags html para minúsculo

\begin{verbatim}
     :%s/<\([^>]*\)>/<\L\1>/g
\end{verbatim}

Move linhas 10 a 12 para além da linha 30

\begin{verbatim}
     :10,12m30
\end{verbatim}

\section{O comando global ``g''}\label{sec:O comando global ``g''}

buscando um padrão e gravando em outro arquivo

\begin{verbatim}
     :'a,'b g/^Error/ . w >> errors.txt
\end{verbatim}

Apenas imprimir linhas que contém determinada palavra, isto é útil 
quando você quer ter uma visão sobre um determina aspecto 
do seu arquivo vejamos:

\begin{verbatim}
     :set nu ..... habilita numeração 
     :g/Error/p .. apenas mostra as linhas correspondentes
\end{verbatim}

numerar linhas

\begin{verbatim}
     :let i=1 | g/^/s//\=i."\t"/ | let i=i+1
\end{verbatim}

Para copiar linhas começadas com {\em Error} para o final do arquivo faça:

\begin{verbatim}
     :g/^Error/ copy $
\end{verbatim}

Obs: O comando {\em copy} pode ser abreviado `co' ou ainda você pode usar `t'
para mais detalhes leia

\begin{verbatim}
     :h co
\end{verbatim}

Entre as linhas que contiverem ``fred'' e ``joe'' substitua

\begin{verbatim}
     :g/fred/,/joe/s/isto/aquilo/gic
\end{verbatim}

As opções `gic' correspondem a {\em global}, {\em ignore case} e {\em
confirm}, podendo ser omitidas deixando só o {\em global}. \\


Pegar caracteres numéricos e jogar no final do arquivo?

\begin{verbatim}
     :g/^\d\+.*/m $
\end{verbatim}

Inverter a ordem das linhas do arquivo?

\begin{verbatim}
     :g/^/m0
\end{verbatim}

Apagar as linhas que contém {\em Line commented}

\begin{verbatim}
     :g/Line commented/d
\end{verbatim}

Copiar determinado padrão para um registro

\begin{verbatim}
     :g/pattern/ normal "Ayy
\end{verbatim}

Copiar linhas que contém um padrão e a linha subsequente para o final

\begin{verbatim}
     :g/padrão/;+1 copy $
\end{verbatim}

\section{Dicas }
Para colocar a última busca em uma substituição faça:

\begin{verbatim}
     :%s/Ctrl-r//novo/g
\end{verbatim}

A dupla barra corresponde ao ultimo padrão procurado, e portanto o
comando abaixo fará a substituição da ultima busca por casinha

\begin{verbatim}
     :%s//casinha/g
\end{verbatim}

\section{Filtrando arquivos com o vimgrep}
\label{Filtrando arquivos com o vimgrep}

Por vezes sabemos que aquela anotação foi feita, mas no momento esquecemos em qual
arquivo está, no exemplo abaixo procuramos a palavra dicas à partir da nossa pasta pessoal
pela palavra `dicas' em todos os arquivos com extensão `txt'.

\begin{verbatim}
     ~/ ............ equivale a /home/user
     :lvimgrep /dicas/ ~/**/*.txt | ls
\end{verbatim}


\section{Copiar a partir de um ponto}

\begin{verbatim}
     :19;+3 co $
\end{verbatim}

O Vim sempre necessita de um intervalo (inicial e final) mas se você
usar ``;'' ele considera a primeira linha como segundo ponto do
intervalo, e no caso acima estamos dizendo (nas entrelinhas) linhas
19 e 19+3     \\


De forma análoga podemos usar como referência um padrão qualquer

\begin{verbatim}
     :/palavra/;+10 m 0
\end{verbatim}

O comando acima diz: à partir da linha que contém ``palavra'' incluindo as 10 próximas linhas
mova ``m'' para a primeira linha ``0'', ou seja, antes da linha 1.

\section{Dicas das lista vi-br}

 Fonte: \url{http://groups.yahoo.com/group/vi-br/message/853}

 Problema:
 Essa deve ser uma pergunta comum.
 Suponha o seguinte conteúdo de arquivo:

\begin{verbatim}
     ... // várias linhas
     texto1000texto    // linha i
     texto1000texto    // linha i+1
     texto1000texto    // linha i+2
     texto1000texto    // linha i+3
     texto1000texto    // linha i+4
     ... // várias linhas
\end{verbatim}

Gostaria de um comando que mudasse para

\begin{verbatim}
     ... // várias linhas
     texto1001texto    // linha i
     texto1002texto    // linha i+1
     texto1003texto    // linha i+2
     texto1004texto    // linha i+3
     texto1005texto    // linha i+4
     ... // várias linhas
\end{verbatim}

 Ou seja, somasse 1 a cada um dos números entre os textos
 especificando como range as linhas i,i+4

\begin{verbatim}
     :10,20! awk 'BEGIN{i=1}{if (match($0, ``+'')) print ``o''
     (substr($0, RSTART, RLENGTH) + i++) ``o'``}''
\end{verbatim}

 Mas muitos sistemas não tem awk, e logo a melhor solução mesmo é usar o Vim:

\begin{verbatim}
     :let i=1 | 10,20 g/texto\d\+texto/s/\d\+/\=submatch(0)+i/ | let i=i+1
\end{verbatim}

Observação: 10,20 é o intervalo, ou seja, da linha 10 até a linha 20

\begin{verbatim}
     :help /
     :help :s
     :help pattern
\end{verbatim}

\section{Dicas do dicas-l}

fonte: \url{http://www.dicas-l.com.br/dicas-l/20081228.php}

\section{Junção de linhas com Vim}
\label{Junção de linhas com Vim}
Colaboração: Rubens Queiroz de Almeida

Recentemente precisei combinar, em um arquivo, duas linhas
consecutivas. O arquivo original continha linhas como:

\begin{verbatim}
     Matrícula: 123456
     Senha: yatVind7kned
     Matrícula: 123456
     Senha: invanBabnit3
\end{verbatim}

E assim por diante. Eu precisava converter este arquivo para algo como:

\begin{verbatim}
     Matrícula: 123456 - Senha: yatVind7kned
     Matrícula: 123456 - Senha: invanBabnit3
\end{verbatim}

Para isto, basta executar o comando:

\begin{verbatim}
     :g/^Matrícula/s/\n/ - /
\end{verbatim}

Explicando:

\begin{verbatim}
     s/isto/aquilo/g .. substitui isto por aquilo
     g ................ comando global
     /................. inicia padrão de busca
     ^ ................ indica começo de linha
     Matrícula ........ palavra a ser buscada
     s ................ inicia substituição
     /\n/ - / ......... troca quebra de linha (\n), por -
\end{verbatim}

\chapter{Trabalhando com janelas}\label{cha:Trabalhando com janelas}

O Vim trabalha com o conceito de múltiplos ``buffers''. Cada
``buffer'' é um arquivo carregado para edição. Um ``buffer'' pode
estar visível ou não, e é possível dividir a tela em janelas, de forma
a visualizar mais de um ``buffer'' simultaneamente.

\section{Dividindo a janela }
Observação: \verb+Ctrl = ^+

\begin{verbatim}
     Ctrl-w-s   Divide a janela atual em duas (:split)
     Ctrl-w-o   Faz a janela atual ser a única (:only)
\end{verbatim}

Caso tenha duas janelas e use o atalho acima \verb|^wo| lembre-se de salvar
tudo ao fechar, pois apesar de a outra janela estar fechada o arquivo
ainda estará carregado, portanto faça:

\begin{verbatim}
     :wall ... salva todos `write all'
     :qall ... fecha todos `quite all'
\end{verbatim}

\section{Abrindo e fechando janelas }

\begin{verbatim}
     Ctrl-w-n   Abre uma nova janela acima
     Ctrl-w-q   Fecha a janela atual
     Ctrl-w-c   Fecha a janela atual (:close)
\end{verbatim}

\section{Manipulando janelas }

\begin{verbatim}
     Ctrl-w-w ... Alterna entre janelas
     Ctrl-w-j ... desce uma janela `j'
     Ctrl-w-k ... sobe  uma janela `k'
     Ctrl-w-r ... Rotaciona janelas na tela
     Ctrl-w-+ ... Aumenta o espaço da janela atual
     Ctrl-w-- ... Diminui o espaço da janela atual
\end{verbatim}

\section{File Explorer }
\label{File Explorer }
Para abrir o gerenciador de arquivos do Vim use:

\begin{verbatim}
     :Vex ........... abre o file explorer verticalmente
     :e .   ......... abre o file explorer na janela atual
     após abrir chame a ajuda <F1>
\end{verbatim}

Para abrir o arquivo sob o cursor em nova janela coloque a linha abaixo no seu \verb|~/.vimrc|

\begin{verbatim}
     let g:netrw_altv = 1
\end{verbatim}

Caso queira pode mapear um atalho "no caso abaixo F2" para abrir o File Explorer.

\begin{verbatim}
     map <F2> <Esc>:Vex<cr>
\end{verbatim}

Maiores informações:

\begin{verbatim}
     :help buffers
     :help windows
\end{verbatim}

\section{Dicas}
Caso esteja editando um arquivo e nele houver referência a outro arquivo tipo:

\begin{verbatim}
     /etc/hosts
\end{verbatim}

Você pode usar este comando para abrir uma nova janela com o arquivo citado

\begin{verbatim}
     Ctrl-w f
\end{verbatim}

Mas lembre-se que posicionar o cursor sobre o nome do arquivo
Veja também mapeamentos na seção \ref{Mapeamentos}.

\chapter{Repetição de comandos}\label{Repetição de comandos}

Para repetir a última edição saia do modo de Inserção e pressione ponto (.):

\begin{verbatim}
     .
\end{verbatim}

Para inserir um texto que deve ser repetido várias vezes:

\begin{verbatim}
     # Posicione o cursor no local desejado;
     # Digite o número de repetições;
     # Entre em modo de inserção;
     # Digite o texto;
     # Saia do modo de inserção (tecle Esc).
\end{verbatim}

Por exemplo, se você quiser inserir oitenta traços numa linha, em vez
de digitar um por um, você pode digitar o comando:

\begin{verbatim}
     80i-<Esc>
\end{verbatim}

Veja, passo a passo, o que aconteceu:

 Antes de entrar em modo de inserção usamos um quantificador

\begin{verbatim}
     `80'
\end{verbatim}

 depois iniciamos o modo de inserção

\begin{verbatim}
     i
\end{verbatim}

depois digitamos o caractere a ser repetido

\begin{verbatim}
     -
\end{verbatim}

e por fim saímos do modo de inserção

\begin{verbatim}
     <Esc>
\end{verbatim}

Se desejássemos digitar 10 linhas com o texto

\begin{verbatim}
     isto é um teste
\end{verbatim}

deveríamos então fazer assim:
   
\begin{verbatim}
     <Esc> .. para ter certeza que ainda estamos no modo normal
     10 ..... quantificador antes
     i ...... entrar no modo de inserção
     isto é um teste <Enter>
     <Esc> .. voltar ao modo normal
\end{verbatim}

\section{Repetindo a digitação de uma linha }
Para repetir a linha acima (modo de inserção) use

\begin{verbatim}
     Ctrl-y
\end{verbatim}

Para repetir a linha abaixo (modo de inserção)

\begin{verbatim}
     Ctrl-e
\end{verbatim}

Para copiar a linha atual

\begin{verbatim}
     yy
\end{verbatim}

Para colar a linha copiada

\begin{verbatim}
     p
\end{verbatim}

Para repetir uma linha completa

\begin{verbatim}
     Ctrl-x Ctrl-l
\end{verbatim}

O atalho acima só funcionará para uma linha semelhante, experimente
digitar

\begin{verbatim}
     uma linha qualquer com algum conteúdo
     uma linha <Ctrl-x Ctrl-l>
\end{verbatim}

e veja o resultado

\section{Guardando trechos em ``registros''}
\label{sec:Guardando trechos em ``registros''}

Os registradores ``z'' são uma espécie de área de transferência múltipla.

Você deve estar em modo normal e então digitar uma aspa dupla e uma
das 26 letras do alfabeto, em seguida uma ação por exemplo, `y'
(copiar) `d' (apagar). Depois, mova o cursor para a linha
desejada e cole com "rp, onde `r' corresponde ao
registrador para onde o trecho foi copiado.

\begin{verbatim}
     "ayy ... copia a linha atual para o registrador '``a'''
     "ap  ... cola o conteúdo do registrador '``a''' abaixo
     "bdd ... apaga a linha atual para o registrador '``b'''
\end{verbatim}

\section{Macros: gravando comandos}\label{Macros: gravando comandos}

Imagine que você tem o seguinte trecho de código:

\begin{verbatim}
     stdio.h
     fcntl.h
     unistd.h
     stdlib.h
\end{verbatim}

e quer que ele fique assim:

\begin{verbatim}
     #include "stdio.h"
     #include "fcntl.h"
     #include "unistd.h"
     #include "stdlib.h"
\end{verbatim}

Não podemos simplesmente executar repetidas vezes um comando do Vim, pois
precisamos incluir texto tanto no começo quanto no fim da linha?  É necessário
mais de um comando para isso.  É aí que entram as macros. Podemos gravar até 26
macros, já que elas são guardadas nos registros do Vim, que são identificados
pelas letras do alfabeto. Para começar a gravar uma macro no registro '``a''',
digitamos

\begin{verbatim}
     qa
\end{verbatim}

No modo Normal. Tudo o que for digitado a partir daí será gravado no
registro ``a'' até que terminemos com o comando
\verb|<Esc>q| novamente (no modo Normal). Assim,
podemos solucionar nosso problema:

\begin{verbatim}
     <Esc> ....... para garantir que estamos no modo normal
     qa .......... inicia a gravação da macro 'a'
     I ........... entra no modo de inserção no começo da linha
     #include " .. insere #include "
     <Esc> ....... sai do modo de inserção
     A" .......... insere o último caractere
     <Esc> ....... sai do modo de inserção
     j ........... desce uma linha
     <Esc> ....... sai do modo de inserção
     q ........... para a gravação da macro
\end{verbatim}

Agora você só precisa posicionar o cursor na primeira letra de uma linha como esta

\begin{verbatim}
     stdio.h
\end{verbatim}

E executar a macro do registro ``a'' quantas vezes for necessário,
usando o comando \verb|@a|. Para executar quatro vezes, digite:

\begin{verbatim}
     4@a
\end{verbatim}

Este comando executa quatro vezes o conteúdo do registro ``a''.

Caso tenha executado a macro uma vez pode repeti-la com o comando

\begin{verbatim}
     @@
\end{verbatim}

\section{Repetindo substituições }
Se você fizer uma substituição em um intervalo como abaixo

\begin{verbatim}
     :5,32s/isto/aquilo/g
\end{verbatim}

Pode repetir esta substituição em qualquer linha que estiver apenas usando este símbolo

\begin{verbatim}
     &
\end{verbatim}

O Vim substituirá na linha corrente ``isto'' por ``aquilo''. Podemos
repetir a última substituição globalmente assim:
   
\begin{verbatim}
     g&
\end{verbatim}

\section{Repetindo comandos}\label{Repetindo comandos}

\begin{verbatim}
     @:
\end{verbatim}

O atalho acima repete o último comando no próprio modo de comandos

\section{Scripts Vim}\label{Scripts Vim}
Usando um {\em script} para modificar um nome em vários arquivos: 
Crie um arquivo chamado `subst.vim' contendo os comandos de substituição e o
comando de salvamento :wq.

\begin{verbatim}
     %s/bgcolor="e"/bgcolor="e"/g
     wq
\end{verbatim}

Para executar um {\em script}, digite o comando

\begin{verbatim}
     :source nome_do_script.vim
\end{verbatim}

\section{Usando o comando bufdo}\label{Usando o comando bufdo}

Com o comando :bufdo podemos executar um comando em um
conjunto de arquivos de forma rápida. No exemplo a seguir, abriremos
todos os arquivos HTML do diretório atual, efetuarei uma substituição
e em seguida salvo todos.

\begin{verbatim}
     vim *.html
     :bufdo %s/bgcolor="e"/bgcolor="e"/g | :wall
\end{verbatim}

Para fechar todos os arquivos faça:

\begin{verbatim}
     :qall
\end{verbatim}

O comando :wall salva ``write'' todos ``all'' os arquivos
abertos pelo comando vim *.html. Opcionalmente você pode
combinar ``l'' e ``l'' com o comando :wqall, que
salva todos os arquivos abertos e em seguida sai do Vim.

\section{Colocando a última busca em um comando }
Observação: lembre-se \verb|Ctrl = ^|

\begin{verbatim}
     :^r/
\end{verbatim}

\section{Inserindo o nome do arquivo no comando }

\begin{verbatim}
     :^r%
\end{verbatim}

\section{Inserindo o último comando }

\begin{verbatim}
     ^r:
\end{verbatim}

Se preceder com ``:'' você repete o comando, equivale a acessar o histórico de
comandos com as setas

\begin{verbatim}
     :^r:
\end{verbatim}

\section{Para repetir exatamente a última inserção }

\begin{verbatim}
     i<c-a>
\end{verbatim}

\chapter{Comandos externos}
O Vim permite executar comandos externos para processar ou filtrar o
conteúdo de um arquivo. De forma geral, fazemos isso digitando (no
modo normal):

\begin{verbatim}
     :!ls .... visualiza o conteúdo do diretório
\end{verbatim}

Lembrando que anexando um simples ponto, a saída do comando torna-se o 
domcumento que está sendo editado:

\begin{verbatim}
     :.!ls .... imprime na tela o conteúdo do diretório
\end{verbatim}

A seguir, veja alguns exemplos de utilização:

\section{Ordenando}
Podemos usar o comando {\em sort} que ordena o conteúdo de um arquivo dessa forma:

\begin{verbatim}
     :5,15!sort ..... odena da linha 5 até a linha 15
\end{verbatim}

O comando acima ordena da linha 5 até a linha 15.

O comando {\em sort} existe tanto no Windows quanto nos sistemas Unix.
Digitando simplesmente {\em sort}, sem argumentos, o comportamento padrão
é de classificar na ordem alfabética (baseando-se na linha inteira).
Para maiores informações sobre argumentos do comando {\em sort}, digite

\begin{verbatim}
     sort --help ou man sort (no Unix) ou
     sort /? (no Windows).
\end{verbatim}

\section{Removendo linhas duplicadas}

\begin{verbatim}
     :%!uniq
\end{verbatim}

O caractere ``\%'' representa a região equivalente ao arquivo atual inteiro.
A versão do Vim 7 em diante tem um comando {\em sort} que permite remover
linhas duplicadas {\em uniq} e ordenar, sem a necessidade de usar comandos
externos, para mais detalhes:

\begin{verbatim}
     :h sort
\end{verbatim}

\section{Ordenando e removendo linhas duplicadas no Vim 7}

\begin{verbatim}
     :sort u
\end{verbatim}

Quando a ordenação envolver números faça:

\begin{verbatim}
     :sort n
\end{verbatim}

\section{Beautifiers}

A maior parte das linguagens de programação possui ferramentas
externas chamadas {\em beautifiers}, que servem para embelezar o código,
através da indentação e espaçamento. Por exemplo, para embelezar um
arquivo HTML é possível usar a ferramenta ``tildy'', do W3C:

\begin{verbatim}
     :%!tildy
\end{verbatim}

\section{Compilando e verificando erros}
Se o seu projeto já possui um Makefile, então você pode fazer uso do comando
{\tt :make} para poder compilar seus programas no conforto de seu Vim:

\begin{verbatim}
		:make
\end{verbatim}

A vantagem de fazer isso é poder usar outra ferramenta bastante interessante, a janela
de {\em quickfix}:

\begin{verbatim}
		:cwindow
\end{verbatim}

O comando {\tt cwindow} abrirá uma janela em um {\em split} horizontal com a
listagem de erros e {\em warnings}.  Você poderá navegar pela lista usando os
cursores e ir diretamente para o arquivo e linha da ocorrência.

\section{Grep}

Do mesmo jeito que você usa {\em grep} na sua linha de comando você pode usar
o {\em grep} interno do Vim. Exatamente do mesmo jeito:

\begin{verbatim}
		:grep <caminho> <padrão> <opções>
\end{verbatim}

Use a janela de {\em quickfix} aqui também para exibir os resultados do {\em
grep} e poder ir diretamente 
a eles.

\section{Referências}
* \url{http://www.dicas-l.com.br/dicas-l/20070119.php}

\chapter{Como editar preferências no Vim}\label{cha:Como editar preferências no Vim}
O arquivo de preferências do Vim é ``vimrc'', um arquivo oculto que
pode ser criado no {\em profile} do usuário.

\begin{verbatim}
     ~/.vimrc
     /home/seuusuario/.vimrc
\end{verbatim}

Caso use o Windows o arquivo é:

\begin{verbatim}
     ~\_vimrc
     c:\documents and settings\seuusuario\_vimrc
\end{verbatim}

\section{Onde colocar plugins e temas de cor}
\label{Onde colocar plugins e temas de cor}

No Windows procure ou crie uma pasta chamada ``vimfiles'' que fica em

\begin{verbatim}
     c:\documents and settings\seuusuario\
\end{verbatim}

No GNU/Linux procure ou crie uma pasta chamada .vim que deve ficar em

\begin{verbatim}
     /home/user/.vim
\end{verbatim}

Nesta pasta '.vim' ou 'vimfiles' deve haver pastas tipo

\begin{verbatim}
     vimfiles
        |
        +--color
        |
        +--doc
        |
        +--syntax
        |
        +--plugin
\end{verbatim}

Na seção Plugins \ref{Plugins} página \pageref{Plugins}
estão descritos alguns plugins interessantes!

\section{Comentários }
\label{Comentários }

\begin{verbatim}
     " linhas começadas com aspas são comentários
     " e portanto serão ignoradas pelo Vim
\end{verbatim}

Ao fazer modificações comente usando aspas duplas no começo da linha,
os comentários lhe ajudarão mais tarde, pois à medida que o seu vimrc
cresce podem aparecer dúvidas sobre o que determinado trecho faz :)


As alterações no vimrc só será efetivadas na próxima vez que o Vim for aberto
a não ser que você faça um mapeamento para recarregar, mais adiante você verá isto
por enquanto faça

\begin{verbatim}
     :source ~/vimrc ....... se estiver no GNU/Linux
     :source ~/_vimrc ...... caso use o sitema virótico
\end{verbatim}

\section{``Set''}
\label{``Set''}
Os comandos `set' podem ser colocados no \verb|.vimrc|:

\begin{verbatim}
     set nu
\end{verbatim}

ou digitados como comandos:

\begin{verbatim}
     :set nu
\end{verbatim}

\begin{verbatim}
     set nu       "mostra numeração de linhas
     set showmode "mostra o modo em que estamos
     set showcmd  "mostra no status os comandos inseridos
     set ts=4     "tamanho das tabulações
     syntax on    "habilita cores
     set hls      "destaca com cores os termos procurados
     set incsearch "habilita a busca incremental
     set ai       "auto identação
     set aw       "salvamento automático - veja :help aw
     set ignorecase "faz o Vim ignorar maiúsculas e minúsculas nas buscas
     set smartcase  "Se começar uma busca em maiúsculo ele habilita o case
     set ic        "ignora maiúscula e minúsculas em uma busca
     set scs       "ao fazer uma busca com maiúsculos considerar case sensitive
     set backup
     set backupext=.backup
     set backupdir=~/.backup,./
     set cul        "abreviação de cursor line (destaca linha atual)
     set ve=all     "permite mover o cursor para áreas onde não há texto
     set ttyfast    "Envia mais caracteres ao terminal, melhorando o redraw de janelas.
     set columns=88 "Determina a largura da janela.
     set mousemodel=popup "exibe o conteúdo de folders e sugestões spell
\end{verbatim}

O comando \verb|gqap| ajusta o parágrafo atual em modo normal

\begin{verbatim}
     " * coloca 2 espaços após o . quando usando o gq
     "set nojoinspaces
     " ****************************************************************
     " *                                                              *
     " *   geralmente usamos ^I para representar uma tabulação        *
     " *   <Tab>, e $ para indicar o fim de linha. Mas é possível     *
     " *   customizar essas opções. sintaxe:                          *
     " *                                                              *
     " *   set listchars=key:string,key:string                        *
     " *                                                              *
     " *                                                              *
     " * - eol:{char}                                                 *
     " *                                                              *
     " *     Define o caracter a ser posto depois do fim da linha     *
     " *                                                              *
     " * - tab:{char1}{char2}                                         *
     " *                                                              *
     " *     O tab é mostrado pelo primeiro caracter {char1} e        *
     " *     seguido por {char2}                                      *
     " *                                                              *
     " * - trail:{char}                                               *
     " *                                                              *
     " *     Esse caracter representa os espaços em branco.           *
     " *                                                              *
     " * - extends:{char}                                             *
     " *                                                              *
     " *    Esse caracter representa o início do fim da linha         *
     " *    sem quebrá-la                                             *
     " *    Está opção funciona com a opção nowrap habilitada         * 
     " *                                                              *
     " ****************************************************************
     "exemplo 1:
     "set listchars=tab:>-,trail:.,eol:#,extends:@
     
     "exemplo 2:
     "set listchars=tab:>-
     
     "exemplo 3:
     "set listchars=tab:>-
     
     "exemplo 4:
     set nowrap    "Essa opção desabilita a quebra de linha
     "set listchars=extends:+
     
     Caso esteja usando o gvim pode setar um esquema de cores
     set colo desert
\end{verbatim}

\section{Exibindo caracteres invisíveis}
\label{Exibindo caracteres invisíveis}

\begin{verbatim}
     :set list
\end{verbatim}

\section{Definindo macros previamente}
\label{Definindo macros previamente}
Definindo uma macro de nome \verb|s| para ordenar e retirar linhas duplicadas

\begin{verbatim}
     let @s=":sort u"
\end{verbatim}

Para executar a macro \verb|s| definida acima faça:

\begin{verbatim}
     @s
\end{verbatim}

O Vim colocará no comando

\begin{verbatim}
     :sort -u
\end{verbatim}

Bastando pressionar \verb|<Enter>|.
Observação: esta macro prévia pode ficar no vimrc ou ser digitada em comando ``:''


\begin{verbatim}
     :5,20sort u
     "da linha 5 até a linha 20 ordene e retire duplicados
     
     :sort n
     " ordene meu documento considerando números
     " isto é útil pois se a primeira coluna contiver
     " números a ordenação pode ficar errada caso não usemos
     " o parâmetro ``n''
\end{verbatim}

\section{Mapeamentos}\label{Mapeamentos}

Mapeamentos permitem criar atalhos de teclas para quase tudo. Tudo depende de
sua criatividade e do quanto conhece o Vim. 


\subsection{Notas sobre mapeamentos}\label{Notas sobre mapeamentos}
Mapeamentos são um ponto importante do Vim, com eles podemos controlar
ações com quaisquer teclas, mas antes temos que saber que:

Para criar mapeamentos, precisamos conhecer a maneira de representar
as teclas e combinações. Alguns exemplos:

\begin{verbatim}
     tecla       : tecla mapeada
     <c-x>       : Ctrl-x
     <left>      : seta para a esquerda
     <right>     : seta para a direita
     <c-m-a>     : Ctrl-Alt-a
     <cr>        : Enter
     <Esc>       : Escape
     <leader>    : normalmente \
     <bar>       : | pipe
     <cword>     : palavra sob o cursor
     <cfile>     : arquivo sob o cursor
     <cfile><    : arquivo sob o cursor sem extensão
     <sfile>     : conteúdo do arquivo sob o cursor
     <left>      : salta um caractere para esquerda
     <up>        : equivale clicar em 'seta acima'
     <m-f4>      : a tecla alt -> m  mais a tecla f4
     <c-f>       : Ctrl-f
     <bs>        : backspace
     <space>     : espaço
     <tab>       : tab
\end{verbatim}

Para ler mais sobre atalhos de tecla no Vim acesse 

\begin{verbatim}
     :h index
\end{verbatim}

No Vim podemos mapear uma tecla para o modo normal, realizando
determinada operação e a mesma tecla pode desempenhar outra função
qualquer em modo de inserção ou comando, veja:

\begin{verbatim}
     " mostra o nome do arquivo com o caminho
     map <F2> :echo expand("%:p")
     
     " insere um texto qualquer
     imap <F2> Nome de uma pessoa
\end{verbatim}

A única diferença nos mapeamentos acima é que o mapeamento para modo
de inserção começa com ``i'', assim como para o modo ``comando'' ``:'' começa
com ``c'' no caso \verb|cmap|.

\subsection{Recarregando o arquivo de configuração}
\label{sec:Recarregando o arquivo de configuração}

Cada alteração no arquivo de configuração do Vim só terá efeito na próxima vez que você
abrir o Vim a menos que você coloque isto dentro do mesmo

\begin{verbatim}
     " recarregar o vimrc
     " Source the .vimrc or _vimrc file, depending on system
     if &term == "win32" || "pcterm" || has("gui_win32")
        map ,v :e $HOME/_vimrc<CR>
        nmap <F12> :<C-u>source ~/_vimrc <BAR> echo "Vimrc recarregado!"<CR>
     else
        map ,v :e $HOME/.vimrc<CR>
        nmap <F12> :<C-u>source ~/.vimrc <BAR> echo "Vimrc recarregado!"<CR>
     endif
\end{verbatim}

Agora basta pressionar ``\verb|<F12>|'' em modo normal e as alterações passam a valer
instantaneamente, e para chamar o vimrc basta usar.

\begin{verbatim}
     ,v
\end{verbatim}



Os mapeamentos abaixo são úteis
para quem escreve códigos html, permitem inserir caracteres reservados do html
usando uma barra invertida para proteger os mesmos, o Vim substituirá os "barra
alguma coisa" pelo caractere correspondente.

\begin{verbatim}
     inoremap \&amp; &amp;amp;
     inoremap \&lt; &amp;lt;
     inoremap \&gt; &amp;gt;
     inoremap \. &amp;middot;
\end{verbatim}

O termo {\em inoremap} significa: em modo de inserção não remapear, ou seja
ele mapeia o atalho e não permite que o mesmo seja remapeado, e o
mapeamento só funciona em modo de inserção, isso significa que um atalho
pode ser mapeado para diferentes modos de operação. \\


Veja este outro mapeamento:

\begin{verbatim}
     map <F11> <Esc>:set nu!<cr>
\end{verbatim}

Permite habilitar ou desabilitar números de linha do arquivo corrente.
A exclamação ao final torna o comando booleano, ou seja, se a
numeração estiver ativa será desabilitada, caso contrário será
ativada. O ``\verb|<cr>|'' ao final representa um {\tt Enter}.

\subsection{Limpando o ``registro'' de buscas}\label{Limpando o ``registro'' de buscas}

A cada busca, se a opção `hls' estiver habilitada o Vim faz um
destaque colorido, mas se você quiser limpar pode fazer este
mapeamento

\begin{verbatim}
     nno <S-F11> <Esc>:let @/=""<CR>
\end{verbatim}

É um mapeamento para o modo normal que faz com que a combinação de
teclas Shift-F11 limpe o ``registro`' de buscas

\subsection{Destacar palavra sob o cursor }
\label{Destacar palavra sob o cursor }

\begin{verbatim}
     nmap <s-f> :let @/=">"<CR>
\end{verbatim}

O atalho acima `s-f' corresponde a Shift-f

\subsection{Remover linhas em branco duplicadas }
\label{Remover linhas em branco duplicadas }

\begin{verbatim}
     map ,d <Esc>:%s/\(^\n\{2,}\)/\r/g<cr>
\end{verbatim}

No mapeamento acima estamos associando o atalho:

\begin{verbatim}
     ,d
\end{verbatim}

\dots~à ação desejada, fazer com que linhas em branco sucessivas sejam
substituídas por uma só linha em branco, vejamos como funciona:

\begin{verbatim}
     map ......... mapear
     ,d .......... atalho que quermos
     <Esc> ....... se estive em modo de inserção sai
     : ........... em modo de comando
     % ........... em todo o arquivo
     s ........... substitua
     \n .......... quebra de linha
     {2,} ........ duas ou mais vezes
     \r .......... trocado por \r Enter
     g ........... globalmente
     <cr> ........ confirmação do comando
\end{verbatim}

As barras invertidas podem não ser usadas se o seu Vim estiver com a opção
{\em magic} habilitada

\begin{verbatim}
     :set magic
\end{verbatim}

Por acaso este é um padrão portanto tente usar assim pra ver se funciona

\begin{verbatim}
     map ,d :%s/\n{2,}/\r/g<cr>
\end{verbatim}

\subsection{Mapeamentos globais}


Podemos fazer mapeamentos globais ou que funcionam em apenas um modo:

\begin{verbatim}
     map  - funciona em qualquer modo
     nmap - apenas no modo Normal
     imap - apenas no modo de Inserção
\end{verbatim}

Mover linhas com {\tt Ctrl-(seta abaixo)} ou {\tt Ctrl-(seta acima)}:

\begin{verbatim}
     " tem que estar em modo normal!
     nmap <C-Down> ddp
     nmap <C-Up> ddkP
\end{verbatim}

Salvando com uma tecla de função:

\begin{verbatim}
     " salva com F9
     nmap <F9> :w<cr>
     " F10 - sai do Vim
     nmap <F10> <Esc>:q<cr>
\end{verbatim}

\subsection{Convertendo as iniciais de um documento para maiúsculas}
\label{Convertendo as iniciais de um documento para maiúsculas}

\begin{verbatim}
     " MinusculasMaiusculas: converte a primeira letra de cada
     " frase para MAIÚSCULAS
     nmap ,mm :%s/\C\([.!?][])"']*\($\|[ ]\)\_s*\)\(\l\)/\1\U\3/g<CR>
     " caso queira confirmação coloque uma letra ``c'' no final da linha acima:
     " (...) \3/gc<CR>
\end{verbatim}

\section{Autocomandos }\label{Autocomandos }

Autocomandos habilitam comandos automáticos para situações
específicas. Se você deseja que seja executada uma determinada ação ao
iniciar um novo arquivo o seu autocomando deverá ser mais ou menos
assim:

\begin{verbatim}
     au BufNewFile tipo ação
\end{verbatim}

Veja um exemplo:

\begin{verbatim}
     au BufNewFile,BufRead *.txt source ~/.vim/syntax/txt.vim
\end{verbatim}

No exemplo acima o Vim aplica autocomandos para arquivos novos
``BufNewfile'' ou existentes ``BufReadd'' do tipo `txt' e para estes tipos
carrega um arquivo de syntax, ou seja, um esquema de cores específico.

\begin{verbatim}
     " http://aurelio.net/doc/vim/txt.vim    coloque em ~/.vim/syntax
     au BufNewFile,BufRead *.txt source ~/.vim/syntax/txt.vim
\end{verbatim}

Para arquivos do tipo txt `*.txt' use um arquivo de syntax em particular

O autocomando abaixo coloca um cabeçalho para scripts `bash' caso a
linha 1 esteja vazia, observe que os arquivos em questão tem que ter a
extensão .sh

\begin{verbatim}
     au BufNewFile,BufRead *.sh if getline(1) == "" | normal ,sh
\end{verbatim}


\section{Funções}
\label{sec:Funções}

\subsection{Fechamento automático de parênteses}
\label{sec:Fechamento automático de parênteses}

\begin{verbatim}
     " --------------------------------------
     " Ativa fechamento automático para parêntese
     " Set automatic expansion of parenthesis/brackets
     inoremap ( ()<Esc>:call BC_AddChar(``)'')<cr>i
     inoremap { {}<Esc>:call BC_AddChar(``}'')<cr>i
     inoremap [ []<Esc>:call BC_AddChar(``]'')<cr>i
     `` '' ``''<Esc>:call BC_AddChar(``''")<cr>i
     "
     " mapeia Ctrl-j para pular fora de parênteses colchetes etc...
     inoremap <C-j> <Esc>:call search(BC_GetChar(), ``W'')<cr>a
     " Function for the above
     function! BC_AddChar(schar)
        if exists(``k'')
            let b:robstack = b:robstack . a:schar
        else
            let b:robstack = a:schar
        endif
     endfunction
     function! BC_GetChar()
        let l:char = b:robstack[strlen(b:robstack)-1]
        let b:robstack = strpart(b:robstack, 0, strlen(b:robstack)-1)
        return l:char
     endfunction
    
    '''Outra opção para fechamento de parênteses'''
    
     " Fechamento automático de parênteses
     imap { {}<left>
     imap ( ()<left>
     imap [ []<left>
    
     " pular fora dos parênteses, colchetes e chaves, mover o cursor
     " no modo de inserção
     imap <c-l> <Esc><right>a
     imap <c-h> <Esc><left>a
\end{verbatim}

\subsection{Função para barra de status}\label{Função para barra de status}

\begin{verbatim}
     set statusline=%F%m%r%h%w\
        [FORMAT=%{&ff}]\
        [TYPE=%Y]\
        [ASCII=\%03.3b]\
        [HEX=\%02.2B]\
        [POS=%04l,%04v][%p%%]\ [LEN=%L]
\end{verbatim}
Caso este código não funcione acesse este link: \url{http://www.linux.com/feature/120126}.


\subsection{Rolar outra janela}\label{Rolar outra janela}

Se você dividir janelas tipo

\begin{verbatim}
     Ctrl-w n
\end{verbatim}

pode colocar esta função no seu \verb|.vimrc|

\begin{verbatim}
     " rola janela alternativa
     fun! ScrollOtherWindow(dir)
     if a:dir == ``n''
        let move = ``>''
     elseif a:dir == ``p''
        let move = ``>''
     endif
     exec ``p'' . move . ``p''
     endfun
     nmap <silent> <M-Down> :call ScrollOtherWindow(``n'')<CR>
     nmap <silent> <M-Up> :call ScrollOtherWindow(``p'')<CR>
\end{verbatim}

Esta função é acionada com o atalho `alt' + setas acima e abaixo

\subsection{Função para numerar linhas}\label{Função para numerar linhas}
link para a dica: \url{http://vim.wikia.com/wiki/Number_a_group_of_lines}

\subsection{Função para trocar o esquema de cores}

\begin{verbatim}
     function! <SID>SwitchColorSchemes()
       if exists(``e'')
        if g:colors_name == 'native'
          colorscheme billw
        elseif g:colors_name == 'billw'
          colorscheme desert
        elseif g:colors_name == 'desert'
          colorscheme navajo-night
        elseif g:colors_name == 'navajo-night'
          colorscheme  zenburn
        elseif g:colors_name == 'zenburn'
          colorscheme bmichaelsen
        elseif g:colors_name == 'bmichaelsen'
          colorscheme wintersday
        elseif g:colors_name == 'wintersday'
          colorscheme summerfruit
        elseif g:colors_name == 'summerfruit'
          colorscheme native
        endif
       endif
     endfunction
     map <silent> <F6> :call <SID>SwitchColorSchemes()<CR>
\end{verbatim}

baixe os esquemas aqui:
\url{http://nanasi.jp/old/colorscheme_0.html}

\subsection{Uma função para inserir cabeçalho de script}
\label{Uma função para inserir cabeçalho de script bash}
para chamar a função
basta pressionar, sh em modo normal

\begin{verbatim}
     " Cria um cabeçalho para scripts bash
     fun! InsertHeadBash()
        normal(1G)
        :set ft=bash
        :set ts=4
        call append(0, ``h'')
        call append(1, ``:'' . strftime("%a %d/%b/%Y hs %H:%M"))
        call append(2, "# ultima modificação:``(''%a %d/%b/%Y hs %H:%M"))
        call append(3, "# NOME DA SUA EMPRESA")
        call append(3, "# Propósito do script")
        normal($)
     endfun
     map ,sh :call InsertHeadBash()<cr>
\end{verbatim}

\subsection{Função para inserir cabeçalhos Python}
\label{Função para inserir cabeçalhos python}

\begin{verbatim}
     " função para inserir cabeçalhos python
     fun! BufNewFile_PY()
      normal(1G)
      :set ft=python
      :set ts=2
      call append(0, "#!/usr/bin/env python")
      call append(1, "# # -*- coding: ISO-8859-1 -*-")
      call append(2, ``:'' . strftime("%a %d/%b/%Y hs %H:%M"))
      call append(3, `` '' . strftime("%a %d/%b/%Y hs %H:%M"))
      call append(4, "# Instituicao: <+nome+>")
      call append(5, "# Proposito do script: <+descreva+>")
      call append(6, "# Autor: <+seuNome+>")
      call append(7, "# site: <+seuSite+>")
      normal gg
     endfun
     autocmd BufNewFile *.py call BufNewFile_PY()
     map ,py :call BufNewFile_PY()<cr>A
   
     " Ao editar um arquivo será aberto no último ponto em
     " que foi editado
   
     autocmd BufReadPost *
       \ if line('``\''``('''\``'') <= line(``$'') |
       \   exe ''normal g`\``" |
       \ endif
\end{verbatim}

\begin{verbatim}
     " Permite recarregar o Vim para que modificações no
     " Próprio vimrc seja ativadas com o mesmo sendo editado
     nmap <F12> :<C-u>source $HOME/.vimrc <BAR> echo "Vimrc recarregado!"<CR>
\end{verbatim}

Redimensionar janelas

\begin{verbatim}
     " Redimensionar a janela com
     " Alt-seta à direita e esquerda
     map <M-right> <Esc>:resize +2 <CR>
     map <M-left> <Esc>:resize -2 <CR>
\end{verbatim}

\subsection{Função para pular para uma linha}
\label{Função para pular para uma linha}

\begin{verbatim}
     "ir para linha
     " ir para uma linha específica
     function! GoToLine()
     let ln = inputdialog("ir para a linha...")
     exe ``:'' . ln
     endfunction
     "no meu caso o mapeamento é com Ctrl-l
     "use o que melhor lhe convier
     imap <S-l> <C-o>:call GoToLine()<CR>
     nmap <S-l> :call GoToLine()<CR>
\end{verbatim}

\subsection{Função para gerar backup}
\label{Função para gerar backup}

A função abaixo é útil para ser usada quando você vai editar um arquivo
gerando modificações significativas, assim você poderá restaurar o backup se necessário

\begin{verbatim}
     " A mapping to make a backup of the current file.
     fun! WriteBackup()
        let fname = expand("%:p") . "__" . strftime("%d-%m-%Y--%H.%M.%S")
        silent exe ":w " . fname
        echo "Wrote " . fname
     endfun
     nnoremap <Leader>ba :call WriteBackup()<CR>
\end{verbatim}

O atalho

\begin{verbatim}
     <leader>
\end{verbatim}

em geral é a barra invertida ``$\backslash$'', na dúvida

\begin{verbatim}
     :help <leader>
\end{verbatim}

\section{Como adicionar o Python ao {\em path} do Vim?}
\label{Como adicionar o python ao path do Vim?}

fonte:
\url{http://vim.wikia.com/wiki/Automatically_add_Python_paths_to_Vim_path}
Coloque o seguinte script em:

\begin{verbatim}
     * ~/.vim/after/ftplugin/python.vim    (on Unix systems)
     %* $HOME/vimfiles/after/ftplugin/python.vim    (on Windows systems)
\end{verbatim}

\begin{verbatim}
     python << EOF
     import os
     import sys
     import vim
     for p in sys.path:
         # Add each directory in sys.path, if it exists.
         if os.path.isdir(p):
             # Command 'set' needs backslash before each space.
             vim.command(r``s'' % (p.replace(`` '', r`` '')))
     EOF
\end{verbatim}

Isto lhe permite usar 'gf' ou {\tt Ctrl-w Ctrl-F} para abrir um arquivo sob o cursor

\section{Criando um menu}
\label{Criando um menu}

Como no Vim podemos ter infinitos comandos fica complicado memorizar tudo
é aí que entram os menus, podemos colocar nossos plugins e atalhos favoritos
em um menu veja este exemplo

\begin{verbatim}
     amenu Ferramentas.ExibirNomeDoTema :echo g:colors_name<cr>
\end{verbatim}

O comando acima diz:

\begin{verbatim}
     amenu ........................ cria um menu
     Ferramentas.ExibirNomeDoTema . Menu plugin submenu ExibirNomeDoTema
     :echo g:colors_name<cr> ...... comando para exibir o nome do tema de cores atual
\end{verbatim}

Caso haja espaços no nome a definir você pode fazer assim

\begin{verbatim}
     amenu Ferramentas.Exibir\ nome\ do\ tema :echo g:colors_name<cr>
\end{verbatim}

\section{Criando menus para um modo específico}
\label{Criando menus para um modo específico}

\begin{verbatim}
     :menu .... Normal, Visual e Operator-pending
     :nmenu ... Modo Normal
     :vmenu ... Modo Visual
     :omenu ... Operator-pending modo
     :menu! ... Insert e Comando
     :imenu ... Modo de inserção
     :cmenu ... Modo de comando
     :amenu ... Todos os modos
\end{verbatim}

\section{Exemplo de menu}
\label{Exemplo de menu}

\begin{verbatim}
     " cores
     menu T&emas.cores.quagmire :colo quagmire<CR>
     menu T&emas.cores.inkpot :colo inkpot<CR>
     menu T&emas.cores.google :colo google<CR>
     menu T&emas.cores.ir_black :colo ir_black<CR>
     menu T&emas.cores.molokai :colo molokai<CR>
     " Fontes
     menu T&emas.fonte.Inconsolata :set gfn=Inconsolata:h10<CR>
     menu T&emas.fonte.Anonymous :set anti gfn=Anonymous:h8<CR>
     menu T&emas.fonte.Envy\ Code :set anti gfn=Envy_Code_R:h10<CR>
     menu T&emas.fonte.Monaco :set gfn=monaco:h9<CR>
     menu T&emas.fonte.Crisp :set anti gfn=Crisp:h12<CR>
     menu T&emas.fonte.Liberation\ Mono :set gfn=Liberation\ Mono:h10<CR>
\end{verbatim}

O comando

\begin{verbatim}
     :update
\end{verbatim}

Atualiza o menu recém modificado.

Quando o comando

\begin{verbatim}
     :amenu
\end{verbatim}

É usado sem nenhum argumento o Vim mostra os menus definidos atualmente

Para listar todas as opções de menu para 'Plugin' por exemplo faça:

\begin{verbatim}
     :amenu Plugin
\end{verbatim}

\section{Outros mapeamentos}
\label{Outros mapeamentos}

Destaca espaços e tabulações redundantes {\tt <br>}
%Highlight redundant whitespace and tabs.

\begin{verbatim}
     highlight RedundantWhitespace ctermbg=red guibg=red
     match RedundantWhitespace /\s\+$\| \+\ze\t/
\end{verbatim}

Explicando com detalhes

\begin{verbatim}
     \s ..... espaço
     \+ ..... uma ou mais vezes
     $ ...... no final da linha
     \| ..... ou
     `` '' .. espaço (veja imagem acima)
     \+ ..... uma ou mais vezes
     \ze .... até o fim
     \t ..... tabulação
\end{verbatim}

Portanto a expressão regular acima localizará espaços ou tabulações no final de linha
e destacará em vermelho.

"Remove espaços redundantes no fim das linhas

\begin{verbatim}
     map <F7> <Esc>mz:%s/\s\+$//g<cr>`z
\end{verbatim}

Um detalhe importante

\begin{verbatim}
     mz ... marca a posição atual do cursor para retornar no final do comando
     `z ... retorna à marca criada
\end{verbatim}

Se não fosse feito isto o cursor iria ficar na linha da última substituição!

"Abre o vim-vim explorer

\begin{verbatim}
     map <F6> <Esc>:vne .<cr><bar>:vertical resize -30<cr><bar>:set nonu<cr>
\end{verbatim}

Podemos usar ``Expressões Regulares\footnote{\url{http://guia-er.sourceforge.net}}'' em
buscas do Vim veja um exemplo para retirar todas as tags html

\begin{verbatim}
     "mapeamento para retirar tags html com Ctrl-Alt-t
     nmap <C-M-t> :%s/<[^>]*>//g <cr>
     " Quebra a linha atual no local do cursor com F2
     nmap <F2> a<CR><Esc>
     " join lines  -- Junta as linhas com Shift-F2
     nmap <S-F2> A<Del><Space>
\end{verbatim}

Para mais detalhes sobre buscas acesse ``\ref{Buscas e substituições}
na página \pageref{Buscas e substituições}''

\section{Complementação com ``tab''}\label{Complementação com ``tab''}

\begin{verbatim}
     "Word completion
     "Complementação de palavras
     
     set dictionary+=/usr/dict/words
     set complete=.,w,k
     
     "------ complementação de palavras ----
     "usa o tab em modo de inserção para completar palavras
     
     function! InsertTabWrapper(direction)
        let col = col(``.'') - 1
        if !col || getline(``.'')[col - 1] !~ '\k'
           return ``>''
        elseif ``d'' == a:direction
           return ``>''
        else
           return ``>''
        endif
     endfunction
     
     inoremap <tab> <c-r>=InsertTabWrapper (``d'')<cr>
     inoremap <s-tab> <c-r>=InsertTabWrapper (``d'')<cr>
\end{verbatim}

\section{Abreviações}\label{Abreviações}
Também no \verb|.vimrc| você pode colocar abreviações, que são uma espécie de
auto-texto para o Vim.

\begin{verbatim}
     iab slas Sérgio Luiz Araújo Silva
     iab Linux GNU/Linux
     iab linux GNU/Linux
     
     " Esta abreviação é legal para usar com o python
     imap :<CR> :<CR><TAB>
\end{verbatim}

\section{Referências}
\label{Referências}
* \url{http://www.dicas-l.com.br/dicas-l/20050118.php}

\chapter{Um wiki para o Vim}\label{cha:Um wiki para o Vim}

É inegável a facilidade que um wiki nos traz, os documentos são
indexados e linkados de forma simples. Já pesquisei uma porção de
wikis e, para uso pessoal recomendo o potwiki.  O ``linkk'' do potwiki é:
\url{http://www.vim.org/scripts/script.php?script_id=1018}.
O potwiki é um wiki completo para o Vim, funciona localmente embora
possa ser aberto remotamente via ssh\footnote{Sistema de acesso remoto}.
Para criar um ``link'' no potwiki basta usar WikiNames, são nomes
iniciados com letra maiúscula e que contenham outra letra em maiúsculo
no meio. \\


Ao baixar o arquivo salve em \verb|~/.vim/plugin|. \\



Mais ou menos na linha 53 do potwiki \verb|~/.vim/plugin/potwiki.vim| você
define onde ele guardará os arquivos, no meu caso
\verb|/home/docs/textos/wiki|. a linha ficou assim:

\begin{verbatim}
     call s:default('home',"~/.wiki/HomePage")
\end{verbatim}

Outra forma de indicar a página inicial seria colocar no seu .virmc

\begin{verbatim}
     let potwiki_home = "$HOME/.wiki/HomePage"
\end{verbatim}

\section{Como usar}
\label{Como usar}

O potwiki trabalha com WikiWords, ou seja, palavras iniciadas com
letras em maiúsculo e que tenham outra letra em maiúsculo no meio (sem
espaços) para iniciá-lo abra o Vim e pressione \verb|\ww|.

\begin{verbatim}
     <Leader> é igual a \   - veja :help leader
     \ww  .... abra a sua HomePage
     \wi  .... abre o Wiki index
     \wf  .... segue uma WikiWords (can be used in any buffer!)
     \we  .... edite um arquivo Wiki
     \\   .... Fecha o arquivo
     <CR> .... segue WikiWords embaixo do cursor <CR> é igual a Enter
     <Tab>.... move para a próxima WikiWords
     <BS> .... move para os WikiWords anteriores (mesma página)
     \wr  .... recarrega WikiWords
\end{verbatim}

\section{Salvamento automático para o Wiki }
\label{Salvamento automático para o wiki }
Procure por uma seção {\em autowrite} no manual do potwiki

\begin{verbatim}
     :help potwiki
\end{verbatim}

O valor que está em zero deverá ficar em 1

\begin{verbatim}
     call s:default(`autowrite',0)
\end{verbatim}

\section{Dicas}
\label{Dicas}
Como eu mantenho o meu wiki oculto ``.wiki'' criei um ``link'' para a pasta de textos

\begin{verbatim}
     ln -s ~/.wiki /home/sergio/docs/textos/wiki
\end{verbatim}

Vez por outra entro na pasta \verb|~/docs/textos/wiki| e crio um
pacote tar.gz e mando para ``web'' como forma de manter um ``backup''.

\section{Problemas com codificação de caracteres}
\label{Problemas com codificação de caracteres}

Atualmente uso o Ubuntu em casa e ele já usa utf-8. Ao restaurar meu
``backup'' do wiki no Kurumin os caracteres ficaram meio estranhos,
daí fiz:

\begin{verbatim}
     baixei o pacote [recode]
     # apt-get install recode
     
     para recodificar caracteres de utf-8 para isso faça:
     recode -d u8..l1 arquivo
\end{verbatim}

\chapter{Hábitos para edição efetiva}
\label{cha:Hábitos para edição efetiva}


Um dos grandes problemas relacionados com os softwares é sua subutilização. Por
inércia o usuário tende a aprender o mínimo para a utilização de um programa e
deixa de lado recursos que poderiam lhe ser de grande valia. O mantenedor do
Vim  Bram Moolenaar\footnote{http://www.moolenaar.net} recentemente publicou vídeos
e manuais sobre os ``7 hábitos para edição efetiva de
textos\footnote{http://br-linux.org/linux/7-habitos-da-edicao-de-texto-efetiva}'',
este capítulo pretende resumir alguns conceitos mostrados por Bram Moolenaar em
seu artigo.

\section{Mova-se rapidamente no texto}
\label{sec:Mova-se rapidamente no texto}

Leia sobre como mover-se no documento no capítulo \ref{cha:Movendo-se no documento}

\section{Use marcas}
veja a seção \ref{sec:Usando marcas} na página \pageref{sec:Usando marcas}.

\begin{verbatim}
     ma ..... em modo normal cria uma marca `a'
     'a ..... move o cursor até a marca `a'
     d'a .... deleta até a marca `a'
     y'a .... copia até a marca `a'
\end{verbatim}



\begin{verbatim}
     gg ... vai para a linha 1 do arquivo
     G .... vai para a última linha do arquivo
     0 .... vai para o início da linha
     $ .... vai para o fim da linha
     fx ... pula até a próxima ocorrência de ``x''
     dfx .. deleta até a próxima ocorrência de ``x''
     g, ... avança na lista de alterações
     g; ... retrocede na lista de alterações
     p .... cola o que foi deletado/copiado abaixo
     P .... cola o que foi deletado/copiado acima
     H .... posiciona o cursor no primeiro caractere da tela
     M .... posiciona o cursor no meio da tela
     L .... posiciona o cursor na última linha da tela
\end{verbatim}

\begin{verbatim}
     * Use asterisco  para localizar a palavra sob o cursor
     * Use o percent % serve para localizar fechamento de parêntese chaves etc
\end{verbatim}

\begin{verbatim}
     '.  apostrofo + ponto retorna ao último local editado
     '' retorna ao local do ultimo salto
\end{verbatim}

Suponha que você está procurando a palavra `argc':

\begin{verbatim}
     /argc
\end{verbatim}

Digita `n' para buscar a próxima ocorrência

\begin{verbatim}
     n
\end{verbatim}

Um jeito mais fácil seria:

\begin{verbatim}
     "coloque a linha abaixo no seu vimrc
     :set hlsearch
\end{verbatim}

Agora use asterisco para destacar todas as ocorrências do padrão desejado
e use a letra `n' para pular entre ocorrências, caso deseje seguir o caminho
inverso use `N'.

\section{Use quantificadores}
\label{Use quantificadores}
Em modo normal você pode fazer

\begin{verbatim}
     10j ..... desce 10 linhas
     5dd ..... apaga as próximas 5 linhas
     :50 ..... vai para a linha 50
     50gg .... vai para a linha 50
\end{verbatim}


\section{Edite vários arquivos de uma só vez }
\label{Edite vários arquivos de uma só vez }

O Vim pode abrir vários arquivos que contenham um determinado padrão.
Um exemplo seria abrir dezenas de arquivos html e trocar a ocorrência
bgcolor=``f'' Para bgcolor=``e'' Usaríamos o seguinte comando

\begin{verbatim}
     vim *.html :bufdo :%s/bgcolor=``f''/bgcolor=``e''/g :wall :qall
\end{verbatim}

Ainda com relação à edição de vários arquivos poderia-mos abrir alguns
arquivos txt e mudar de um para o outro assim:

\begin{verbatim}
     :wn
\end{verbatim}

O ``w'' significa gravar e o ``n'' significa {\em next}, ou seja, gravaria-mos
o que foi modificado no arquivo atual e mudaríamos para o próximo.

Veja também: \ref{cha:Movendo-se no documento}

\section{Não digite duas vezes}
\label{Não digite duas vezes}

\begin{itemize}
\item O Vim complementa com ``tab''. Veja mais na seção \ref{Complementação com ``tab''} na página \pageref{Complementação com ``tab''}.
\item Use macros. Detalhes na seção \ref{Macros: gravando comandos}
página \pageref{Macros: gravando comandos}.
\item Use abreviações coloque abreviações como abaixo em seu \verb|~/.vimrc|. Veja mais na seção \ref{Abreviações}.
\item as abreviações fazem o mesmo que auto-correção e auto-texto em outros editores
\end{itemize}

\begin{verbatim}
     iab tambem também
     iab linux GNU/Linux
\end{verbatim}



* No modo de inserção você pode usar

\begin{verbatim}
     Ctrl-y  ....... copia caractere a caractere a linha acima
     Ctrl-e  ....... copia caractere a caractere a linha abaixo
     Ctrl-x Ctrl-l .. completa linhas inteiras
\end{verbatim}

* Para um trecho muito copiado coloque o seu conteúdo em um registrador

\begin{verbatim}
     "ayy ... copia a linha atual para o registrador ``a''
     "ap  ... cola o registrador ``a''
\end{verbatim}

Crie abreviações para erros comuns no seu arquivo de configuração (~/.vimrc)

\begin{verbatim}
     iabbrev teh the
     syntax keyword WordError teh
\end{verbatim}

As linhas acima criam uma abreviação para erro de digitação da palavra 'the'
e destaca textos que você abrir que contenham este erro.

\section{Use dobras}\label{sec:Use folders}

O Vim pode ocultar partes do texto que não estão sendo utilizadas permitindo
uma melhor visualização do conteúdo. Mais detalhes no capítulo
\ref{cha:Folders} página \pageref{cha:Folders}.

\section{Use autocomandos}
\label{Use autocomandos}

No arquivo de configuração do Vim \verb|~/.vimrc| você pode criar comandos
automáticos que serão executados diante de uma determinada
circunstância

O comando abaixo será executado em qualquer arquivo existente, posicionando o cursor no último local editado

\begin{verbatim}
     "autocmd BufEnter * lcd %:p:h
     autocmd BufReadPost *
       \ if line("'\"") > 0 && line("'\"") <= line("$") |
       \   exe "normal g`\"" |
       \ endif
\end{verbatim}


Grupo de comandos para arquivos do tipo ``html''. Observe que o
autocomando carrega um arquivo de configuração do Vim exclusivo para o
tipo html/htm e no caso de arquivos novos ``BufNewFile'' ele já cria um
esqueleto puxando do endereço indicado.

\begin{verbatim}
     augroup html
      au! <--> Remove all html autocommands
       au!
       au BufNewFile,BufRead *.html,*.shtml,*.htm set ft=html
       au BufNewFile,BufRead,BufEnter  *.html,*.shtml,*.htm so ~/docs/vim/.vimrc-html
       au BufNewFile *.html 0r ~/docs/vim/skel.html
       au BufNewFile *.html*.shtml,*.htm /body/+  " coloca o cursor após o corpo <body>
       au BufNewFile,BufRead *.html,*.shtml,*.htm set noautoindent
     augroup end
\end{verbatim}

Documentação sobre autocomandos do Vim \url{http://www.vim.org/htmldoc/autocmd.html}.

\section{Use o file explorer}\label{Use o file explorer}

O Vim pode navegar em pastas assim:

\begin{verbatim}
     vim .
\end{verbatim}

Você pode usar ``j'' e ``k'' para navegar e {\tt Enter} para editar o arquivo
selecionado

\section{Torne as boas práticas um hábito }\label{Torne as boas práticas um hábito }

Para cada prática produtiva procure adquirir um hábito e mantenha-se
atento ao que pode ser melhorado. Imagine tarefas complexas, procure
um meio melhor de fazer e torne um hábito.

\section{Referências}
\label{Referências}
\begin{itemize}
   \item \url{http://www.moolenaar.net/habits\_2007.pdf} por Bram Moolenaar
   \item \url{http://vim.wikia.com/wiki/Did\_you\_know}
\end{itemize}


\chapter{Plugins}\label{Plugins}

``Plugins'' são um meio de estender as funcionalidades do Vim, há
``plugins'' para diversas tarefas, desde wikis para o Vim até
ferramentas de auxílio a navegação em arquivos com é o caso do
``plugin'' \url{http://www.vim.org/scripts/script.php?script\_id=1658}
NerdTree, que divide uma janela que permite navegar pelos diretórios
do sistema a fim de abrir arquivos a serem editados.

\section{Como testar um plugin sem instalá-lo?}
\label{Como testar um plugin sem instala-lo?}

\begin{verbatim}
     :source <path>/<plugin>
\end{verbatim}

Caso o plugin atenda suas necessidades você pode instala-lo. Este
procedimento também funciona para temas de cor!



No GNU/Linux
\begin{verbatim}
     ~/.vim/plugin/
\end{verbatim}

No Windows

\begin{verbatim}
     ~/vimfiles/plugin/
\end{verbatim}

Obs: Caso não exista a pasta você pode criá-la!

Exemplo no GNU/Linux

\begin{verbatim}
     + /HOME/USER
           |
           |
            + .VIM
                |
                |
                + PLUGIN
\end{verbatim}

Obs: Alguns plugins dependem da versão do Vim, para saber qual
a que está atualmente instalada:

\begin{verbatim}
     :ver
\end{verbatim}

\section{Plugin para \LaTeX}
\label{Plugin para LaTeX}
Um plugin completo para \LaTeX está acessível aqui: \url{http://vim-latex.sourceforge.net/}
Uma vez adicionado o plugin você pode inserir seus {\em templates}
em:

\begin{verbatim}
     ~/.vim/ftplugin/latex-suite/templates
\end{verbatim}


\section{Criando folders para arquivos \LaTeX}
\label{Criando folders para arquivos LaTeX}

\begin{verbatim}
     set foldmarker=\\begin,\\end
     set foldmethod=marker
\end{verbatim}

Adicionar marcadores ({\em labels}) às seções de um documento \LaTeX
\begin{verbatim}
     .s/^\(\\section\)\({.*}\)/\1\2\r\\label\2
\end{verbatim}

\section{Criando seções \LaTeX}\label{Criando seções latex}
o comando abaixo substitui

\begin{verbatim}
     ==seção==
\end{verbatim}

   por

\begin{verbatim}
     \section{seção}
\end{verbatim}

\begin{verbatim}
     :.s/^==\s\?\([^=]*\)\s\?==/\\section{\1}/g
     
     : ......... comando
     . ......... linha atual
     s ......... substitua
     ^ ......... começo de linha
     == ........ dois sinais de igual
     \s\? ...... seguido ou não de espaço
     [^=] ...... não pode haver = (^ dentro de [] é negação)
     * ......... diz que o que vem antes pode vir zero ou mais vezes
     \s\? ...... seguido ou não de espaço
     \\ ........ insere uma barra invertida
     \1 ........ repete o primeiro trecho entre ()
\end{verbatim}

\section{Plugin para manipular arquivos}
\url{http://www.vim.org/scripts/script.php?script_id=2337#0.1.9}
Para entender este plugin acesse este vídeo:
 \url{http://www.screencast.com/t/P6nJkJ0DE}


\section{Complementação de códigos}
\label{Complementação de códigos}

O ``plugin'' snippetsEmu é um misto entre complementação de códigos e
os chamados modelos ou {\em templates}. Insere um trecho de código pronto,
mas vai além disso, permitindo saltar para trechos do modelo inserido
através de um atalho configurável de modo a agilizar o trabalho do
programador. \url{http://www.vim.org/scripts/script.php?script\_id=1318}

\section{Instalação}
\label{Instalação}

Um artigo ensinando como instalar o ``plugin'' snippetsEmu pode ser lido aqui:
 \url{http://vivaotux.blogspot.com/2008/03/instalando-o-plugin-snippetsemu-no-vim.html}

\section{Um wiki para o Vim}
\label{Um wiki para o Vim}

O ``plugin'' wikipot implementa um wiki para o Vim no qual você define
um ``link'' com a notação WikiWord, onde um ``link'' é uma palavra que
começa com uma letra maiúscula e tem outra letra maiúscula no meio
Obtendo o plugin neste link: \url{http://www.vim.org/scripts/script.php?script\_id=1018}.

\section{Acessando documentação do python no Vim}\label{Acessando documentação do python no Vim}

 \url{http://www.vim.org/scripts/script.php?script\_id=910}

\section{Formatando textos planos com syntax}
\label{Formatando textos planos com syntax}
\url{http://www.vim.org/scripts/script.php?script\_id=2208&rating=helpful#1.3}

Veja como instalar o este plugin no capítulo \ref{cha:Um wiki para o Vim}.

\chapter{Referências}
\begin{itemize}
\item \url{http://www.vivaolinux.com.br/artigos/impressora.php?codigo=2914} VIM avançado (parte 1)]
\item \url{http://www.rayninfo.co.uk/vimtips.html}
\item \url{http://www.geocities.com/yegappan/vim\_faq.txt}
\item \url{http://br.geocities.com/cesarakg/vim-cook-ptBR.html}
\item \url{http://larc.ee.nthu.edu.tw/~cthuang/vim/files/vim-regex/vim-regex.htm}
\item \url{http://aurelio.net/vim/vimrc-ivan.txt}
\item \url{http://vivaotux.blogspot.com/search/label/vim}
\item \url{http://www.tug.dk/FontCatalogue/seriffonts.html}
\end{itemize}

\end{document}
