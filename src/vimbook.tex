%%%%%%%%%%%%%%%%%%%%%%%%%%%%%%%%%%%%%%%%%%%%%%%%%%%%%%%%%%%%%%%%%%%%%%%%%%%%%%%%
% vim:enc=utf-8:ts=5:sw=5:et:ff=unix:
%
% Permission is granted to copy, distribute and/or modify this document
% under the terms of the GNU Free Documentation License, Version 1.3 or
% any later version published by the Free Software Foundation.
%	
% A copy of the license is included in the file called "COPYING".
%%%%%%%%%%%%%%%%%%%%%%%%%%%%%%%%%%%%%%%%%%%%%%%%%%%%%%%%%%%%%%%%%%%%%%%%%%%%%%%%
\documentclass[10pt,a4paper,openany]{book}
\usepackage[utf8]{inputenc}		% se for usar no windows mude esta linha
\usepackage[portuges]{babel}	% idioma utilizado

\usepackage[T1]{fontenc}		% fonte computer modern
% se alguém ver alguma diferença, desabilite o comentário abaixo.
%\usepackage{ae}			% pacote de fontes opcionais

\usepackage{graphicx}			% permite inserir figuras
\usepackage{xy}                    % acho que é o pacote necessário para colocar as figuras das teclas
\usepackage{subfigure}			% suporte a subfiguras
\usepackage{color}			% suporte a cor

\usepackage{longtable}			% tabelas longas
\usepackage{multirow}			% suporte a múltiplas colunas


\newcommand{\mc}[3]{\multicolumn{#1}{#2}{#3}}	% novo comando para alinhamento de colunas
\newcommand{\mr}[3]{\multirow{#1}{#2}{#3}}	% novo comando pára alinhamento de linhas 

\RequirePackage{enumerate}		% extensão para o pacote enumerate
\usepackage{marvosym}			% gera alguns símbolos
\usepackage{pifont}			% simbolos de setas
\usepackage{fancyhdr}			% pacote para personalizar o layout

\usepackage{makeidx}			% inserir índice remissivo
\makeindex				% construindo arquivo .idx

\usepackage{appendix}			% suporte a apendices

\usepackage{url}			% inserir urls
\usepackage{hyperref}			% gera os links no pdf
\hypersetup{
pdfpagemode=UseOutlines,
colorlinks=true,
a5paper,
breaklinks=true,
linkcolor=blue,
anchorcolor=black,
citecolor=green,
filecolor=magenta,
menucolor=red,
pagecolor=red,
urlcolor=cyan,
bookmarksopen=false,
pdfpagelayout=SinglePage,
pdfpagetransition=Dissolve
}				% configurando o hyperref

% Comando \vimhelp
% Uso: \vimhelp{termo_1, termo_2, ..., termo_n}
%      onde termo_i é um argumento válido para o comando :help do Vim, isto é,
%      ':help termo_i' precisa abrir o help para o termo_i.
%
% Exemplo de uso: 
%   \chapter{Verificação Ortográfica}
%   \vimhelp{spell, spelllang}
%
%   Será impresso mais ou menos assim:
%
%   Verificação Ortográfica
%                                                          :h spell, spelllang
%
% O comando \vimhelp indica ao leitor como buscar ajuda dentro do próprio Vim
% para o assunto em questão. Uso indicado para capítulos e seções.
\newcommand{\vimhelp}[1]{\nopagebreak\par\addvspace{-1.5ex}%
     \vskip -\parskip\noindent{%
     \begin{flushright}\scriptsize \textbf{:h} \textsf{#1}\end{flushright}}\par\nopagebreak%
     \addvspace{1ex plus 0.8ex minus 0.2ex}% 
     \vskip -\parskip\noindent\ignorespaces}

% novo comando para teclas de atalho: para usar control-n faça --> \key{Ctrl}+\key{N} 
%\newcommand{\key}[1]{\mbox{\rule[-8.5pt]{0pt}{18pt}}\rule{1pt}{0pt}\framebox[\width]{\vphantom{Yg}\hphantom{-}\texttt{#1}\hphantom{-}}\rule{1pt}{0pt}}
\newcommand{\key}[1]{\rule[-5.5pt]{0pt}{16pt}\rule{1pt}{0pt}\raisebox{2.5pt}{\xymatrix@1{*+[F-:<2pt>]{\rule[-2.5pt]{0pt}{11pt}\hphantom{|}\texttt{#1}\hphantom{|}}}\rule{1pt}{0pt}}}

% Defininindo o estilo de página
\pagestyle{fancy}
\renewcommand{\chaptermark}[1]{\markboth{#1}{}}
\renewcommand{\sectionmark}[1]{\markright{\thesection\ #1}{}}
\fancyhf{}                              % eliminando a configuração anterior
\fancyhead[LE,RO]{\bfseries\thepage}	% núm. a esq. na pág. da esq. e a direita em contrário
\fancyhead[LO]{\bfseries\rightmark}	% para página da esquerda, o título fica a direita
\fancyhead[RE]{\bfseries\leftmark}	% para página da direita, o título fica a esquerda
\renewcommand{\headrulewidth}{0.5pt}
\renewcommand{\footrulewidth}{0.0pt}
\addtolength{\headheight}{2.5pt}    % 0.5pt
\fancypagestyle{plain}{
     \fancyhead{}
     \renewcommand{\headrulewidth}{0pt}
     }

\newcommand{\dica}[1]{{{\large \ding{42}}}}	% inserir imagem para dica, provido por pifont

\setlength{\parindent}{0em} % indentação dos parágrafos
\setlength{\parskip}{5pt plus 1pt minus 1pt} 


\input xy
\xyoption{all}

% Numeração romana para as páginas iniciais
\pagenumbering{roman}

\begin{document}
\DeclareGraphicsExtensions{.jpg,.pdf,.mps,.png,.eps}	% extensões de imagens

%%%%%%%%%%%%%%%%%%%%%%%%%%%%%%%% CAPA %%%%%%%%%%%%%%%%%%%%%%%%%%%%%%%%%%
%\begin{titlepage}
\thispagestyle{empty}
\begin{figure}[htp]
    \centering
    % o ideal seria o viewport no lugar de bb
    \includegraphics[bb=100 0 500 716]{img/capa.png}	
    %\includegraphics[scale=0.5]{img/capa.png}
    %\includegraphics[width=\paperwidth,height=\paperheight]{img/capa.png}
\end{figure}
\clearpage
%\cleardoublepage
%\end{titlepage}

%%%%%%%%%%%%%%%%%%%% Título do documento %%%%%%%%%%%%%%%%%%%%%%%
\thispagestyle{empty}
%\begin{titlepage}
\begin{center}

{\Huge \sc o editor de texto Vim}

% logo do Vim
  \vspace{2cm}
  \begin{figure}[h]
    \center
    \includegraphics{img/vimlogo.png}
    \label{logodovim}
\end{figure}

   \vspace{5cm}
   \begin{flushright}
   \begin{minipage}[t]{8cm}
          ``Um livro escrito em português sobre o editor de texto {\bf Vim}. A ideia é que este
          material cresça e torne-se uma referência confiável
          e prática. Use este livro nos termos da {\em Licença de Documentação Livre GNU} (GFDL).'' \\
          \par Este trabalho está em constante aprimoramento, e é fruto da
          colaboração de voluntários. Participe do desenvolvimento enviando sugestões e
          melhorias; acesse o site do projeto no endereço: \\
\begin{center}
          \url{http://code.google.com/p/vimbook}
\end{center}
        
   \end{minipage} \\
   \end{flushright}

   \vspace{3cm}

   {\small Versão gerada em \\ \bf \today} % especifica a versão do PDF pelo dia que o gerou.
\end{center}
%\end{titlepage}
%%%%%%%%%%%%%%%%%%%%%%%%%%%%%%%%%%%%%%%%%%%%%%%%%%%%%%%

\newpage
\thispagestyle{empty}

\begin{center}
{\Huge \bf Autores}

\vspace{2cm}

\begin{tabular}{cc}
\bf Sérgio Luiz Araújo Silva 	& \tt <voyeg3r@gmail.com> 			\\
\bf Douglas Adriano Augusto 	& \tt <daaugusto@gmail.com>			\\
\bf Eustáquio Rangel 			& \tt <eustaquiorangel@gmail.com> 	\\
\bf Eduardo Otubo 				& \tt <eduardo.otubo@gmail.com> 	\\
\bf Gustavo Dutra 				& \tt <gustavotkg@gmail.com> 		\\
\bf João Felipe Mitre 			& \tt <jfmitre@gmail.com> 			\\
$\vdots$ 						& $\vdots$ 							\\
\end{tabular}

\end{center}

\newpage
\tableofcontents

% Numeração arábica
\clearpage
\pagebreak
\pagenumbering{arabic}
% \setcounter{page}{0} %reset the page counter

%%%%%%%%%%%%%%%%%%%%%%%%%%%%%%%%%%%%%%%%%%%%%%%%%%%%%%%%%%%%%%%%%%%%%%%%
% vim:enc=utf-8:ts=5:sw=5:et:ff=unix:
%%%%%%%%%%%%%%%%%%%%%%%%%%%%%%%%%%%%%%%%%%%%%%%%%%%%%%%%%%%%%%%%%%%%%%%%

\chapter{Introdução}
%
A edição de texto é uma das tarefas mais frequentemente executadas por seres
humanos em ambientes computacionais, em qualquer nível. Usuários finais,
administradores de sistemas, programadores de software, desenvolvedores {\em
web}, e tantas outras categorias, todos eles, constantemente, necessitam
editar textos. 

Usuários finais editam texto para criar documentos, enviar e-mails, atualizar
o blog, escrever recados ou simplesmente trocar mensagens instantâneas pela
internet. Administradores de sistemas editam arquivos de configuração, criam
regras de segurança, editam {\em scripts} e manipulam saídas de comandos
armazenados em arquivos de texto. Programadores desenvolvem códigos-fonte e a
documentação de programas essencialmente em editores de texto.  Desenvolvedores
{\em web} interagem com editores de texto para criarem {\em layout} e dinâmica 
de sites.

Tamanha é a frequência e onipresença da tarefa de edição de texto que a
eficiência, flexibilidade e o repertório de ferramentas de editores de texto
tornam-se quesitos críticos para se atingir {\em produtividade} e {\em
conforto} na edição de textos.

% falar da não trivialidade do aprendizado (curva de aprendizado), mas também
% da eficiência e produtividade a médio/longo prazo
% Mitre - adicionando essa sugestão acima... 
%         requer revisão, quem o fizer, elimine esses comentários.
Qualquer tarefa de aprendizado requer um certo esforço. Todo programa 
introduz novos conceitos, opções e configurações que transformam o \textit{
modus operanti} do usuário. Em princípio, quanto maior o esforço, maior o
benefício. Quem quer apenas escrever textos, pode-se contentar com um editor
básico, cuja as únicas opções são digitar o texto, abrir e salvar o documento 
ou pode utilizar um editor que permita pré-configurar ações, formatar o 
conteúdo, revisar a ortografia, etc, além da ação básica que é escrever textos.

Qualquer usuário de computador pode abrir o primeiro tipo de editor e
imediatamente começar a escrever, a curto prazo, sua ação terá consequências
imediatas e não requer conhecimentos adicionais. Por outro lado, esse usuário
terá que fazer esforço para digitar o mesmos cabeçalho todos os dias. 

O outro tipo de editor permite que o usuário pré-configure o cabeçalho do
documento e todos os dias esse trecho já estará digitado. Em contrapartida, o
usuário deve aprender como pré-configurar o editor. O que requer esforço para
aprender a utilizar o programa escolhido. O benefício somente será observado a
médio/longo prazo, quando o tempo ganho ao utilizar a configuração será
superior ao tempo consumido aprendendo sobre o programa.
% ---------------------

O ``\href{http://www.vim.org}{Vim}''\footnote{Vim - \url{http://www.vim.org}}
\index{vim} é um editor de texto extremamente configurável, criado para
permitir a edição de forma eficiente, tornando-a produtiva e confortável. 
Também é uma aprimoração do editor ``Vi'', um tradicional programa dos
sistemas Unix. Possui uma série de mudanças em relação a este último. O
próprio slogan do Vim é {\em Vi IMproved}, ou seja, {\em Vi Melhorado}.  O Vim
é tão conhecido e respeitado entre programadores, e tão útil para programação,
que muitos o consideram uma verdadeira ``IDE\index{ide} (\textit{Integrated 
Development Environment}, em português, Ambiente Integrado de 
Desenvolvimento''.

Ele é capaz de reconhecer mais de 500 sintaxes de linguagens de programação e
marcação, possui mapeamento para teclas, macros, abreviações, busca por
{\em{Expressões
Regulares}}\index{expressões regulares}\footnote{Expressões Regulares - 
\url{http://guia-er.sourceforge.net/guia-er.html}}, entre outras facilidades.

% NOTA: não estou convencido sobre a relevância desse conteúdo, fica pelo menos
% como um o primeiro exemplo.
% referenciando a figura. O til serve para evitar que o número fique 
% em um linha e o nome em outro.
% toda referência é feita antes de inserir a figura.
A figura~\ref{fig:vimedittex} mostra o vim sendo usando para editar o arquivo
o desse livro sobre vim.

% Inserindo apenas uma figura
% A opção [htp] é o melhor conjunto para livros, pois evita que a figura
% se afaste muito do texto onde é citado.
\begin{figure}[htp]
  % centralizando a figura
  \centering 
  % especficiando a figura
  % a figura não precisa ter extensão
  % width : define o tamanho lateral
  %         use a unidade que conhecer, px, m, mm, cm, in, etc.
  %         usar apenas essa opção assegura que a proporção seja mantida.
  %         é possível especificar a outra dimensão ou uma fator global, 
  %         exemplos estão comentados, para height (altura) e scale (escala)
  % tudo que está entre colchetes é opcional, e existem muitas opções para 
  % esse campo, mas o que está abaixo é suficiente em 90 % dos casos
  \includegraphics[width=9cm]{img/vimedittex} % 9 cm de lado
  %\includegraphics[height=7cm]{img/vimedittex} % 7 cm de altura
  %\includegraphics[scale=0.4]{img/vimedittex} % 40 % do tamanho real
  % legenda da figura, a legenda vem depois da figura.
  \caption{Usando o vim para editar o código em \LaTeX}
  % rotulado essa figura, é usando por \ref{} para citar a figura.
  \label{fig:vimedittex}
\end{figure}

O Vim conta com uma comunidade bastante atuante e é, ao lado do
Emacs\footnote{Emacs - \url{http://www.gnu.org/software/emacs/}}, um dos 
editores mais usados nos sistemas GNU/Linux\footnote{O kernel Linux sem os 
programas GNU não serviria para muita coisa.}, embora esteja também disponível
em outros sistemas, como o Windows e o Macintosh. 
%O site oficial do Vim é \url{http://www.vim.org}.

\section{Instalação do Vim}\index{vim!instalar}
\vimhelp{install}
%
\subsection{Instalação no Windows}
%
Há uma versão gráfica do Vim disponível para vários sistemas operacionais, 
incluindo o Windows; esta versão pode ser encontrada no 
\href{http://www.vim.org/download.php}{site oficial}~\cite{SiteOficialDownloads}. 
Para instalá-lo basta baixar o instalador no link indicado e dispará-lo com um
duplo clique (este procedimento requer privilégios de administrador).

\subsection{Instalação no GNU/Linux}
%
A maioria das distribuições GNU/Linux traz o Vim em seus repositórios, sendo
que é bastante comum o Vim já vir incluído na instalação típica da distribuição.
A forma de instalação preferível depende do Vim:
\begin{itemize}
\item Já vir instalado por {\em default} -- neste caso nada precisa ser feito.

\item Estar disponível no repositório, mas não instalado -- em distribuições
derivadas da Debian GNU/Linux\footnote{Debian GNU/Linux - \url{http://www.debian.org/index.pt.html}},
a instalação do Vim através dos repositórios é usualmente executada
digitando-se {\tt `apt-get install vim'}\footnote{Recomenda-se também instalar
a documentação em HTML do Vim: {\tt `apt-get install vim-doc'}} em um {\em terminal} (este procedimento
requer privilégios de administrador e, tipicamente, conexão com a internet).
 
Algumas distribuições GNU/Linux dividem o programa vim em vários pacotes. 
Pacotes adicionais como \texttt{gvim} e \texttt{vim-enhanced}, entre outros,
representam diferentes versões do mesmo aplicativo. O  \texttt{gvim} é a versão
gráfica do Vim e o \texttt{vim-enhanced} é uma versão do vim compilada com um
suporte interno ao Python\footnote{O Python (\url{http://www.python.org}) é uma
linguagem de programação orientada a objetos muito comum no meio profissional e
acadêmico}.
A alternativa para resolver esse problema é buscar na documentação da 
distribuição o que significa cada pacote.

\item Não estar disponível no repositório da distribuição -- cenário {\em muito}
improvável, mas nas sua ocorrência o Vim pode ser instalado através da compilação do
código-fonte; basta seguir as instruções do 
\href{http://www.vim.org/download.php}{site oficial}~\cite{SiteOficialDownloads}.

\end{itemize}

\section{Dicas iniciais}\label{Dicas iniciais}
%
Ao longo do livro alguns comandos ou dicas podem estar duplicados, o que
é útil devido ao contexto e também porque o aprendizado por saturação
é um ótimo recurso. Ao perceber uma dica duplicada, antes de
reclamar veja se já sabe o que está sendo passado. % Tem certeza que esse
% comentário é necessário, parece ofensivo...
Contudo dicas e sugestões serão bem vindas! 

Para abrir um arquivo\index{iniciar} com Vim digite num terminal:
%
\begin{verbatim}
     vim texto.txt
\end{verbatim}
onde {\tt texto.txt} é o nome do arquivo que deseja-se criar ou editar.

Em algumas distribuições, pode-se usar o comando {\tt vi} ao invés de {\tt vim}.

\section{Ajuda integrada}
%
O Vim possui uma ajuda\index{ajuda}\index{manual} integrada muito completa, são mais
de 100 arquivos somando milhares de linhas. O único inconveniente é não haver ainda
tradução para o português, sendo o inglês seu idioma oficial; entretanto, as explicações
costumam ser sintéticas e diretas, de forma que noções em inglês seriam
suficientes para a compreensão de grande parte do conteúdo da ajuda integrada.

Obs: No Vim quase todos os comandos podem ser abreviados, no caso
``\verb+help+'' pode ser chamado por ``\verb+h+'' e assim por diante. Um
comando só pode ser abreviado até o ponto em que este nome mais curto não
coincida com o nome de algum outro comando existente.  Para chamar a ajuda do
Vim pressione `\texttt{Esc}' e em seguida:
%Vim pressione \verb|<Esc>| e em seguida:
%
\begin{verbatim}
     :help .... versão longa, ou
     :h ....... versão abreviada
\end{verbatim}
%
ou simplesmente `\texttt{F1}'.

Siga os links usando o atalho `\verb|ctrl+]|', em modo gráfico o clique com o
mouse também funciona, e para voltar use `\verb|ctrl+o|' ou
`\verb|ctrl+t|' Para as situações de desespero pode-se digitar:

\begin{verbatim}
     :help!
\end{verbatim}

{\Large {\ding{45}}} Quando um comando puder ser abreviado poderá aparecer
desta forma: `\texttt{:so[urce]}'. Deste modo se está indicando que o comando 
`\texttt{:source}' pode ser usado de forma abreviada, no caso `\texttt{:so}'.

\section{Em caso de erros }\label{Em caso de erros }
%
Recarregue\index{em caso de erros} o arquivo que está sendo editado pressionando
`\verb|Esc|' e em seguida usando o comando `\texttt{:e}'.
ou simplesmente inicie outro arquivo ignorando o atual, com o comando `\texttt{:enew!}',
ou saia do arquivo sem modifica-lo, com `\texttt{:q!}'. Pode-se ainda tentar gravar
forçado com o comando `\texttt{:wq!}'


\section{Como interpretar atalhos e comandos}\label{Como interpretar atalhos e comandos}
%
A tecla ``\verb|<Ctrl>|''\index{tecla!\texttt{<ctrl>}} é representada na maioria dos manuais e na ajuda
pelo caractere ``\verb|^|'' circunflexo, ou seja, o atalho \verb|Ctrl-L| aparecerá assim:
\begin{verbatim}
     ^L
\end{verbatim} % aqui pula-se linha, porque abaixo começa um novo parágrafo

No arquivo de configuração do Vim, um ``\verb|<Enter>|'' pode aparecer como:
\begin{verbatim}
     <cr>
\end{verbatim}

Para saber mais sobre como usar atalhos\index{atalhos}\index{mapeamento} no Vim
veja a seção \ref{Mapeamentos} na página~\pageref{Mapeamentos} e para ler sobre
o arquivo de configuração veja o capítulo \ref{cha:Como editar preferências no
Vim} na página \pageref{cha:Como editar preferências no Vim}.

\section{Modos de operação}\label{Modos de operação}

A tabela abaixo mostra uma referência rápida para os modos de operação\index{modos de operação}do Vim,
a seguir mais detalhes sobre cada um dos modos. 

\begin{tabular}{|l|l|l|}
\hline
\textbf{Modo} & \textbf{Descrição} & \textbf{Atalho} \tabularnewline
\hline \hline
Normal\index{modo normal} & Para deletar, copiar, formatar, etc & 
                            {\tt <Esc>}\tabularnewline \hline
                            %
Inserção\index{modo de inserção} & Prioritariamente, digitação de texto &
                            {}{\tt i,a,I,A,o,O}\tabularnewline \hline
                            %
Visual\index{modo visual} & Seleção de blocos verticais e linhas inteiras &
                            {}{\tt V, v, Ctrl-v} \tabularnewline \hline
                            %
Comando\index{modo de comando} & Uma verdadeira linguagem de programação &
                            {}{\tt <Esc>:}\tabularnewline \hline
                            %
\end{tabular}

Em oposição à esmagadora maioria dos editores o Vim é um editor que trabalha
com ``modos de operação (modo de inserção, modo normal, modo visual e etc)'', o
que a princípio dificulta a vida do iniciante, mas abre um universo de
possibilidades, pois ao trabalhar com modos distintos uma tecla de atalho pode
ter vários significados, exemplificando\index{modos de operação!exemplos}: Em
modo normal pressionar `{\tt dd}' apaga a linha atual, já em modo de inserção
ele irá se comportar como se você estivesse usando qualquer outro editor, ou
seja, irá inserir duas vezes a letra `{\tt d}'.  Em modo normal pressionar a
tecla `{\tt v}' inicia uma seleção visual (use as setas de direção).  Para sair
do novo visual \verb|<Esc>|.  Como um dos princípios do vim é agilidade pode-se
usar ao invés de setas (em modo normal) as letras {\tt h,l,k,j} como se fossem
setas:

\begin{verbatim}
         k
     h       l
         j
\end{verbatim}

Imagine as letras acima como teclas de direção, a letra `{\tt k}' é uma seta acima
a letra `{\tt j}' é uma seta abaixo e assim por diante.

\section{Entrando em modo de edição}\label{Entrando em modo de edição}
\index{modo de inserção}
Estando no modo normal, digita-se:
\begin{verbatim}
     a .... inicia inserção de texto após o atual
     i .... inicia inserção de texto antes do caractere atual
     A .... inicia inserção de texto no final da linha
     I .... inicia inserção de texto no começo da linha
     o .... inicia inserção de texto na linha abaixo
     O .... inicia inserção de texto na linha acima
\end{verbatim}

Outra possibilidade é utilizar a tecla \verb|<Insert>| para entrar no modo de inserção de
texto antes do caractere atual, ou seja, o mesmo que a tecla \verb|i|. Uma vez no modo de 
inserção, a tecla \verb|<Insert>| permite alternar o modo de digitação de inserção de 
simples de caracteres para substituição de caracteres.

Agora começamos a sentir o gostinho de usar o Vim, uma tecla seja
maiúscula ou minúscula, faz muita diferença se você não estiver em
modo de inserção, e para sair do modo de inserção e voltar ao modo normal sempre 
use \verb|<Esc>|.

\section{Erros comuns}\label{sec:Erros comuns}
\index{modos de operação!errors comuns}
\begin{itemize}

\item Estando em {\em{modo de inserção}} pressionar `{\tt j}' na intenção
de rolar o documento, neste caso estaremos inserindo simplesmente a letra `{\tt j}'. 

\item Estando em {\em{modo normal}} acionar acidentalmente o ``\verb+<Caps Lock>+'' 
e tentar rolar o documento usando a letra ``\verb+J+'', o efeito é a
junção das linhas, aliás um ótimo recurso quando a intenção é de fato esta.

\item Em {\em{modo normal}} tentar digitar {\em{um número seguido de uma palavra}} e ao perceber que 
nada está sendo digitado, iniciar o modo de inserção, digitando por fim o que se queria, 
o resultado é que o número que foi digitado inicialmente vira um quantificador para o que 
se digitou ao entrar no modo de inserção. A palavra aparecerá repetida na quantidade do 
número digitado. Assim, se você quiser digitar 10 vezes ``\verb+isto é um teste+''
 faça assim:

\begin{verbatim}
     <Esc> ........... se assegure de estar em modo normal
     10 .............. quantificador
     i ............... entra no modo de inserção
     isto é um teste <Enter> <Esc>  
\end{verbatim}

\end{itemize}

{\Large {\ding{45}}} Alguns atalhos úteis\dots
\index{atalhos}
\begin{verbatim}
     Ctrl-O ..... comando do modo normal no modo insert
     i Ctrl-a ... repetir a última inserção
     @: ......... repetir o último comando
     Shift-insert colar texto da área de transferência
     gi ......... modo de inserção no mesmo ponto da última vez
     gv ......... repete seleção visual
\end{verbatim}

Para saber mais sobre repetição de comandos veja o capítulo \ref{Repetição de comandos},
na página \pageref{Repetição de comandos}.

No Vim, cada arquivo aberto é chamado de \verb|buffer|, ou seja, dados
carregados na memória. Você pode acessar o mesmo {\em buffer} em mais de uma
janela, bem como dividir a janela em vários {\em buffers} distintos o que veremos
mais adiante.



%%%%%%%%%%%%%%%%%%%%%%%%%%%%%%%%%%%%%%%%%%%%%%%%%%%%%%%%%%%%%%%%%%%%%%%%
% vim:enc=utf-8:ts=5:sw=5:et:ff=unix:
%%%%%%%%%%%%%%%%%%%%%%%%%%%%%%%%%%%%%%%%%%%%%%%%%%%%%%%%%%%%%%%%%%%%%%%%

\chapter{Editando}
\label{Editando}
\index{editando}A principal função de um editor de textos é editar textos. 
Parece óbvio, mas em meio a inúmeros recursos extras essa simples e crucial função
perde-se entre todos os demais.

\section{Abrindo o arquivo para a edição}
Portanto, a primeira coisa a fazer é abrir um arquivo.
Como visto, para abrir um arquivo\index{iniciar} com Vim, digite em um terminal:
%
\begin{verbatim}
     vim texto.txt
\end{verbatim}
onde {\tt texto.txt} é o nome do arquivo que deseja-se criar ou editar.

Caso deseje abrir o arquivo na linha 10\index{iniciar!linha específica}, usa-se:
\begin{verbatim}
     vim +10 /caminho/para/o/arquivo
\end{verbatim}
se quiser abrir o arquivo na linha que contém um determinado padrão
\index{iniciar!padrão específico}, digite:
\begin{verbatim}
     vim +/padrão arquivo
\end{verbatim}

{\Large {\ding{45}}} Caso o padrão tenha espaços no nome coloque entre parênteses ou
use {\em escape} ``$\backslash$'' a fim de não obter erro.

\section{Escrevendo o texto}
O Vim é um editor que possuí diferentes modos de edição. Entre eles está o modo
de inserção, que é o modo onde escreve-se o texto naturalmente.

Para se entrar em modo de inserção, estando em modo normal, pode-se pressionar 
qualquer uma das teclas abaixo:
\index{modos de operação}
\begin{verbatim}
     i ..... entra no modo de inserção antes do caractere atual
     I ..... entra no modo de inserção no começo da linha
     a ..... entra no modo de inserção após o caractere atual
     A ..... entra no modo de inserção no final da linha
     o ..... entra no modo de inserção uma linha abaixo
     O ..... entra em modo de inserção uma linha cima
     <Esc> . sai do modo de inserção
\end{verbatim}

Uma vez no modo de inserção\index{modo de inserção} todas as teclas são exatamente 
como nos outros editores simples, caracteres que constituem o conteúdo do texto 
sendo digitado. 
O que inclui as teclas de edição de caracteres.

Para salvar\index{salvar o texto} o conteúdo escrito, digite a tecla \verb|<Esc>| para 
sair do modo de inserção e digite o comando:
\begin{verbatim}
     :w 
\end{verbatim}
para gravar o conteúdo.

Caso queira sair do editor\index{fechar o programa}, digite o comando:
\begin{verbatim}
     :q
\end{verbatim}
caso tenha ocorrido modificações no arquivo desde que ele foi salvo pela última vez
haverpa uma mensagem informando que o documento foi modificado e não foi salvo,
 nesse caso, digite o comando: 
\begin{verbatim}
     :q! 
\end{verbatim}
para fechar o Vim {\bf sem salvar} as últimas modificações feitas.

Caso queira salvar e sair do arquivo, digite o comando:
\begin{verbatim}
     :wq
\end{verbatim}

Nesse ponto, conhece-se o vim de forma suficiente para editar
qualquer coisa nele. Daqui por diante o que existe são as formas de
realizar a edição do arquivo com maior naturalidade e produtividade.

O usuário iniciante do Vim pode cometer o erro de tentar decorar 
todos os comandos que serão apresentados. {\bf Não faça isso}. Tentar decorar
comando é exatamente o caminho contrário da naturalidade exigida por 
um editor texto para aumentar a produtividade.

Ao contrário, sugere-se que leia-se todo o conteúdo. Identifique 
quais são as atividades de maior recorrência no estilo individual de 
escrita e busque como realizar tais funções com mais fluência nesse
editor. A prática levará ao uso de fluente desse comandos principais,
abrindo espaço para os demais comandos.

Isso não impede que o usuário experimente cada comando conforme for lendo.
De fato, essa prática pode ajudar a selecionar as formas de edição que lhe
são mais simpáticas ao uso.

\section{Copiar, Colar e Deletar}\label{sec:CopiarColarEDeletar}
\vimhelp{delete, d}
\index{deletar}
No modo normal, ato de deletar ou eliminar o texto está associado
à letra ``\verb|d|''. No modo de inserção as teclas usuais também 
funcionam.

ORIGINAL
\begin{verbatim}
     dd .... deleta linha atual
     D ..... deleta restante da linha
     d$ .... deleta do ponto atual até o final da linha
     d^ .... deleta do cursor ao primeiro caractere não-nulo da linha
     d0 .... deleta do cursor ao início da linha
\end{verbatim}

Modificado Tipo 1:
\begin{verbatim}
     dd .... deleta linha atual
     D ..... deleta restante da linha
     d$ .... deleta do ponto atual até o final da linha
     d^ .... deleta do cursor ao primeiro caractere não-nulo da 
             linha
     d0 .... deleta do cursor ao início da linha
\end{verbatim}

Modificado Tipo 2:
\begin{table}[htb]
\begin{center}
\begin{tabular}{ll}
\hline
\verb|dd| & deleta linha atual \\
\verb|D|  & deleta restante da linha \\
\verb|d$| & deleta do ponto atual até o final da linha \\
\mr{2}{*}{\texttt{d\^}} &  deleta do cursor ao primeiro caractere não-nulo da \\
                          & linha \\ 
\verb|d0| & deleta do cursor ao início da linha \\ \hline
\end{tabular}
\end{center}
\end{table} 

Contudo, como a tabela acima não tem tamanho fixo na fonte para a descrição, pode-se fazer
\begin{table}[htb]
\begin{center}
\begin{tabular}{ll}
\hline
\verb|dd| & deleta linha atual \\
\verb|D|  & deleta restante da linha \\
\verb|d$| & deleta do ponto atual até o final da linha \\
\verb|d^| & deleta do cursor ao primeiro caractere não-nulo da linha \\ 
\verb|d0| & deleta do cursor ao início da linha \\ \hline
\end{tabular}
\end{center}
\end{table} 

{\Large \ding{45}} Pode-se combinar o comando de deleção ``\verb+d+'' com o
comando de movimento (considere o modo normal) para apagar até a
próxima vírgula use: ``\verb+df,+''. \\

\index{copiar}Copiar está associado à letra ``\verb|y|''.

\begin{verbatim}
     yy .... copia a linha atual
     Y ..... copia a linha atual
     ye .... copia do cursor ao fim da palavra
     yb .... copia do começo da palavra ao cursor
\end{verbatim}

O que foi deletado ou copiado pode ser colado\index{colar}:
\begin{verbatim}
     p .... cola o que foi copiado ou deletado abaixo
     P .... cola o que foi copiado ou deletado acima
     [p ... cola o que foi copiado ou deletado antes do cursor
     ]p ... cola o que foi copiado ou deletado após o cursor
\end{verbatim}

\subsection{Deletando uma parte do texto}\label{Deletando uma parte do texto}
\vimhelp{deleting}

\index{deletar}O comando `{\tt d}' remove o conteúdo para a memória.

\begin{verbatim}
     x .... apaga o caractere sob o cursor
     xp ... troca letras de lugar
     ddp .. troca linhas de lugar
     d5x .. apaga os próximos 5 caracteres
     dd  .. apaga a linha atual
     5dd .. apaga 5 linhas (também pode ser: d5d)
     d5G .. apaga até a linha 5
     dw  .. apaga uma palavra
     5dw .. apaga 5 palavras (também pode ser: d5w)
     dl  .. apaga uma letra (sinônimo: x)
     5dl .. apaga 5 letras (também pode ser: d5l ou 5x)
     d0  .. apaga até o início da linha
     d^  .. apaga até o primeiro caractere da linha
     d$  .. apaga até o final da linha (sinônimo: D)
     dgg .. apaga até o início do arquivo
     dG  .. apaga até o final do arquivo
     D .... apaga o resto da linha
\end{verbatim}

Depois do texto ter sido colocado na memória, digite `{\tt p}' para `inserir' o
texto em uma outra posição. Outros comandos:

\begin{verbatim}
     diw .. apaga palavra mesmo que não esteja posicionado no início
     dip .. apaga o parágrafo atual
     d4b .. apaga as quatro palavras anteriores
     dfx .. apaga até o próximo ``x''
     d/casa/+1 - deleta até a linha após a palavra casa
\end{verbatim}

Trocando a letra `{\tt d}' nos comandos acima por `{\tt c}' de {\em change}
(``mudança'') ao invés de deletar será feita uma mudança de conteúdo.  Por
exemplo:
\vimhelp{change}

\begin{verbatim}
     ciw .............. modifica uma palavra
     cip .............. modifica um parágrafo
     cis .............. modifica uma sentença
     C ................ modifica até o final da linha
\end{verbatim}

\subsection{Copiando sem deletar}\label{Copiando sem deletar}
\vimhelp{yank}

\index{copiar}O comando `{\tt y}' ({\em yank}) permite copiar uma parte do 
texto para a memória sem deletar.  Existe uma semelhança muito grande entre 
os comandos `{\tt y}' e os comandos `{\tt d}', um ativa a `cópia' e outro a 
`exclusão' de conteúdo, suportando ambos quantificadores:

\begin{verbatim}
     yy  .... copia a linha atual (sinônimo: Y)
     5yy .... copia 5 linhas (também pode ser: y5y ou 5Y)
     y/pat .. copia até `pat'
     yw  .... copia uma palavra
     5yw .... copia 5 palavras (também pode ser: y5w)
     yl  .... copia uma letra
     5yl .... copia 5 letras (também pode ser: y5l)
     y^  .... copia da posição atual até o início da linha (sinônimo: y0)
     y$  .... copia da posição atual até o final da linha
     ygg .... copia da posição atual até o início do arquivo
     yG  .... copia da posição atual até o final do arquivo
\end{verbatim}

Digite `{\tt P}' (p maiúsculo) para \index{colar}colar o texto recém copiado na p
osição onde encontra-se o cursor, ou `{\tt p}' para colar o texto na posição 
imediatamente após o cursor.

\begin{verbatim}
     yi" .... copia trecho entre aspas (atual - inner)
     vip .... seleção visual para parágrafo atual 
              `inner paragraph'
     yip .... copia o parágrafo atual
     yit .... copia a tag agual `inner tag' útil para arquivos 
              HTML, XML, etc.
\end{verbatim}

\subsection{Usando a área de transferência {\em Clipboard}}
\vimhelp{paste, clipboard, quoteplus}

\index{copiar}\index{colar}\index{clipboard}
\index{área de transferência}Exemplos para o modo visual:

\begin{verbatim}
     Ctrl-insert .... copia área selecionada 
     Shift-insert ... cola o que está no clipboard
     Ctrl-del ....... recorta para o clipboard
\end{verbatim}

Caso obtenhamos erro ao colar textos da área de transferência usando os
comandos acima citados podemos usar outra alternativa.  Os comandos abaixo
preservam a indentação\footnote{Espaçamento entre o começo da linha e o início
do texto}.

\begin{verbatim}
     "+p ............ cola preservando indentação
     "+y ............ copia área selecionada
\end{verbatim}

{\Large \ding{45}} Para evitar erros ao colar usando {\tt Shift-insert} 
use este comando `{\tt :set paste}'.

\subsection{Removendo linhas duplicadas}
\index{deletar!linhas duplicadas}
\begin{verbatim}
     :sort u
\end{verbatim}

\section{Forçando a edição de um novo arquivo}\label{sec:Forçando a edição de um novo arquivo}
\vimhelp{edit!}

\index{iniciar!novo arquivo}
O Vim, como qualquer outro editor, é muito exigente no que se refere a alterações
de arquivo.  Ao tentar abandonar um arquivo editado e não salvo, o Vim irá se certificar
da ação. Para abrir um novo arquivo sem salvar o antigo:

\begin{verbatim}
     :enew!
\end{verbatim}

O comando acima é uma abreviação de {\em edit new}. De modo similar pode-se
ignorar todas as alterações feitas desde a abertura do arquivo:

\begin{verbatim}
     :e!
\end{verbatim}

\section{Ordenando}\index{ordenando}
\vimhelp{sort}

O Vim, versão 7 ou superior, passa a ter um comando de ordenação que também 
permite a retirada de linhas duplicadas, tal como foi apresentado.

\begin{verbatim}
     :sort u ... ordena e retira linhas duplicadas
     :sort n ... ordena numericamente
\end{verbatim}

Obs: a ordenação numérica é diferente da ordenação alfabética se em um
trecho contendo algo como:

\begin{verbatim}
     8
     9
     10
     11
     12
\end{verbatim}

Você tentar fazer:

\begin{verbatim}
     :sort
\end{verbatim}

O Vim colocará nas três primeiras linhas

\begin{verbatim}
     10
     11
     12
\end{verbatim}

Portanto lembre-se que se a ordenação envolver números use:

\begin{verbatim}
     :sort n
\end{verbatim}

Você pode fazer a ordenação em um intervalo assim:

\begin{verbatim}
     :1,15 sort n
\end{verbatim}

O comando acima diz ``{\em Ordene numericamente da linha 1 até a linha 15}''.  
Podemos ainda ordenar à partir de uma coluna:

\begin{verbatim}
     :sort /.*\%8v/   ..... ordena à partir do 8º caractere
\end{verbatim}

\section{Usando o \texttt{grep} interno do Vim}
\label{sec:Usando o grep interno do Vim}\index{grep}
\vimhelp{vimgrep, lvimgrep}

Para editar todos os arquivos que contenham a palavra ``inusitada'':

\begin{verbatim}
    :vimgrep /\cinusitada/ *
\end{verbatim}
a opção `\c' torna a busca indiferente a letras maiúsculas e minúsculas.

Obs: o Vim busca à partir do diretório atual, para se descobrir 
o diretório atual ou mudá-lo:

\begin{verbatim}
    :pwd ........... exibe o diretório atual
    :cd /diretório   muda de diretório
\end{verbatim}

\section{Lista de alterações}\index{lista de alterações}\index{histórico}
\vimhelp{changelist, changes}

O Vim mantém uma lista de alterações, veremos agora como usar este recurso.

\begin{verbatim}
     g, ................. avança na lista de alterações
     g; ................. recua na lista de alterações
     :changes ........... visualiza a lista de alterações
\end{verbatim}

\section{Substituindo tabulações por espaços}
\label{sec:Substituindo tabulações por espaços}
\vimhelp{expandtab, retab}

Se houver necessidade\footnote{Em códigos Python por exemplo não se pode
misturar espaços e tabulações} de trocar tabulações por espaços
fazemos assim:

\begin{verbatim}
	 :set expandtab
	 :retab
\end{verbatim}

Para fazer o contrário usamos algo como:

\begin{verbatim}
    :%s/\s\{4,}/<pressiona-se ctrl-i>/g
\end{verbatim}
onde
\begin{verbatim}
    <Ctrl-i>...... insere uma tabulação
\end{verbatim}

Explicando:
\begin{verbatim}
    : ............ comando
    % ............ em todo arquivo 
    s ............ substitua 
    / ............ padrão de busca
    \s ........... localiza espaço
    \{4,} ........ quatro vezes
    / ............ inicio da substituição
    <Ctrl-i> ..... pressione Ctrl-i para inserir <Tab>
    / ............ fim da substituição
    g ............ global
\end{verbatim}

\section{Convertendo para maiúsculas}
\label{sec:Convertendo para maiúsculas}
\vimhelp{case}

\begin{verbatim}
     gUU ....... converte a linha para maiúsculo
     guu ....... converte a linha para minúsculo
     gUiw ...... converte a palavra atual para maiúsculo
     ~ ......... altera o case do caractere atual
\end{verbatim}

\section{Editando em modo de comando}
\label{sec:Editando em modo de comando}
\vimhelp{put, move, global, holy-grail}

Para mover um trecho usando o modo de comandos faça:

\begin{verbatim}
     :10,20m $
\end{verbatim}

O comando acima move `{\tt m}' da linha 10 até a linha 20 para o final \verb|$|.

\begin{verbatim}
     :g /palavra/ m 0
\end{verbatim}

Move as linhas contendo `palavra' para o começo (linha zero)


\begin{verbatim}
     :10,20y a
\end{verbatim}

Copia da linha `10' até a linha `20' para o registro `a'

\begin{verbatim}
     :56pu a
\end{verbatim}

Cola o registro `a' na linha 56

\begin{verbatim}
     :g/padrão/d
\end{verbatim}

O comando acima deleta todas as linhas contendo a palavra `padrão'.

Podemos inverter a lógica do comando global \verb+g+:

\begin{verbatim}
     :g!/padrão/d
\end{verbatim}

Não delete as linhas contendo padrão, ou seja, delete tudo menos as linhas
contendo a palavra `padrão'. 

\begin{verbatim}
     :v/padrão/d ......... apaga linhas que não contenham `padrão'
     :v/\S/d ............. apaga linhas vazias
     \S .................. significa `sem string'
\end{verbatim}

A opção acima equivale a ``\verb+:g!/padrão/d+''.  Para ler mais sobre
o comando ``global'' utilizado nesta seção veja o capítulo~\ref{sec:O comando global ``g''}.

\begin{verbatim}
     :7,10copy $
\end{verbatim}

Da linha 7 até a linha 10 copie para o final. {\Large \ding{45}}
Veja mais sobre edição no modo de comando na seção ``\ref{cha:Buscas e
substituições} Buscas e substituições na página~\pageref{cha:Buscas e substituições}''.

\subsubsection{Gerando sequências}
Para inserir uma sequência de 1 a 10 à partir da linha inicial ``zero'' fazemos:

\begin{verbatim}
     :0put =range(1,10)
\end{verbatim}

Caso queira inserir sequências como esta:

\begin{verbatim}
     192.168.0.1
     192.168.0.2
     192.168.0.3
     192.168.0.4
     192.168.0.5
\end{verbatim}

Usamos este comando:

\begin{verbatim}
     :for i in range(1,5) | .put ='192.168.0.'.i | endfor
\end{verbatim}

\section{O arquivo alternativo}
\label{O arquivo alternativo}
\vimhelp{Ctrl-6, alternate-file}

É muito comum um usuário concluir a edição em um arquivo no Vim e
inocentemente imaginar que não vai mais modificar qualquer coisa nele, então
este usuário abre um novo arquivo:

\begin{verbatim}
     :e novo-arquivo.txt
\end{verbatim}

Mas de repente o usuário lembra que seria necessário adicionar uma linha no
arquivo recém editado, neste caso usa-se o atalho

\begin{verbatim}
     Ctrl-6
\end{verbatim}

cuja função é alternar entre o arquivo atual e o último editado. Para retornar
ao outro arquivo basta portanto pressionar \verb|Ctrl-6| novamente. Pode-se 
abrir o arquivo alternativo em nova janela usando-se o atalho:

\begin{verbatim}
    Ctrl-w Ctrl-6
\end{verbatim}

\section{Lendo um arquivo para a linha atual}
\label{sec:Lendo um arquivo para a linha atual}
\vimhelp{:r[ead]}

Se desejamos inserir na linha atual um arquivo qualquer fazemos:

\begin{verbatim}
	 :r /caminho/para/arquivo.txt .. insere o arquivo na linha atual
	 :0r arquivo ................... insere o arquivo na primeira linha
\end{verbatim}


\section{Incrementando números em modo normal}\label{Incrementando números em modo normal}
\vimhelp{Ctrl-a, Ctrl-x}

Posicione o cursor sobre um número e pressione

\begin{verbatim}
     Ctrl-a ..... incrementa o número
     Ctrl-x ..... decrementa o número
\end{verbatim}

\section{Repetindo a digitação de linhas}
\label{Repetindo a digitação de linhas}

\begin{verbatim}
     " atalhos para o modo insert
     Ctrl-y ......... repete linha acima
     Ctrl-e ......... repete linha abaixo
     Ctrl-x Ctrl-l .. repete linhas inteiras
     Ctrl-a ......... repete a última inserção
\end{verbatim}

{\Large \ding{45}} Para saber mais sobre repetição de comandos veja o capítulo
\ref{Repetição de comandos}, na página \pageref{Repetição de comandos}.

\section{Movendo um trecho de forma inusitada}
\label{Movendo um trecho de forma inusitada}

\begin{verbatim}
     :20,30m 0 ..... move da linha `20' até `30' para o começo
     :20,/pat/m 5 .. move da linha `20' até `pat' para a linha 5
\end{verbatim}

\section{Uma calculadora diferente}
\label{Uma calculadora diferente}

Sempre que for necessário digitar  o resultado de uma expressão matemática
(portanto no modo de inserção) pode-se usar o atalho ``{\tt Ctrl-r =}'', ele
ativa o registro de expressões, na linha de comando do Vim aparece um sinal de
igual, digita-se então uma expressão matemática qualquer tipo ``{\tt 35*6}'' e
em seguida pressiona-se ``{\tt Enter}'', o Vim coloca então o resultado da
expressão no lugar desejado.  Portanto não precisa-se recorrer a nenhuma
calculadora para fazer cálculos.  Pode-se fazer uso do ``Registro de
Expressões'' dentro de macros, ou seja, ao gravar ações pode-se fazer uso deste
recurso, aumentando assim sua complexidade e poder! Para ler sobre ``macros''
acesse a seção \ref{sec:Gravando comandos} na \pageref{sec:Gravando comandos}.
Para saber mais sobre o ``registro de expressões'' leia a seção
\ref{sec:Registro de expressões "=} na página~\pageref{sec:Registro de expressões "=}.

{\Large {\ding{45}}} Na seção \ref{sec:Calculadora Científica com o Vim}
``Calculadora Científica com o vim'' página~\pageref{sec:Calculadora Científica com
o Vim} há uma descrição sobre como fazer cálculos com maior precisão e
complexidade.

{\Large {\ding{45}}} Se a intenção for apenas exibir um calculo na barra de comandos
é possível fazer algo assim:

\begin{verbatim}
    :echo 5.2 * 3
\end{verbatim}

\section{Desfazendo}
\label{Desfazendo}
\vimhelp{undo}

Se você cometer um erro, não se preocupe! Use o comando `{\tt u}':

\begin{verbatim}
     u ............ desfazer
     U ............ desfaz mudanças na última linha editada
     Ctrl-r  ...... refazer
\end{verbatim}

\subsection{{\em Undo tree}}
\label{Undo tree}

Um novo recurso muito interessante foi adicionado ao Vim ``a partir da
versão 7''  é a chamada árvore do desfazer.  Se
você desfaz alguma coisa, fez uma alteração um novo {\em branch} ou
galho, derivação de alteração é criado.  Basicamente, os {\em branches}
nos permitem acessar quaisquer alterações ocorridas no arquivo.

\subsubsection{Um exemplo didático}
\label{Um exemplo didático}

Siga estes passos (para cada passo \verb|<Esc>|, ou seja, saia do modo
de inserção)

\begin{description}
\item [Passo 1] - digite na linha 1 o seguinte texto
\begin{verbatim}
     # controle de fluxo <Esc>
\end{verbatim}

\item [Passo 2] - digite na linha 2 o seguinte texto
\begin{verbatim}
     # um laço for <Esc>
\end{verbatim}

\item [Passo 3] - Nas linhas 3 e 4 digite...

\begin{verbatim}
     for i in range(10):
         print i  <Esc>
\end{verbatim}

\item [Passo 4] - pressione `{\tt u}' duas vezes (você voltará ao passo 1)
\item [Passo 5] - Na linha 2 digite

\begin{verbatim}
     # operador ternário <Esc>
\end{verbatim}

\item [Passo 6] - na linha 3 digite

\begin{verbatim}
     var = (1 if teste == 0 else 2)  <Esc>
\end{verbatim}

\end{description}

Obs: A necessidade do {\tt Esc} é para demarcar as ações, pois o Vim
considera cada inserção uma ação.  Agora usando o atalho de desfazer
tradicional ``u'' e de refazer {\tt Ctrl-r} observe que não é mais possível
acessar todas as alterações efetuadas. Em resumo, se você fizer uma
nova alteração após um desfazer (alteração derivada) o comando refazer
não mais vai ser possível para aquele momento. \\

Agora volte até a alteração 1 e use seguidas vezes:

\begin{verbatim}
     g+
\end{verbatim}

e/ou

\begin{verbatim}
     g-
\end{verbatim}

Dessa forma você acessará todas as alterações ocorridas no texto.


\subsection{{\em Máquina do tempo}}
\label{Maquina do tempo}

Se não bastasse tudo isto o vim pode voltar o arquivo ao estado 
que estava há 10 minutos atrás, por exemplo:

\begin{verbatim}
    :ealier 10m ...... 10 minutos antes
    :later 30s ....... avança 30 segundos
\end{verbatim}

\section{Salvando}
\label{sec:Salvando}
\vimhelp{writing}

A maneira mais simples de salvar um arquivo, é usar o comando:

\begin{verbatim}
     :w
\end{verbatim}


Para especificar um novo nome para o arquivo, simplesmente digite:

\begin{verbatim}
     :w! >> ``file''
\end{verbatim}

O conteúdo será gravado no arquivo ``{\tt file}'' e você continuará no arquivo original.

Também existe o comando

\begin{verbatim}
     :saveas nome
\end{verbatim}

salva o arquivo com um novo nome e muda para esse novo arquivo (o arquivo
original não é apagado).  Para sair do editor, salvando o arquivo atual, digite
{\tt :x} (ou {\tt :wq}).

\begin{verbatim}
     :w ............................ salva
     :w `novonome' ................. salvar como
     :wq  .......................... salva e sai'
     :saveas nome .................. salvar como
     :x ............................ salva se existirem modificações
     :10,20 w! ~/Desktop/teste.txt . sava um trecho para outro arquivo
     :w! ........................... salvamento forçado
     :e! ........................... reinicia a edição ignorando alterações
\end{verbatim}


\section{Abrindo o último arquivo rapidamente}

O Vim guarda um registro para cada arquivo editado veja
mais no capítulo \ref{Registros} na página \pageref{Registros}.

\begin{verbatim}
     '0 ........ abre o último arquivo editado
     '1 ........ abre o penúltimo arquivo editado
     Ctrl-6 .... abre o arquivo alternativo (booleano)
\end{verbatim}

Bom, já que abrimos o nosso último arquivo editado com o comando

\begin{verbatim}
     `0
\end{verbatim}

podemos, e provavelmente o faremos, editar no mesmo ponto em que estávamos
editando da última vez:

\begin{verbatim}
     gi
\end{verbatim}

{\Large \ding{45}} Pode-se criar um `{\tt alias}'\footnote{Abreviação para um
comando do GNU/Linux} para que ao abrir o vim o mesmo abra o último arquivo
editado: `\verb|alias lvim="vim -c \"normal '0\""|'.  No capítulo
\ref{cha:Buscas e substituições} página \pageref{cha:Buscas e substituições}
você encontra mais dicas de edição.


\section{Modelines}\label{sec:Modelines}
\vimhelp{modeline}

São um modo de guardar preferências no próprio arquivo, suas
preferências viajam literalmente junto com o arquivo, basta usar em
uma das 5 primeiras linhas ou na última linha do arquivo algo
como:

\begin{verbatim}
     # vim:ft=sh:
\end{verbatim}

OBS: Você deve colocar um espaço entre a palavra `{\tt vim}' e a primeira
coluna, ou seja, a palavra `{\tt vim}' deve vir precedida de um espaço, daí
em diante cada opção fica assim:

\begin{verbatim}
     :opção:
\end{verbatim}

Por exemplo: posso salvar um arquivo com extensão \verb|.sh| e dentro do
mesmo indicar no {\em modeline} algo como:

\begin{verbatim}
     # vim:ft=txt:nu:
\end{verbatim}

Apesar de usar a extensão `{\tt sh}' o Vim reconhecerá este arquivo como `{\tt txt}', e
caso eu não tenha habilitado a numeração, ainda assim o Vim usará por causa da
opção `{\tt nu}'.  Portanto o uso de {\em modelines} pode ser um grande recurso para o seu
dia-a-dia pois você pode coloca-las dentro dos comentários!

\section{Edição avançada de linhas}

Seja o seguinte texto:

\begin{verbatim}
     1  este é um texto novo
     2  este é um texto novo
     3  este é um texto novo
     4  este é um texto novo
     5  este é um texto novo
     6  este é um texto novo
     7  este é um texto novo
     8  este é um texto novo
     9  este é um texto novo
     10 este é um texto novo
\end{verbatim}

Suponha que queira-se apagar ``{\tt é um texto}'' da linha 5 até o fim (linha
10). Isto pode ser feito assim:

\begin{verbatim}
     :5,$ normal 0wd3w
\end{verbatim}

Explicando o comando acima:

\begin{verbatim}
     :5,$ .... indica o intervalo que é da linha 5 até o fim `$'
     normal .. executa em modo normal
     0 ....... move o cursor para o começo da linha
     w ....... pula uma palavra
     d3w ..... apaga 3 palavras `w'
\end{verbatim}

Obs: É claro que um comando de substituição simples

\begin{verbatim}
     :5,$s/é um texto//g
\end{verbatim}

Resolveria neste caso, mas a vantagem do método anterior é que
é válido para três palavras, sejam quais forem.  
Também é possível empregar comandos de inserção como `{\tt i}' ou `{\tt a}' e
retornar ao modo normal, bastando para isso usar o recurso \verb|Ctrl-v Esc|,
de forma a simular o acionamento da tecla \verb|Esc| (saída do modo de
inserção). Por exemplo, suponha agora que deseja-se mudar a frase ``{\em este
é um texto novo}'' para ``{\em este não é um texto velho}''; pode ser feito
assim:

\begin{verbatim}
     :5,$ normal 02winão ^[$ciwvelho
\end{verbatim}

Decompondo o comando acima temos:

\begin{verbatim}
     :5,$ .... indica o intervalo que é da linha 5 até o fim `$'
     normal .. executa em modo normal
     0 ....... move o cursor para o começo da linha
     2w ...... pula duas palavras (vai para a palavra "é")
     i ....... entra no modo de inserção
     não  .... insere a palavra "não" seguida de espaço " "
     ^[ ...... sai do modo de inserção (através de Ctrl-v seguido de Esc)
     $ ....... vai para o fim da linha
     ciw ..... apaga a última palavra ("novo") e entra em modo de inserção
     velho ... insere a palavra "velho" no lugar de "novo"
\end{verbatim}

A combinação \verb|Ctrl-v| é utilizada para inserir caracteres de controle na
sua forma literal, prevenindo-se assim a interpretação destes neste exato
momento.

\section{Comentando rapidamente um trecho}

Tomando como exemplo um trecho de código como abaixo:

\begin{verbatim}
     1   input{capitulo1}
     2   input{capitulo2}
     3   input{capitulo3}
     4   input{capitulo4}
     5   input{capitulo5}
     6   input{capitulo6}
     7   input{capitulo7}
     8   input{capitulo8}
     9   input{capitulo9}
\end{verbatim}

Se desejamos comentar da linha 4 até a linha 9 podemos fazer:

\begin{verbatim}
     posicionar o cursor no começo da linha 4
     Ctrl-v ........... inicia seleção por blocos
     5j ............... estende a seleção até o fim
     Shift-i .......... inicia inserção no começo da linha
     % ................ insere comentário (LaTeX)
     Esc .............. sai do modo de inserção
\end{verbatim}

\section{Comparando arquivos com o vimdiff}
\label{sec:Comparando arquivos com o vimdiff}

\vimhelp{diff}

O vim possui um modo para checagem de diferenças entre arquivos, é bastante
útil especialmente para programadores, para saber quais
são as diferenças entre dois arquivos faz-se:

\begin{verbatim}
    vimdiff arquivo1.txt arquivo2.txt ... exibe as diferenças
    ]c .................................. mostra próxima diferença
    vim -d .............................. outro modo de abrir o vimdiff mode
\end{verbatim}

Para usuários do GNU/Linux é possível ainda checar diferenças remotamente assim:

\begin{verbatim}
    vimdiff projeto scp://usuario@estacao//caminho/projeto
\end{verbatim}

O comando acima irá exibir lado a lado o arquivo local chamado `{\tt projeto}' e o 
arquivo remoto contido no computador de nome `{\tt estacao}' de mesmo nome.



%%%%%%%%%%%%%%%%%%%%%%%%%%%%%%%%%%%%%%%%%%%%%%%%%%%%%%%%%%%%%%%%%%%%%%%%
% vim:enc=utf-8:ts=5:sw=5:et:ff=unix:
%%%%%%%%%%%%%%%%%%%%%%%%%%%%%%%%%%%%%%%%%%%%%%%%%%%%%%%%%%%%%%%%%%%%%%%%

\chapter{Movendo-se no Documento}
\label{cha:Movendo-se no Documento}
\vimhelp{motion.txt}

A fim de facilitar o entendimento acerca das teclas e atalhos de movimentação,
faz-se útil uma breve recapitulação de conceitos relacionados. Para se entrar
em modo de inserção, estando em modo normal, pode-se pressionar qualquer uma
das teclas abaixo:
\index{modos de operação}
\begin{verbatim}
     i ..... entra no modo de inserção antes do caractere atual
     I ..... entra no modo de inserção no começo da linha
     a ..... entra no modo de inserção após o caractere atual
     A ..... entra no modo de inserção no final da linha
     o ..... entra no modo de inserção uma linha abaixo
     O ..... entra em modo de inserção uma linha cima
     <Esc> . sai do modo de inserção
\end{verbatim}

\index{movendo-se!entre caracteres}
Uma vez no modo de inserção todas as teclas são exatamente como nos outros
editores simples, caracteres que constituem o conteúdo do texto sendo digitado.
Para sair do modo de inserção e retornar ao modo normal digita-se \verb+<Esc>+
ou \verb+Ctrl-[+. As letras {\tt h}, {\tt k}, {\tt l}, {\tt j} funcionam como
setas:

% A imagem que está aqui: http://www.linuxconfig.org/Vim_Tutorial
% ficaria muito boa no lugar desse verbatim abaixo.
% Apenas não a coloquei porque ela está em inglês e acho que não cabe.
% Fica como sugestão para alguém que saiba usar o gimp melhor que eu.
\begin{verbatim}
        k
     h     l
        j
\end{verbatim}
ou seja, a letra `{\tt k}' é usada para subir no texto, a letra `{\tt j}' para descer,
a letra `{\tt h}' para mover-se para a esquerda e a letra `{\tt l}' para mover-se para
a direita. A ideia é que se consiga ir para qualquer lugar do texto sem tirar
as mãos do teclado, sendo portanto alternativas para as setas de movimentação
usuais do teclado. Ao invés de manter os quatro dedos sobre H, J, K e L, é aconselhável 
manter o padrão de digitação com o indicador da mão esquerda sobre a tecla F e o da mão
direita sobre a letra J, sendo que seu indicador da mão direita vai alternar entre a
tecla J e H para a movimentação. 

\index{movendo-se!entre linhas}
Para ir para linhas específicas `{\em em modo normal}' digite:
\begin{verbatim}
     :n<Enter>  ..... vai para linha `n'
     ngg ............ vai para linha `n'
     nG ............. vai para linha `n'
\end{verbatim}
onde ``\verb|n|'' corresponde ao número da linha.  Para retornar ao modo normal
pressione \verb|<Esc>| ou use \verb|Ctrl-[| (\verb|^[|).

%\section{Os saltos}\label{Os saltos}
% a idéia é simplificar.
% sections demais prejudica o entendimento
% misturar o comando básico com o avançado afasta o usuário iniciante.
% eliminar o comando avançado afasta o usuário avançado
% O ideal é separar o básico (mais comum) do avançado (pouco comum para quem apenas edita texto) 
\index{movendo-se!efetuando saltos}
No vim é possível realizar diversos tipos de movimentos, também conhecidos como saltos no 
documento. A lista abaixo aponta o comandos de salto típicos.
\begin{verbatim}
     gg .... vai para o início do arquivo
     G ..... vai para o final do arquivo
     0 ..... vai para o início da linha
     ^ ..... vai para o primeiro caractere da linha (ignora 
             espaços)
     $ ..... vai para o final da linha
     25gg .. salta para a linha 25
     '' .... salta para a linha da última posição em que o cursor
             estava
     fx .... para primeira ocorrência de x
     tx .... Para ir para uma letra antes de x
     Fx .... Para ir para ocorrência anterior de x
     Tx .... Para ir para uma letra após o último x
     * ..... Próxima ocorrência de palavra sob o cursor
     `' .... salta exatamente para a posição em que o cursor
             estava
     gd .... salta para declaração de variável sob o cursor
     gD .... salta para declaração (global) de variável sob o 
             cursor
     w ..... move para o início da próxima palavra
     W ..... pula para próxima palavra (desconsidera hífens)
     E ..... pula para o final da próxima palavra (desconsidera 
             hifens)
     e ..... move o cursor para o final da próxima palavra
     zt .... movo o cursor para o topo da página
     zm .... move o cursor para o meio da página
     zz .... move a página de modo com que o cursor fique no 
             centro
     n ..... move o cursor para a próxima ocorrência da busca
     N ..... move o cursor para a ocorrência anterior da busca
\end{verbatim}

Também é possível efetuar saltos e fazer algo mais ao mesmo tempo, a lista abaixo aponta 
algumas dessas possibilidades. % Na verdade nenhum desses comandos são apenas movimentos. 
% Deixar aqui é interessante, mas não misturar com a lista acima
\begin{verbatim}     
     gv .... repete a última seleção visual e posiciona o cursor
             neste local
     % ..... localiza parênteses correspondente
     o ..... letra `o', alterna extremos de seleção visual
     yG .... copia da linha atual até o final do arquivo
     d$ .... deleta do ponto atual até o final da linha
     gi .... entra em modo de inserção no ponto da última edição
     gf .... abre o arquivo sob o cursor
\end{verbatim}

%\section{Big words}\label{Big words}
\index{movendo-se!em palavras grandes}\index{movendo-se!big words}
Para o Vim ``{\em{palavras-separadas-por-hífen}}'' são consideradas em separado, portanto se você usar,
em modo normal ``\verb+w+'' para avançar entre as palavras ele pulará uma de
cada vez, no entanto se usar ``\verb+W+''
em maiúsculo (como visto) ele pulará a ``a-palavra-inteira'' :)

\begin{verbatim}
     E .... pula para o final de palavras com hifen
     B .... pula palavras com hifen (retrocede)
     W .... pula palavras hifenizadas (começo)
\end{verbatim}

Podemos pular sentenças:
\index{movendo-se!entre sentenças}
\begin{verbatim}
     ) .... pula uma sentença para frente
     ( .... pula uma sentença para trás
     } .... pula um parágrafo para frente
     { .... pula um parágrafo para trás
     y) ... copia uma sentença para frente
     d} ... deleta um parágrafo para frente
\end{verbatim}

% muito deslocado, isso é movimentar no arquivo ? Parece que sim.
Caso tenha uma estrutura como abaixo:

\begin{verbatim}
     def pot(x):
        return x**2
\end{verbatim}

E tiver uma referência qualquer para a função \verb+pot+ e desejar
mover o cursor até sua definição basta posicionar o cursor sobre a palavra
\verb+pot+ e pressionar (em modo normal):

\begin{verbatim}
     gd
\end{verbatim}

Se a variável for global, ou seja, estiver fora do documento
(provavelmente em outro) use:

\begin{verbatim}
     gD
\end{verbatim}

Quando definimos uma variável tipo
\index{variável}

\begin{verbatim}
     var = `teste'
\end{verbatim}

e em algum ponto do documento houver referência a esta variável e se
desejar ver seu conteúdo fazemos

\begin{verbatim}
     [i
\end{verbatim}

Na verdade o atalho acima lhe mostrará o último ponto onde foi feita a
atribuição àquela variável que está sob o cursor, uma mão na roda para os
programadores de plantão!  {\Large {\ding{45}}} Observe a  barra de status do
Vim se o tipo de arquivo está certo, tipo. Para detalhes sobre como
personalizar a barra de status na seção \ref{Função para barra de status}.

\begin{verbatim}
     ft=python
\end{verbatim}

a busca por definições de função só funciona se o tipo de arquivo
estiver correto

\begin{verbatim}
     :set ft=python
\end{verbatim}

{\Large \ding{45}} Um mapeamento interessante que facilita a movimentação até
linahas  que contenham determinada palavra de um modo bem simples, bastando
pressionar \verb|,f| pode ser feito assim:

\begin{verbatim}
map ,f [I:let nr = input("Which one: ")<Bar>exe "normal " . nr ."[\t"<CR>
\end{verbatim}

Uma vez definido o mapeamento acima e pressionando-se o atalho associado, que
neste caso é \verb|,f| o vim exibirá as uma oppção para pular para as ocorrências 
da palavra assim:

\begin{verbatim}
    1:  trecho contendo a palavra
    2:  outro trecho contendo a palavra
    Which one: 
\end{verbatim}


outro detalhe para voltar ao último ponto em que você estava

\begin{verbatim}
     ''
\end{verbatim}

A maioria dos comandos do Vim pode ser precedida por um quantificador:

\begin{verbatim}
     5j ..... desce 5 linhas
     d5j .... deleta as próximas 5 linhas
     k ...... em modo normal sobe uma linha
     5k ..... sobe 5 linhas
     y5k .... copia 5 linhas (para cima)
     w ...... pula uma palavra para frente
     5w ..... pula 5 palavras
     d5w .... deleta 5 palavras
     b ...... retrocede uma palavra
     5b ..... retrocede 5 palavras
     fx ..... posiciona o cursor em ``x''
     dfx .... deleta até o próximo ``x''
     dgg .... deleta da linha atual até o começo do arquivo
     dG ..... deleta até o final do arquivo
     yG ..... copia até o final do arquivo
     yfx .... copia até o próximo ``x''
     y5j .... copia 5 linhas
\end{verbatim}

% isso é movimentar, e movimentar diferente...
\section{Paginando}
\label{Paginando}

Para rolar uma página de cada vez (em modo normal)

\begin{verbatim}
     Ctrl-f
     Ctrl-b
\end{verbatim}

\begin{verbatim}
     :h jumps . ajuda sobre a lista de saltos
     :jumps ... exibe a lista de saltos
     Ctrl-i ... salta para a posição mais recente
     Ctrl-o ... salta para a posição mais antiga
     '0 ....... abre o último arquivo editado
     '1 ....... abre o penúltimo arquivo editado
     gd ....... pula para a definição de uma variável
     } ........ pula para o fim do parágrafo
     10| ...... pula para a coluna 10
     [i ....... pula para definição de variável sob o cursor
\end{verbatim}

Observação: lembre-se

\begin{verbatim}
     ^ .... equivale a Ctrl
     ^I ... equivale a Ctrl-I
\end{verbatim}

É possível abrir vários arquivos tipo \verb|vim *.txt|. Editar
algum arquivo, salvar e ir para o próximo arquivo com o comando à
seguir:

\begin{verbatim}
     :wn
\end{verbatim}

Ou voltar ao arquivo anterior

\begin{verbatim}
     :wp
\end{verbatim}

É possível ainda ``rebobinar'' sua lista de arquivos.

\begin{verbatim}
     :rew[wind]
\end{verbatim}

Ir para o primeiro

\begin{verbatim}
     :fir[st]
\end{verbatim}

Ou para o último

\begin{verbatim}
     :la[st]
\end{verbatim}

% idem, movimentar no arquivo e movimentar de forma avançada.
\section{Usando marcas}
\label{sec:UsandoMarcas}
\vimhelp{mark-motions}

As marcas são um meio eficiente de se pular para um local no arquivo. Para
criar uma,  estando em modo normal faz-se:

\begin{verbatim}
     ma
\end{verbatim}

Onde `{\tt m}' indica a criação de uma marca e `{\tt a}' é o nome da marca.
Para pular para a marca `{\tt a}':

\begin{verbatim}
     `a
\end{verbatim}

Para voltar ao ponto do último salto:

\begin{verbatim}
     ''
\end{verbatim}

Para deletar de até a marca `{\tt a}' (em modo normal):

\begin{verbatim}
     d'a
\end{verbatim}

\subsection{Marcas globais}

Durante a edição de vários arquivos pode-se definir uma marca global com o
comando:

\begin{verbatim}
     mA
\end{verbatim}

Onde `{\tt m}' cria a marca e `{\tt A}' (maiúsculo) define uma marca `{\tt A}'
acessível a qualquer momento com o comando:

\begin{verbatim}
     'A
\end{verbatim}

Isto fará o Vim dar um salto até a marca `{\tt A}' mesmo que esteja em outro arquivo,
mesmo que você tenha acabado de fecha-lo. Para abrir e editar vários arquivos
do Vim fazemos:

\begin{verbatim}
     vim *.txt ......... abre todos os arquivos `txt'
     :bn ............... vai para o próximo da lista
     :bp ............... volta para o arquivo anterior
     :ls ............... lista todos os arquivos abertos
     :wn ............... salva e vai para o próximo
     :wp ............... salva e vai para o prévio
\end{verbatim}

%\section{Usando marcadores}
% \label{Usando marcadores}
% 
% No Vim podemos marcar o ponto em que o cursor está para poder retornar
% rapidamente, certifique-se de estar em modo normal, para tanto pressiona-se \verb+<Esc>+,
% pressiona-se então a tecla ``\verb+m+'' seguida de uma das letras do alfabeto:
% 
% \begin{verbatim}
%      ma ....... cria uma marca `a'
%      `a ....... move o cursor para a marca `a'
% \end{verbatim}
% 
% % Isso abaixo é uma subsection da section usando marcadores.
% \subsection{Marcas globais}
% \label{Marcas globais}
% Marcas globais são marcas que permitem pular de um arquivo a outro.
% Para criar uma marca global use a letra que designa a marca em
% maiúsculo.
% 
% \begin{verbatim}
%      mA ....... cria uma marca global A
% \end{verbatim}

%%%%%%%%%%%%%%%%%%%%%%%%%%%%%%%%%%%%%%%%%%%%%%%%%%%%%%%%%%%%%%%%%%%%%%%%
% vim:enc=utf-8:ts=5:sw=5:et:ff=unix:
%%%%%%%%%%%%%%%%%%%%%%%%%%%%%%%%%%%%%%%%%%%%%%%%%%%%%%%%%%%%%%%%%%%%%%%%

\chapter{Folders}
\label{cha:Folders}
{\em Folders} são como dobras nas quais o Vim esconde partes do texto,
algo assim:

\begin{verbatim}
     +-- 10 linhas ---------------------------
\end{verbatim}

Deste ponto em diante chamaremos os {\em folders} descritos no manual do
Vim como dobras!  Quando tiver que manipular grandes quantidades de
texto tente usar dobras, isto permite uma visualização completa do
texto.  Um modo de entender rapidamente como funcionam as dobras no
Vim seria criando uma ``dobra'' para as próximas 10 (dez) linhas com o
comando abaixo:

\begin{verbatim}
     zf10j
\end{verbatim}

Você pode ainda criar uma seleção visual

\begin{verbatim}
     Shift-v ............ seleção por linha
     j .................. desce linha
     zf ................. cria o folder
     zo ................. abre o folder
\end{verbatim}

\section{Métodos de dobras }
\label{Métodos de dobras }
O Vim tem seis modos {\em fold}, são eles:

\begin{itemize}
\item Sintaxe ({\em syntax})
\item Identação ({\em indent})
\item Marcas ({\em marker})
\item Manual ({\em manual})
\item Diferenças ({\em diff})
\item Expressões Regulares ({\em expr})
\end{itemize}

Para determinar o tipo de dobra faça

\begin{verbatim}
     :set foldmethod=tipo
\end{verbatim}

onde o tipo pode ser um dos tipos listados acima, exemplo:

\begin{verbatim}
     :set foldmethod=marker
\end{verbatim}

Outro modo para determinar o método de dobra seria colocando na última
linha do seu arquivo algo assim:

\begin{verbatim}
     vim:fdm=marker:fdl=0:
\end{verbatim}

Obs: \verb|fdm| significa {\em foldmethod}, e \verb|fdl| significa
{\em foldlevel}. Deve haver um espaço entre a palavra inicial ``vim'' e o
começo da linha este recurso chama-se {\em modeline}, leia mais na seção
``\ref{sec:Modelines} modelines'' na página \pageref{sec:Modelines}.

\section{Manipulando dobras }
\label{Manipulando dobras }

\begin{table}[htb]\begin{center} \begin{tabular}{ll} \hline
     \verb|zo| &  abre a dobra\\
     \verb|zO| &  abre a dobra, recursivamente\\
     \verb|za| &  abre/fecha (alterna) a dobra\\
     \verb|zA| & abre/fecha (alterna) a dobra, recursivamente\\
     \verb|zR| &  abre todas as dobras do arquivo atual\\
     \verb|zc| &  fecha uma dobra\\
     \verb|zC| &  fecha a dobra abaixo do cursor, recursivamente\\
     \verb|zfap| &  cria uma dobra para o parágrafo `ap' atual\\
     \verb|zf/casa| &  cria uma dobra até a palavra casa\\
     \verb|zf'a| &  cria uma dobra até a marca `a'\\
     \verb|zd| &  apaga a dobra (não o seu conteúdo)\\
     \verb|zj| &  move para o início da próxima dobra\\
     \verb|zk| &  move para o final da dobra anterior\\
     \verb|[z| &  move o cursor para início da dobra aberta\\
     \verb|]z| &  move o cursor para o fim da dobra aberta\\
     \verb|zi| &  desabilita ou habilita as dobras\\
     \verb|zm, zr| &  diminui/aumenta nível da dobra `fdl'\\
     \verb|:set fdl=0| &  nível da dobra 0 (foldlevel)\\
     \verb|:set foldcolumn=4| & mostra uma coluna ao lado da numeração\\
\hline \end{tabular}\end{center}\end{table}


Para abrir e fechar as dobras usando a barra de
espaços coloque o trecho abaixo no seu arquivo de configuração do Vim
\verb|.vimrc| - veja o capítulo \ref{cha:Como editar preferências no Vim}, página
\pageref{cha:Como editar preferências no Vim}.

\begin{verbatim}
     nnoremap <space> @=((foldclosed(line(".")) < 0) ? 'zc' : 'zo')<CR>
\end{verbatim}

Para abrir e fechar as dobras utilizando o clique do mouse, basta
acrescentar na configuração do seu \verb|.vimrc|:

\begin{verbatim}
     set foldcolumn=2
\end{verbatim}

o que adiciona uma coluna ao lado da coluna de enumeração das linhas.

\section{Criando dobras usando o modo visual}
\label{Criando folders usando o modo visual}
Para iniciar a seleção visual

\begin{verbatim}
     Esc ........ vai para o modo normal
     shift-v .... inicia seleção visual
     j .......... aumenta a seleção visual (desce)
     zf ......... cria a dobra na seleção ativa
\end{verbatim}

Um modo inusitado de se criar dobras é:

\begin{verbatim}
     Shift-v ..... inicia seleção visual
     /chapter/-2 . extende a seleção até /chapter -2 linhas
     zf .......... cria a dobra
\end{verbatim}

%%%%%%%%%%%%%%%%%%%%%%%%%%%%%%%%%%%%%%%%%%%%%%%%%%%%%%%%%%%%%%%%%%%%%%%%
% vim:enc=utf-8:ts=5:sw=5:et:ff=unix:
%%%%%%%%%%%%%%%%%%%%%%%%%%%%%%%%%%%%%%%%%%%%%%%%%%%%%%%%%%%%%%%%%%%%%%%%

\chapter{Registros}
\label{Registros}

O Vim possui nove tipos de registros, cada tipo tem uma utilidade específica,
por exemplo você pode usar um registro que guarda o último comando digitado,
pode ainda imprimir dentro do texto o nome do próprio arquivo, armazenar
porções distintas de texto (área de transferência múltipla) etc. Vamos aos
detalhes.

\begin{itemize}
   \item O registro sem nome ``''
   \item 10 registros nomeados de ``9''
   \item O registro de pequenas deleções "-
   \item 26 registros nomeados de ``z'' ou de ``Z''
   \item 4 registros somente leitura
   \item O registro de expressões "=
   \item Os registro de seleção e  "*, "+ and "~
   \item O registro ``o''
   \item Registro do último padrão de busca "/
\end{itemize}

\section{O registro sem nome ``''}
\label{O registro sem nome ``''}

Armazena o conteúdo de ações como:

\begin{verbatim}
     d ....... deleção
     s ....... substituição
     c ....... modificação `change'
     x ....... apaga um caractere
     yy ...... copia uma linha inteira
\end{verbatim}

Para acessar o conteúdo deste registro basta usar as letras ``{\tt p}'' ou ``{\tt P}'' que
na verdade são comandos para colar abaixo da linha atual e acima da
linha atual (em modo normal).

\section{Registros nomeados de 0 a 9}
\label{Registros nomeados de 0 a 9}
\vimhelp{registers}

O registro zero armazena o conteúdo da última cópia `{\tt yy}', à partir do
registro 1 vão sendo armazenadas as deleções sucessivas de modo que a
mais recente deleção será armazenada no registro 1 e os registros vão
sendo incrementados em direção ao nono.  Deleção menores que uma linha
não são armazenadas nestes registros, caso em que o Vim usa o registro
de pequenas deleções ou que se tenha especificado algum outro
registro.


\section{Registro de pequenas deleções "-}
\label{Registro de pequenas deleções "-}

Quando se {\em deleta} algo menor que uma linha o Vim armazena os dados
deletados neste registro.

\section{Registros nomeados de ``a até z'' ou ``A até Z''}
\label{Registros nomeados de ``a até z'' ou ``A até Z''}
Pode-se armazenar uma linha em modo normal assim:

\begin{verbatim}
     "ayy
\end{verbatim}

Desse modo o Vim guarda o conteúdo da linha no registro `{\tt a}' caso
queira armazenar mais uma linha no registro `{\tt a}' use este comando:

\begin{verbatim}
     "Add
\end{verbatim}

Neste caso a linha corrente é apagada `{\tt dd}' e  adicionada ao final do
registro ``a''.

\begin{verbatim}
     "ayip .. copia o parágrafo atual para o registro ``a''
     "a ..... registro a
     y ...... yank (copia)
     ip ..... inner paragraph (este parágrafo)
\end{verbatim}

\section{Registros somente leitura ``: . \% \#''}
\label{Registros somente leitura}

\begin{verbatim}
     ": ..... armazena o último comando
     ". ..... armazena uma cópia do último texto inserido
     "% ..... contém o nome do arquivo corrente
     "# ..... contém o nome do arquivo alternativo
\end{verbatim}

Uma forma prática de usar registros em modo de inserção é usando:
\verb|Ctrl-r|


\begin{verbatim}
     Ctrl-r % .... insere o nome do arquivo atual
     Ctrl-r : .... insere o último comando digitado
     Ctrl-r / .... insere a última busca efetuada
     Ctrl-r a .... insere o registro `a'
\end{verbatim}

Em modo de inserção pode-se repetir a última inserção de texto
simplesmente pressionando:

\begin{verbatim}
     Ctrl-a
\end{verbatim}

\section{Registro de expressões "=}
\label{sec:Registro de expressões "=}

\begin{verbatim}
     "=
\end{verbatim}

O registro de expressões permite efetuar cálculos diretamente no editor, usando
o atalho ``{\tt Ctrl-r =}'' {\em no modo de inserção}, o editor mostrará um
sinal de igualdade na barra de status e o usuário digita então uma expressão
matemática como uma multiplicação ``{\tt 6*9}'' e em seguida pressiona {\tt
Enter} para que o editor finalize a operação.  Veja um vídeo demonstrando sua
utilização \href{http://vimeo.com/2967392}{neste link}~\cite{RegistroDeExpressoes}.


Para entender melhor como funciona o registro de expressões tomemos um exemplo.
Para fazer uma sequência como abaixo:

\begin{verbatim}
     linha 1 tem o valor 150,
     linha 2 tem o valor 300,
     linha 3 tem o valor 450,
     ...
\end{verbatim}

Acompanhe os passos para a criação de uma macro permite fazer uma
sequência de quantas linhas forem necessárias com o incremento proposto acima.

\begin{verbatim}
     <Esc>  ......... sai do modo de inserção
     qa ............. inicia a macro
     yy ............. copia a primeira linha
     p .............. cola a linha copiada
     w .............. pula para o número `1'
     <Ctrl-a> ....... incrementa o número (agora 2)
     4w ............. avança 4 palavras até 150
     "ndw ........... apaga o `150' para o registro "n
     a .............. entra em modo de inserção
     Ctrl-r = ....... abre o registro de expressões
     Ctrl-r n + 150   insere dentro do registro de expressões
                      o registro "n
    <Enter>  ........ executa o registro de expressões
    <Esc> ........... sai do modo de inserção
    0 ............... vai para o começo da linha
    q ............... para a gravação da macro
\end{verbatim}

Agora posicione o cursor no começo da linha e pressione ``\verb|10@a|''.


{\Large \ding{45}} Na seção \ref{sub:Mapeamento para Calcular Expressões} página
\pageref{sub:Mapeamento para Calcular Expressões} há mais dicas sobre o uso do
registro de expressões cálculos matemáticos.

\section{Registros de arrastar e mover}
\label{Registros de arrastar e mover}

O registro 
\begin{verbatim}
     "*
\end{verbatim}
 é responsável por armazenar o último texto selecionado (p.e., através do
mouse). Já o registro 
\begin{verbatim}
     "+
\end{verbatim}
é o denominado ``área de transferência'', normalmente utilizado para se
transferir conteúdos entre aplicações---este registro é preenchido, por
exemplo, usando-se a típica combinação {\tt Ctrl-v} encontrada em muitas
aplicações. Finalmente, o registro 
\begin{verbatim}
     "~
\end{verbatim}
armazena o texto colado pela operação mais recente de ``arrastar-e-soltar''
({\em drag-and-drop}). 

\section{Registro buraco negro "\_}
\label{Registro buraco negro}

Use este registro quando não quiser alterar os demais registros, por exemplo:
se você deletar a linha atual,

\begin{verbatim}
     dd
\end{verbatim}

Esta ação irá colocar a linha atual no registro numerado 1, caso não
queira alterar o conteúdo do registro 1 apague para o buraco negro
assim:

\begin{verbatim}
     "_dd
\end{verbatim}

\section{Registros de buscas ``/''}
\label{Registros de buscas ``/''}

Se desejar inserir em uma substituição uma busca prévia, você poderia
fazer assim em modo de comandos:

\begin{verbatim}
     :%s,<Ctrl-r>/,novo-texto,g
\end{verbatim}

Observação: veja que estou trocando o delimitador da busca para deixar
claro o uso do registro de buscas ``/''. {\Large {\ding{45}}} Pode-se usar
um registro nomeado de `{\tt a-z}' assim:

\begin{verbatim}
    let @a="new"
    :%s/old/\=@a/g ...... substitui 'old' por new
    \=@a ................ faz referência ao registro `a'
\end{verbatim}

\section{Manipulando registros}
\label{Manipulando registros}

\begin{verbatim}
     :let @a=@_   ... limpa o registro a
     :let @a=``'' ... limpa o registro a
     :let @a=@"   ... salva registro sem nome *N*
     :let @*=@a   ... copia o registro para o buffer de colagem
     :let @*=@:   ... copia o ultimo comando para o buffer de
                      colagem
     :let @*=@/   ... copia a última busca para o buffer de
                      colagem
     :let @*=@%   ... copia o nome do arquivo para o buffer de
                      colagem
     :reg         ... mostra o conteúdo de todos os registros
\end{verbatim}

Em modo de inserção

\begin{verbatim}
     <C-R>-   ....... Insere o registro de pequenas deleções
     <C-R>[0-9a-z] .. Insere registros 0-9 e a-z
     <C-R>%        .. Insere o nome do arquivo
     <C-R>=somevar .. Insere o conteúdo de uma variável
     <C-R><C-A> ..... Insere `Big-Words' veja seção 2.1 
\end{verbatim}


Um exemplo: pré-carregando o nome do arquivo no registro \verb+n+.

coloque em seu \verb+~/.vimrc+

\begin{verbatim}
     let @n=@%
\end{verbatim}

Como foi atribuído ao registro \verb+n+ o conteúdo de @\%, ou seja, o nome
do arquivo, você pode fazer algo assim em modo de inserção:

\begin{verbatim}
     Ctrl-r n
\end{verbatim}

E o nome do arquivo será inserido

\section{Listando os registros atuais}
\label{Listando os registros atuais}
Digitando o comando

\begin{verbatim}
     :reg
\end{verbatim}

ou ainda

\begin{verbatim}
     :ls
\end{verbatim}

O Vim mostrará os registros numerados e nomeados atualmente em uso

\section{Listando arquivos abertos}
\label{Listando arquivos abertos}
Suponha que você abriu vários arquivos {\tt txt} assim:

\begin{verbatim}
     vim *.txt
\end{verbatim}

Para listar os arquivos aberto faça:

\begin{verbatim}
     :buffers
\end{verbatim}

Usando o comando acima o Vim exibirá a lista de todos os arquivos
abertos, após exibir a lista você pode escolher um dos arquivos da
lista, algo como:

\begin{verbatim}
     :buf 3
\end{verbatim}

Para editar arquivos em sequência faça as alterações no arquivo atual
e acesso o próximo assim:

\begin{verbatim}
     :wn
\end{verbatim}

O comando acima diz  $\rightarrow$ `{\tt w gravar}'  $\rightarrow$ `{\tt  n próximo}'

\section{Dividindo a janela com o próximo arquivo da lista de {\em buffers}}
\label{Dividindo a janela com o próximo arquivo da lista de buffers}

\begin{verbatim}
     :sn
\end{verbatim}

O comando acima é uma abreviação de {\em split next}, ou seja, dividir e próximo.

\section{Como colocar um pedaço de texto em um registro?}
\label{Como colocar um pedaço de texto em um registro?}

\begin{verbatim}
     <Esc> ...... vai para o modo normal
     "a10j ...... coloca no registro `a' as próximas 10 linhas
                  `10j'
\end{verbatim}

Pode-se fazer:

\begin{verbatim}
     <Esc> ...... para ter certeza que está em modo normal
     "ap ........ registro a `paste', ou seja, cole
\end{verbatim}

Em modo de inserção faz-se:

\begin{verbatim}
     Ctrl-r a
\end{verbatim}

{\Large {\ding{45}}} Há situações em que se tem caracteres não 
``\textit{ascii }'' que são complicados de se colocar em uma busca ou
substituição, nestes casos pode-se usar os seguintes comandos:

\begin{verbatim}
    "ayl ............. copia para o registro `a' o caractere sob
                       o cursor
    :%s/<c-r>a/char .. subsitui o conteúdo do registro `a' por
                       char
\end{verbatim}

Pode-se ainda usar esta técnica para copiar rapidamente comentários 
do ``\texttt{bash}\footnote{Interpretador de comandos do GNU/Linux}'', representados pelo caracteres \texttt{\#},
em {\em modo normal} usando o atalho ``\texttt{0yljP}''.

\begin{verbatim}
    0 ............... posiciona o cursor no início a linha
    yl .............. copia o caractere sob o cursor
    j ............... desce uma linha
    P ............... cola o caractere copiado
\end{verbatim}

\section{Como criar um registro em modo visual?}
\label{Como criar um registro em modo visual?}
Inicie a seleção visual com o atalho

\begin{verbatim}
     Shift-v ..... seleciona linhas inteiras
\end{verbatim}

pressione a letra ``\verb|j|'' até chegar ao ponto desejado, agora faça

\begin{verbatim}
     "ay
\end{verbatim}
pressione ``\verb|v|'' para sair do modo visual.

\section{Como definir um registro no \texttt{vimrc}?}
\label{Como definir um registro no vimrc?}

Se você não sabe ainda como editar preferências no Vim
leia antes o capítulo \ref{cha:Como editar preferências no Vim}.


Você pode criar uma variável no {\tt vimrc} assim:

\begin{verbatim}
     let var="foo" ...... define foo para var
     echo var ........... mostra o valor de var
\end{verbatim}

Pode também dizer ao Vim algo como...

\begin{verbatim}
     :let @d=strftime("c")<Enter>
\end{verbatim}

Neste caso estou dizendo a ele que guarde na variável `{\tt d}' at d,
o valor da data do sistema `{\tt strftime("c")}' ou então cole isto no
{\tt vimrc}:

\begin{verbatim}
     let @d=strftime("c")<cr>
\end{verbatim}

A diferença entre digitar diretamente um comando e adicioná-lo ao
{\tt vimrc} é que uma vez no {\tt vimrc} o registro em questão estará sempre
disponível, observe também as sutis diferenças, um {\tt Enter} inserido
manualmente é apenas uma indicação de uma ação que você fará
pressionando a tecla especificada, já o comando mapeado vira
``\verb|<cr>|'', veja ainda que no {\tt vimrc} os dois pontos ``\verb|:|''
somem.

Pode mapear tudo isto

\begin{verbatim}
     let @d=strftime("c")<cr>
     imap ,d <cr-r>d
     nmap ,d "dp
\end{verbatim}

As atribuições acima correspondem a:

\begin{enumerate}
 \item  Guarda a data na variável `{\tt d}'
 \item  Mapeamento para o modo de inserção ``{\tt imap}'' digite {\tt ,d}
 \item  Mapeamento para o modo normal ``{\tt nmap}'' digite {\tt ,d}
\end{enumerate}

E digitar {\tt ,d} normalmente

Desmistificando o {\tt strftime}
\begin{verbatim}
     " d=dia m=mes Y=ano H=hora M=minuto c=data-completa
     :h strftime ........ ajuda completa sobre o comando
\end{verbatim}

e inserir em modo normal assim:

\begin{verbatim}
     "dp
\end{verbatim}

ou usar em modo de inserção assim:

\begin{verbatim}
     Ctrl-r d
\end{verbatim}

\section{Como selecionar blocos verticais de texto?}
\label{Como selecionar blocos verticais de texto?}

\begin{verbatim}
     Ctrl-v
\end{verbatim}

agora use as letras {\tt h,l,k,j} como setas de direção até finalizar
podendo guardar a seleção em um registro que vai de `{\tt a}' a `{\tt z}' exemplo:

\begin{verbatim}
     "ay
\end{verbatim}

Em modo normal você pode fazer assim para guardar um parágrafo inteiro em um registro

\begin{verbatim}
     "ayip
\end{verbatim}

O comando acima quer dizer

\begin{verbatim}
     para o registro `a' ......  "a
     copie ......................  `y'
     o parágrafo atual .......... `inner paragraph'
\end{verbatim}

\section{Referências}
\label{Referências}

\begin{itemize}
   \item \url{http://rayninfo.co.uk/vimtips.html}
   \item \url{http://aprendolatex.wordpress.com}
   \item \url{http://pt.wikibooks.org/wiki/Latex}
\end{itemize}

%%%%%%%%%%%%%%%%%%%%%%%%%%%%%%%%%%%%%%%%%%%%%%%%%%%%%%%%%%%%%%%%%%%%%%%%
% vim:enc=utf-8:ts=5:sw=5:et:ff=unix:
%%%%%%%%%%%%%%%%%%%%%%%%%%%%%%%%%%%%%%%%%%%%%%%%%%%%%%%%%%%%%%%%%%%%%%%%

\chapter{Buscas e Substituições}\label{cha:Buscas e substituições}

Para fazer uma busca, certifique-se de que está em modo normal,
pressione ``/'' e digite a expressão a ser procurada. \\


Para encontrar a primeira ocorrência de ``foo'' no texto:

\begin{verbatim}
     /foo
\end{verbatim}

Para repetir a busca basta pressionar a tecla ``\verb+n+'' e para
repetir a busca em sentido oposto ``\verb+N+''.

\begin{verbatim}
     /teste/+3
\end{verbatim}

Posiciona o cursor três linhas após a ocorrência da palavra ``teste'' \\

\begin{verbatim}
     /\<casa\>
\end{verbatim}

A busca acima localiza `{\tt casa}' mas não `{\tt casamento}'. Em expressões
regulares, \verb|\<| e \verb|\>| são representadas por \verb|\b|, que representa, por sua vez, borda
de palavras. Ou seja, `{\tt casa,}`, `{\tt casa!}` seriam localizado, visto que sinais
de pontuação não fazem parte da palavra.


\section{Usando ``Expressões Regulares'' em buscas}
\label{Usando ``Expressões Regulares'' em buscas}
\index{expressões regulares!buscas}
\vimhelp{regex, pattern}

\begin{verbatim}
     / ........... inicia uma busca (modo normal)
     \%x69 ....... código da letra `i'
     /\%x69 ...... localiza a letra `i' - hexadecimal 069
     \d .......... localiza números
     [3-8] ....... localiza números de 3 até 8
     ^ ........... começo de linha
     $ ........... final de linha
     \+ .......... um ou mais
     /^\d\+$ ..... localiza somente dígitos
     /\r$ ........ localiza linhas terminadas com ^M
     /^\s*$ ...... localiza linhas vazias ou contendo apenas espaços
     /^\t\+ ...... localiza linhas que iniciam com tabs
     \s .......... localiza espaços
     /\s\+$ ...... localiza espaços no final da linha
\end{verbatim}

\subsection{Evitando escapes ao usar Expressões regulares}

O Vim possui um modo chamado ``{\em very magic}'' para uso em expressões
regulares que evita o uso excessivo de escapes, alternativas etc. Usando apenas
uma opção: veja ``\verb+:h /\v+''.

Em um trecho com dígitos + texto + dígitos
no qual se deseja manter só as letras.

\begin{verbatim}
     12345aaa678
     12345bbb678
     12345aac678
\end{verbatim}

Sem a opção ``{\em very magic}'' faríamos:

\begin{verbatim}
     :%s/\d\{5\}\(\D\+\)\d\{3\}/\1/
\end{verbatim}

Já com a opção ``{\em very magic}'' ``\verb+\v+'' usa-se bem menos escapes:

\begin{verbatim}
     :%s/\v\d{5}(\D+)\d{3}/\1/

     " explicação do comando acima
     : ......... comando
     % ......... em todo arquivo
     s ......... substitua
     / ......... inicia padrão de busca
     \v ........ use very magic mode
     \d ........ dígitos
     {5} ....... 5 vezes 
     ( ........ inicia um grupo
     \D ........ seguido de não dígitos
     )  ........ fecha um grupo     
     + ......... uma ou mais vezes
     \d ........ novamente dígitos
     {3} ....... três vezes 
     / ......... inicio da substituição
     \1 ........ referencia o grupo 1
\end{verbatim}

Analisando o exemplo anterior, a linha de raciocínio foi a de ``manter o texto entre os
dígitos'', o que pode ser traduzido, em uma outra forma de raciocínio, como ``remover os
dígitos''.

\begin{verbatim}
     :%s/\d//g

     " explicação do comando acima
     % ......... em todo arquivo
     s ......... substitua
     / ......... inicia padrão de busca
     \d ........ ao encontrar um dígito
     / ......... subtituir por
     vazio ..... exato, substituir por vazio
     /g ........ a expressão se torna gulosa
\end{verbatim}

Por guloso - \verb|/g| - se entende que ele pode e deve tentar achar mais de uma ocorrência
do padrão de busca na mesma linha. Caso não seja gulosa, a expressão irá apenas casar com a
primeira ocorrência em cada linha.

\subsubsection{Classes {\em POSIX} para uso em Expressões Regulares}

Ao fazermos substituições em textos poderemos nos deparar com
erros, pois [a-z] não inclui caracteres acentuados, as classes
{\em POSIX} são a solução para este problema, pois adequam o
sistema ao idioma local, esta é a mágica implementada por estas classes.

\begin{verbatim}
     [:lower:] ...... letras minúsculas incluindo acentos
     [:upper:] ...... letras maiúsculas incluindo acentos
     [:punct:] ...... ponto, virgula, colchete, etc
\end{verbatim}

Para usar estas classes fazemos:

\begin{verbatim}
     :%s/[[:lower:]]/\U&/g

     Explicando o comando acima:
     : ....... modo de comando
     % ....... em todo o arquivo atual
     s ....... substitua
     / ....... inicia o padrão a ser buscado
     [ ....... inicia um grupo
     [: ...... inicia uma classe POSIX
     lower ... letras minúsculas
     :] ...... termina a classe POSIX
     ] ....... termina o grupo
     / ....... inicia substituição
     \U ...... para maiúsculo
     & ....... correponde ao que foi buscado
\end{verbatim}

Nem todas as classes {\em POSIX} conseguem pegar caracteres
acentuados, portanto deve-se habilitar o destaque colorido para
buscas usando:

\begin{verbatim}
     :set hlsearch .... destaque colorido para buscas
     :set incsearch ... busca incremental
\end{verbatim}

Dessa forma podemos testar nossas buscas antes de fazer
uma substituição.


Para aprender mais sobre Expressões Regulares leia:

\begin{itemize}
  \item \href{http://guia-er.sourceforge.net}{Guia sobre Espressões Regulares}~\cite{JargasRegex}
  \item {\tt :help regex}
  \item {\tt :help pattern}
\end{itemize}

{\Large \ding{45}} Uma forma rápida para encontrar a próxima ocorrência de uma
palavra sob o cursor é teclar `\verb|*|'. Para encontrar uma ocorrência
anterior da palavra sob o cursor, existe o \verb|#| (em ambos os casos o cursor
deve estar posicionado sobre a palavra que deseja procurar). As duas opções
citadas localizam apenas se a palavra corresponder totalmente ao padrão sob o
cursor, pode-se bucar por trechos de palavras que façam parte de palavras
maiores usando-se `\verb+g*+'.  Pode-se ainda exibir ``dentro do contexto''
todas as ocorrências de uma palavra sob o cursor usando-se o seguinte atalho em
modo normal:

\begin{verbatim}
     [ Shift-i
\end{verbatim}

\section{Destacando padrões}
\label{sec:Destacando padrões}
\vimhelp{\%>}

Você pode destacar linhas com mais de 30 caracteres assim:

\begin{verbatim}
     :match ErrorMsg /\%>30v/ . destaca linhas maiores que 30 caracteres
     :match none .............. remove o destaque
\end{verbatim}

\section{Inserindo linha antes e depois}

Suponha que se queira um comando, considere ``\verb|,t|'', que faça com que a
linha indentada corrente passe a ter uma linha em branco antes e depois; isto
pode ser obtido pelo seguinte mapeamento:

\begin{verbatim}
     :map ,t <Esc>:.s/^\(\s\+\)\(.*\)/\r\1\2\r/g<cr>
\end{verbatim}

Explicando:

\begin{verbatim}
     : ................ entra no modo de comando
     map ,t ........... mapeia ,t para a função desejada
     <Esc> ............ ao executar sai do modo de inserção
     s/isto/aquilo/g .. substitui isto por aquilo
     : ................ inicia o modo de comando
     . ................ na linha corrente
     s ................ substitua
     ^ ................ começo de linha
     \s\+ ............. um espaço ou mais (barras são escapes)
     .* ............... qualquer coisa depois
     \(grupo\) ........ agrupo para referenciar com \1
     \1 ............... repete na substituição o grupo 1
     \r ............... insere uma quebra de linha
     g ................ em todas as ocorrências da linha
     <cr> ............. Enter
\end{verbatim}

\section{Obtendo informações do arquivo}

\begin{verbatim}
     ga ............. mostra o código do caractere em decimal hexa e octal
     ^g ............. mostra o caminho e o nome do arquivo
     g^g ............ mostra estatísticas detalhadas do arquivo
\end{verbatim}

Obs: O código do caractere pode ser usado para substituições,
especialmente em se tratando de caracteres de controle como tabulações
``\verb|^I|'' ou final de linha DOS/Windows ``\verb|\%x0d|''. Você pode apagar os
caracteres de final de linha Dos/Windows usando uma simples
substituição, veja mais adiante:

\begin{verbatim}
     :%s/\%x0d//g
\end{verbatim}

Uma forma mais prática de substituir o terminador de linha DOS para o
terminador de linha Unix:

\begin{verbatim}
    :set ff=unix
    :w
\end{verbatim}

Na seção \ref{cha:Como editar preferências no Vim}
página~\pageref{cha:Como editar preferências no Vim}
há um código para a barra de
status que faz com que a mesma exiba o código do caractere sob o cursor na
seção \ref{Função para barra de status}. {\Large {\ding{45}}} O caractere de final de linha
do Windows/DOS pode ser inserido com a seguinte combinação de teclas:

\begin{verbatim}
     i ............ entra em modo de inserção
     <INSERT> ..... entra em modo de inserção
     Ctrl-v Ctrl-m  insere o simbolo ^M (terminador de linha DOS)
\end{verbatim}

\section{Trabalhando com registradores}
\label{Trabalhando com registradores}

Pode-se guardar trechos do que foi copiado ou apagado para
registros distintos (área de transferência múltipla).
Os registros são indicados por aspas seguido por uma letra.
Exemplos: {\tt "a}, {\tt "b}, {\tt "c}, etc.


Como copiar o texto para um registrador? É simples: basta especificar
o nome do registrador antes:

\begin{verbatim}
     "add ... apaga linha para o registrador a
     "bdd ... apaga linha para o registrador b
     "ap .... cola" o conteúdo do registrador a
     "ab .... cola" o conteúdo do registrador b
     "x3dd .. apaga 3 linhas para o registrador ``x''
     "ayy  .. copia linha para o registrador `a'
     "a3yy .. copia 3 linhas para o registrador `a'
     "ayw  .. copia uma palavra para o registrador `a'
     "a3yw .. copia 3 palavras para o registrador `a'
\end{verbatim}

No ``modo de inserção'', como visto anteriormente, pode-se usar um atalho
para colar rapidamente o conteúdo de um registrador.

\begin{verbatim}
     Ctrl-r (registro)
\end{verbatim}

Para colar o conteúdo do registrador `{\tt a}'

\begin{verbatim}
     Ctrl-r a
\end{verbatim}

Para copiar a linha atual para a área de transferência

\begin{verbatim}
     "+yy
\end{verbatim}

Para colar da área de transferência

\begin{verbatim}
     "+p
\end{verbatim}

Para copiar o arquivo atual para a área de transferência ``{\em clipboard}'':

\begin{verbatim}
     :%y+
\end{verbatim}

\section{Edições complexas }
\label{Edições complexas }

Trocando palavras de lugar: Posiciona-se o cursor no espaço antes da 1ª palavra e digita-se:

\begin{verbatim}
     deep
\end{verbatim}

Trocando letras de lugar:

\begin{verbatim}
     xp .... com a letra seguinte
     xh[p .. com a letra anterior
\end{verbatim}

Trocando linhas de lugar:

\begin{verbatim}
     ddp ... com a linha de baixo
     ddkP .. com a linha de cima
\end{verbatim}

Tornando todo o texto maiúsculo

\begin{verbatim}
     gggUG
     ggVGU
\end{verbatim}

\section{Indentando }

\begin{verbatim}
     >> ..... Indenta a linha atual
     ^T ..... Indenta a linha atual em modo de inserção
     ^D ..... Remove indentação em modo de inserção
     >ip .... indenta o parágrafo atual
\end{verbatim}

\section{Corrigindo a indentação de códigos}
\label{Corrigindo a indentação de códigos}
Selecione o bloco de código, por exemplo

\begin{verbatim}
     vip  ..... visual ``inner paragraph'' (selecione este parágrafo)
     =  ....... corrige a indentação do bloco de texto selecionado
     ggVG= .... corrige a identação do arquivo inteiro
\end{verbatim}

\section{Usando o File Explorer}
\label{Usando o file explorer}
O Vim navega na árvore de diretórios com o comando

\begin{verbatim}
     vim .
\end{verbatim}

Use o `{\tt j}' para descer e o `{\tt k}' para subir ou {\tt Enter} para editar o
arquivo selecionado. {\Large \ding{45}}  Pressionando {\tt F1} ao abrir o
FileExplorer do Vim, você encontra dicas adicionais sobre este modo de
operação do Vim.

\section{Selecionando ou deletando conteúdo de tags HTML}
\label{Selecionando ou deletando conteúdo de tags html}

\begin{verbatim}
     <tag> conteúdo da tag </tag>
     basta usar (em modo normal) as teclas
     vit ............... visual `inner tag | esta tag'
\end{verbatim}

Este recurso também funciona com parênteses

\begin{verbatim}
     vi( ..... visual select
     vi" ..... visual select
     di( ..... delete inner (, ou seja, seu conteúdo
\end{verbatim}


\section{Substituições }
\label{Substituições }

Para fazer uma busca, certifique-se de que está em modo normal, em
seguida digite use o comando `{\tt s}', conforme será explicado.

Para substituir ``foo'' por ``bar'' na linha atual:

\begin{verbatim}
     :s/foo/bar
\end{verbatim}

Para substituir ``foo'' por ``bar'' da primeira à décima linha do arquivo:

\begin{verbatim}
     :1,10 s/foo/bar
\end{verbatim}

Para substituir ``foo'' por ``bar'' da primeira à última linha do arquivo:

\begin{verbatim}
     :1,$ s/foo/bar
\end{verbatim}

Ou simplesmente:

\begin{verbatim}
     :% s/foo/bar
\end{verbatim}

\begin{verbatim}
     $ ... significa para o Vim final do arquivo
     % ... representa o arquivo atual
\end{verbatim}

O comando `{\tt s}' possui muitas opções que modificam seu comportamento.

\section{Exemplos }
\label{Exemplos }

Busca usando alternativas:

\begin{verbatim}
     /end\(if\|while\|for\)
\end{verbatim}

Buscará `{\tt if}', `{\tt while}' e `{\tt for}'.  Observe que é necessário `escapar' os
caracteres \verb|\(|, \verb|\|| e \verb|\)|, caso contrário eles serão
interpretados como caracteres comuns.

Quebra de linha

\begin{verbatim}
     /quebra\nde linha
\end{verbatim}

Ignorando maiúsculas e minúsculas

\begin{verbatim}
     /\cpalavra
\end{verbatim}

Usando \verb|\c| o Vim encontrará ``{\em{palavra}}'', ``{\em{Palavraa}}'' ou
até mesmo ``{\em{PALAVRA}}''. Uma dica é colocar no seu arquivo de configuração
``vimrc'' veja o capítulo \ref{cha:Como editar preferências no Vim} na
página~\pageref{cha:Como editar preferências no Vim}.

\begin{verbatim}
     set ignorecase .. ignora maiúsculas e minúsculas na bucsca
     set smartcase ... se busca contiver maiúsculas ele passa a considerá-las
     set hlsearch .... mostra o que está sendo buscado em cores
     set incsearch ... ativa a busca incremental
\end{verbatim}

se você não sabe ainda como colocar estas preferências no arquivo de configuração pode
ativa-las em modo de comando precedendo-as com dois pontos, assim:

\begin{verbatim}
     :set ignorecase<Enter>
\end{verbatim}

Substituições com confirmação:

\begin{verbatim}
     :%s/word/palavra/c ..... o `c' no final habilita a confirmação
\end{verbatim}


Procurando palavras repetidas

\begin{verbatim}
     /\<\(\w*\) \1\>
\end{verbatim}

Multilinha

\begin{verbatim}
     /Hello\_s\+World
\end{verbatim}

Buscará `World', separado por qualquer número de espaços,
incluindo quebras de linha. Buscará as três sequências:

\begin{verbatim}
     Hello World

     Hello    World

     Hello
     World
\end{verbatim}

Buscar linhas de até 30 caracteres de comprimento

\begin{verbatim}
     /^.\{,30\}$
\end{verbatim}

\begin{verbatim}
     ^  ..... representa começo de linha
     .  ..... representa qualquer caractere
\end{verbatim}


\begin{verbatim}
     :%s/<[^>]*>//g ... apaga tags HTML/XML
     :%g/^$/d ......... apaga linhas vazias
     :%s/^[\ \t]*\n//g  apagarlinhas vazias
\end{verbatim}


Remover duas ou mais linhas vazias entre parágrafos diminuindo para
uma só linha vazia.

\begin{verbatim}
     :%s/\(^\n\{2,}\)/\r/g
\end{verbatim}

Você pode criar um mapeamento e colocar no seu {\tt ~/.vimrc}

\begin{verbatim}
     map ,s <Esc>:%s/\(^\n\{2,}\)/\r/g<cr>
\end{verbatim}

No exemplo acima, `{\tt ,s}' é um mapeamento para reduzir linhas em branco
sucessivas para uma só  \\


Remove não dígitos (não pega números)

\begin{verbatim}
     :%s/^\D.*//g
\end{verbatim}

Remove final de linha DOS/Windows \verb|^M| que tem código hexadecimal igual a
`{\tt 0d}'

\begin{verbatim}
     :%s/\%x0d//g
\end{verbatim}

Troca palavras de lugar usando expressões regulares:

\begin{verbatim}
     :%s/\(.\+\)\s\(.\+\)/\2 \1/
\end{verbatim}

Modificando todas as tags HTML para minúsculo:

\begin{verbatim}
     :%s/<\([^>]*\)>/<\L\1>/g
\end{verbatim}

Move linhas 10 a 12 para além da linha 30:

\begin{verbatim}
     :10,12m30
\end{verbatim}

\section{O comando global ``g''}\label{sec:O comando global ``g''}

Buscando um padrão e gravando em outro arquivo:

\begin{verbatim}
     :'a,'b g/^Error/ . w >> errors.txt
\end{verbatim}

Apenas imprimir linhas que contém determinada palavra, isto é útil
quando você quer ter uma visão sobre um determina aspecto
do seu arquivo vejamos:

\begin{verbatim}
     :set nu ..... habilita numeração
     :g/Error/p .. apenas mostra as linhas correspondentes
\end{verbatim}

{\Large \ding{45}} Para mostrar o as linhas correspondentes a um padrão, mesmo que 
a numeração de linha não esteja habilitada use ``{\tt :g/padrão/\#}''.

numerar linhas:

\begin{verbatim}
     :let i=1 | g/^/s//\=i."\t"/ | let i=i+1
\end{verbatim}

Para copiar linhas começadas com {\em Error} para o final do arquivo faça:

\begin{verbatim}
     :g/^Error/ copy $
\end{verbatim}

Obs: O comando `{\tt copy}' pode ser abreviado `{\tt co}' ou ainda pode-se usar `{\tt t}'
para mais detalhes:

\begin{verbatim}
     :h co
\end{verbatim}

Como adicionar um padrão copiado com `{\tt yy}' após um determinado padrão?

\begin{verbatim}
    :g/padrao/+put!
\end{verbatim}

Entre as linhas que contiverem `{\tt fred}' e `{\tt joe}' substitua:

\begin{verbatim}
     :g/fred/,/joe/s/isto/aquilo/gic
\end{verbatim}

As opções `gic' correspondem a {\em global}, {\em ignore case} e {\em
confirm}, podendo ser omitidas deixando só o {\em global}. \\


Pegar caracteres numéricos e jogar no final do arquivo:

\begin{verbatim}
     :g/^\d\+.*/m $
\end{verbatim}

Inverter a ordem das linhas do arquivo:

\begin{verbatim}
     :g/^/m0
\end{verbatim}

Apagar as linhas que contém {\tt Line commented}:

\begin{verbatim}
     :g/Line commented/d
\end{verbatim}

Apagar todas as linhas comentadas

\begin{verbatim}
      :g/^\s*#/d
\end{verbatim}

Copiar determinado padrão para um registro:

\begin{verbatim}
     :g/pattern/ normal "Ayy
\end{verbatim}

Copiar linhas que contém um padrão e a linha subsequente para o final:

\begin{verbatim}
     :g/padrão/;+1 copy $
\end{verbatim}

Deletar linhas que não contenham um padrão:

\begin{verbatim}
     :v/dicas/d  ..... deleta linhas que não contenham `dicas'
\end{verbatim}

Incrementar números no começo da linha:

\begin{verbatim}
     :.,20g/^\d/exe "normal! \<c-a>"
\end{verbatim}

Sublinhar linhas começadas com {\em Chapter}:

\begin{verbatim}
     :g/^Chapter/t.|s/./-/g

     : ........ comando
     g ........ global
     / ........ inicio de um padrão
     ^ ........ começo de linha
     Chapter .. palavra literal
     / ........ fim do parão
     t ........ copia
     . ........ linha atual
     s ........ substitua
     / ........ inicio de um padrão
     . ........ qualquer caractere
     / ........ início da substituição
     - ........ por traço
     / ........ fim da substituição
     g ........ em todas as ocorrências
\end{verbatim}

\section{Dicas }
Para colocar a última busca em uma substituição faça:

\begin{verbatim}
     :%s/Ctrl-r//novo/g
\end{verbatim}

A dupla barra corresponde ao ultimo padrão procurado, e portanto o
comando abaixo fará a substituição da ultima busca por casinha:

\begin{verbatim}
     :%s//casinha/g
\end{verbatim}

\section{Filtrando arquivos com o vimgrep}
\label{Filtrando arquivos com o vimgrep}

Por vezes sabemos que aquela anotação foi feita, mas no momento esquecemos em qual
arquivo está, no exemplo abaixo procuramos a palavra dicas à partir da nossa pasta pessoal
pela palavra `dicas' em todos os arquivos com extensão `{\tt txt}'.

\begin{verbatim}
     ~/ ............ equivale a /home/user
     :lvimgrep /dicas/gj ~/**/*.txt | ls
     :h lvim ....... ajuda sobre o comando
\end{verbatim}



\section{Copiar a partir de um ponto}

\begin{verbatim}
     :19;+3 co $
\end{verbatim}

O Vim sempre necessita de um intervalo (inicial e final) mas se usar-mos
`;' ele considera a primeira linha como segundo ponto do
intervalo, e no caso acima estamos dizendo (nas entrelinhas) linhas
19 e 19+3     \\


De forma análoga pode-se usar como referência um padrão qualquer:

\begin{verbatim}
     :/palavra/;+10 m 0
\end{verbatim}

O comando acima diz: à partir da linha que contém ``palavra'' incluindo as 10 próximas linhas
mova `{\tt m}' para a primeira linha `0', ou seja, antes da linha 1.

\section{Dicas das lista vi-br}

Fonte: \href{http://groups.yahoo.com/group/vi-br/message/853}{Grupo vi-br do
yahoo}~\cite{ViBr01}

Problema:
Essa deve ser uma pergunta comum.
Suponha o seguinte conteúdo de arquivo:

\begin{verbatim}
     ... // várias linhas
     texto1000texto    // linha i
     texto1000texto    // linha i+1
     texto1000texto    // linha i+2
     texto1000texto    // linha i+3
     texto1000texto    // linha i+4
     ... // várias linhas
\end{verbatim}

Gostaria de um comando que mudasse para

\begin{verbatim}
     ... // várias linhas
     texto1001texto    // linha i
     texto1002texto    // linha i+1
     texto1003texto    // linha i+2
     texto1004texto    // linha i+3
     texto1005texto    // linha i+4
     ... // várias linhas
\end{verbatim}

Ou seja, somasse 1 a cada um dos números entre os textos
especificando como range as linhas i,i+4

\begin{verbatim}
     :10,20! awk 'BEGIN{i=1}{if (match($0, ``+'')) print ``o''
     (substr($0, RSTART, RLENGTH) + i++) ``o'``}''
\end{verbatim}

Mas muitos sistemas não tem {\tt awk}, e logo a melhor solução mesmo é usar o Vim:

\begin{verbatim}
     :let i=1 | 10,20 g/texto\d\+texto/s/\d\+/\=submatch(0)+i/ | let i=i+1
\end{verbatim}

Observação: 10,20 é o intervalo, ou seja, da linha 10 até a linha 20

\begin{verbatim}
     :help /
     :help :s
     :help pattern
\end{verbatim}



\section{Junção de linhas com Vim}
\label{sec:Junção de linhas com Vim}
\index{junção de linhas}

fonte: \href{http://www.dicas-l.com.br/dicas-l/20081228.php}{dicas-l da
unicamp}~\cite{dicas-lJuncaoDeLinhas} \\ Colaboração: Rubens Queiroz de Almeida

Recentemente precisei combinar, em um arquivo, duas linhas
consecutivas. O arquivo original continha linhas como:

\begin{verbatim}
     Matrícula: 123456
     Senha: yatVind7kned
     Matrícula: 123456
     Senha: invanBabnit3
\end{verbatim}

E assim por diante. Eu precisava converter este arquivo para algo como:

\begin{verbatim}
     Matrícula: 123456 - Senha: yatVind7kned
     Matrícula: 123456 - Senha: invanBabnit3
\end{verbatim}

Para isto, basta executar o comando:

\begin{verbatim}
     :g/^Matrícula/s/\n/ - /
\end{verbatim}

Explicando:

\begin{verbatim}
     s/isto/aquilo/g .. substitui isto por aquilo
     g ................ comando global
     /................. inicia padrão de busca
     ^ ................ indica começo de linha
     Matrícula ........ palavra a ser buscada
     s ................ inicia substituição
     /\n/ - / ......... troca quebra de linha (\n), por -
\end{verbatim}

\section{Buscando em um intervalo de linhas} % (fold)
\label{sec:Buscando em um intervalo de linh}
\index{buscas!intervalo}
Para buscar entre as linhas 10 e 50 por um padrão qualquer fazemos:
\begin{verbatim}
    /padrao\%>10l\$<50l
\end{verbatim}

Esta e outras boas dicas podem ser lidas no site
\href{http://vimdoc.sourceforge.net/htmldoc/vimfaq.html}{vim-faq}~\cite{VimFaq}.
% section Buscando em um intervalo de linh (end)


%%%%%%%%%%%%%%%%%%%%%%%%%%%%%%%%%%%%%%%%%%%%%%%%%%%%%%%%%%%%%%%%%%%%%%%%
% vim:enc=utf-8:ts=5:sw=5:et
%%%%%%%%%%%%%%%%%%%%%%%%%%%%%%%%%%%%%%%%%%%%%%%%%%%%%%%%%%%%%%%%%%%%%%%%

\chapter{Trabalhando com Janelas}\label{cha:Trabalhando com janelas}

O Vim trabalha com o conceito de múltiplos ``buffers''. Cada
``buffer'' é um arquivo carregado para edição. Um ``buffer'' pode
estar visível ou não, e é possível dividir a tela em janelas, de forma
a visualizar mais de um ``buffer'' simultaneamente.

\section{Dividindo a janela }
Observação: \verb+Ctrl = ^+

\begin{verbatim}
     Ctrl-w-s   Divide a janela atual em duas (:split)
     Ctrl-w-o   Faz a janela atual ser a única (:only)
     Ctrl-w-n   Abre nova janela (:new)
     Ctrl-w-q   Fecha a janela atual (:quit)
\end{verbatim}

Caso tenha duas janelas e use o atalho acima \verb|^wo|, é recomendado salvar
tudo ao fechar, pois apesar de a outra janela estar fechada o arquivo
ainda estará carregado, portanto faça:

\begin{verbatim}
     :wall ... salva todos `write all'
     :qall ... fecha todos `quite all'
\end{verbatim}

\section{Abrindo e fechando janelas }

\begin{verbatim}
     Ctrl-w-n   Abre uma nova janela acima
     Ctrl-w-q   Fecha a janela atual
     Ctrl-w-c   Fecha a janela atual (:close)
\end{verbatim}

\section{Manipulando janelas }

\begin{verbatim}
     Ctrl-w-w ... Alterna entre janelas
     Ctrl-w-j ... desce uma janela `j'
     Ctrl-w-k ... sobe  uma janela `k'
     Ctrl-w-r ... Rotaciona janelas na tela
     Ctrl-w-+ ... Aumenta o espaço da janela atual
     Ctrl-w-- ... Diminui o espaço da janela atual
\end{verbatim}

\section{File Explorer }
\label{File Explorer }
\vimhelp{buffers windows}
Para abrir o gerenciador de arquivos do Vim use:

\begin{verbatim}
     :Vex ........... abre o file explorer verticalmente
     :Sex ........... abre o file explorer em nova janela
     :e .   ......... abre o file explorer na janela atual
     após abrir chame a ajuda <F1>
\end{verbatim}

Para abrir o arquivo sob o cursor em nova janela coloque a linha abaixo no seu \verb|~/.vimrc|

\begin{verbatim}
     let g:netrw_altv = 1
\end{verbatim}

É possível mapear um atalho ``no caso abaixo F2'' para abrir o File Explorer.

\begin{verbatim}
     map <F2> <Esc>:Vex<cr>
\end{verbatim}


{\Large {\ding{45}}} Ao editar um arquivo no qual há referência a um outro
arquivo, por exemplo: `{\tt /etc/hosts}', pode-se usar o atalho `{\tt Ctrl-w
f}' para abri-lo em nova janela, ou `{\tt gf}' para abri-lo na janela atual.
Mas é importante posicionar o cursor sobre o nome do arquivo.  Veja também
mapeamentos na seção \ref{Mapeamentos} página \pageref{Mapeamentos}.

%%%%%%%%%%%%%%%%%%%%%%%%%%%%%%%%%%%%%%%%%%%%%%%%%%%%%%%%%%%%%%%%%%%%%%%%
% vim:enc=utf-8:ts=5:sw=5:et
%%%%%%%%%%%%%%%%%%%%%%%%%%%%%%%%%%%%%%%%%%%%%%%%%%%%%%%%%%%%%%%%%%%%%%%%

\chapter{Repetição de Comandos}\label{Repetição de comandos}

Para repetir a última edição saia do modo de Inserção e pressione ponto (.):

\begin{verbatim}
     .
\end{verbatim}

Para inserir um texto que deve ser repetido várias vezes:

\begin{verbatim}
     # Posicione o cursor no local desejado;
     # Digite o número de repetições;
     # Entre em modo de inserção;
     # Digite o texto;
     # Saia do modo de inserção (tecle Esc).
\end{verbatim}

Por exemplo, se você quiser inserir oitenta traços numa linha, em vez
de digitar um por um, você pode digitar o comando:

\begin{verbatim}
     80i-<Esc>
\end{verbatim}

Veja, passo a passo, o que aconteceu:

 Antes de entrar em modo de inserção usamos um quantificador

\begin{verbatim}
     `80'
\end{verbatim}

 depois iniciamos o modo de inserção

\begin{verbatim}
     i
\end{verbatim}

depois digitamos o caractere a ser repetido

\begin{verbatim}
     -
\end{verbatim}

e por fim saímos do modo de inserção

\begin{verbatim}
     <Esc>
\end{verbatim}

Se desejássemos digitar 10 linhas com o texto

\begin{verbatim}
     isto é um teste
\end{verbatim}

deveríamos então fazer assim:
   
\begin{verbatim}
     <Esc> .. para ter certeza que ainda estamos no modo normal
     10 ..... quantificador antes
     i ...... entrar no modo de inserção
     isto é um teste <Enter>
     <Esc> .. voltar ao modo normal
\end{verbatim}

\section{Repetindo a digitação de uma linha }

\begin{verbatim}
     modo de inserção
     Ctrl-y .......... repete a linha acima 
     Ctrl-e .......... repetira linha abaixo 
     Ctrl-x Ctrl-l ... repete linhas completas
\end{verbatim}


O atalho {\tt Ctrl-x Ctrl-l} só funcionará para uma 
linha semelhante, experimente digitar:

\begin{verbatim}
     uma linha qualquer com algum conteúdo
     uma linha <Ctrl-x Ctrl-l>
\end{verbatim}

e veja o resultado

\section{Guardando trechos em ``registros''}
\label{sec:Guardando trechos em ``registros''}

Os registradores ``z'' são uma espécie de área de transferência múltipla.

Você deve estar em modo normal e então digitar uma aspa dupla e uma
das 26 letras do alfabeto, em seguida uma ação por exemplo, `y'
(copiar) `d' (apagar). Depois, mova o cursor para a linha
desejada e cole com "rp, onde `r' corresponde ao
registrador para onde o trecho foi copiado.

\begin{verbatim}
     "ayy ... copia a linha atual para o registrador ``a''
     "ap  ... cola o conteúdo do registrador ``a'' abaixo
     "bdd ... apaga a linha atual para o registrador ``b''
\end{verbatim}

\section{Macros: gravando comandos}\label{Macros: gravando comandos}

Imagine que você tem o seguinte trecho de código:

\begin{verbatim}
     stdio.h
     fcntl.h
     unistd.h
     stdlib.h
\end{verbatim}

e quer que ele fique assim:

\begin{verbatim}
     #include "stdio.h"
     #include "fcntl.h"
     #include "unistd.h"
     #include "stdlib.h"
\end{verbatim}

Não podemos simplesmente executar repetidas vezes um comando do Vim, pois
precisamos incluir texto tanto no começo quanto no fim da linha?  É necessário
mais de um comando para isso.  É aí que entram as macros. Podemos gravar até 26
macros, já que elas são guardadas nos registros do Vim, que são identificados
pelas letras do alfabeto. Para começar a gravar uma macro no registro ``a'',
digitamos

\begin{verbatim}
     qa
\end{verbatim}

No modo Normal. Tudo o que for digitado a partir daí será gravado no
registro ``a'' até que terminemos com o comando
\verb|<Esc>q| novamente (no modo Normal). Assim,
podemos solucionar nosso problema:

\begin{verbatim}
     <Esc> ....... para garantir que estamos no modo normal
     qa .......... inicia a gravação da macro `a'
     I ........... entra no modo de inserção no começo da linha
     #include " .. insere #include "
     <Esc> ....... sai do modo de inserção
     A" .......... insere o último caractere
     <Esc> ....... sai do modo de inserção
     j ........... desce uma linha
     <Esc> ....... sai do modo de inserção
     q ........... para a gravação da macro
\end{verbatim}

Agora você só precisa posicionar o cursor na primeira letra de uma linha como esta

\begin{verbatim}
     stdio.h
\end{verbatim}

E executar a macro do registro ``a'' quantas vezes for necessário,
usando o comando \verb|@a|. Para executar quatro vezes, digite:

\begin{verbatim}
     4@a
\end{verbatim}

Este comando executa quatro vezes o conteúdo do registro ``a''.

Caso tenha executado a macro uma vez pode repeti-la com o comando

\begin{verbatim}
     @@
\end{verbatim}

\section{Repetindo substituições }
Se você fizer uma substituição em um intervalo como abaixo

\begin{verbatim}
     :5,32s/isto/aquilo/g
\end{verbatim}

Pode repetir esta substituição em qualquer linha que estiver apenas usando este símbolo

\begin{verbatim}
     &
\end{verbatim}

O Vim substituirá na linha corrente ``isto'' por ``aquilo''. Podemos
repetir a última substituição globalmente assim:
   
\begin{verbatim}
     g&
\end{verbatim}

\section{Repetindo comandos}\label{Repetindo comandos}

\begin{verbatim}
     @:
\end{verbatim}

O atalho acima repete o último comando no próprio modo de comandos

\section{{\em Scripts} Vim}\label{Scripts Vim}
Usando um {\em script} para modificar um nome em vários arquivos: 
Crie um arquivo chamado {\tt subst.vim} contendo os comandos de substituição e o
comando de salvamento {\tt :wq}.

\begin{verbatim}
     %s/bgcolor="e"/bgcolor="e"/g
     wq
\end{verbatim}

Para executar um {\em script}, digite o comando

\begin{verbatim}
     :source nome_do_script.vim
\end{verbatim}

O comando {\tt :source} também pode ser abrevidado com {\tt :so}
bem como ser usado para testar um esquema de cor:

\begin{verbatim}
    :so tema.vim
\end{verbatim}

\section{Usando o comando {\tt bufdo}}\label{Usando o comando bufdo}

Com o comando {\tt :bufdo} podemos executar um comando em um
conjunto de arquivos de forma rápida. No exemplo a seguir, abriremos
todos os arquivos HTML do diretório atual, efetuarei uma substituição
e em seguida salvo todos.

\begin{verbatim}
     vim *.html
     :bufdo %s/bgcolor="eeeeee"/bgcolor="ffffff"/g | :wall
     :qall
\end{verbatim}

Para fechar todos os arquivos faça:

\begin{verbatim}
     :qall
\end{verbatim}

O comando {\tt :wall} salva ``{\em write}'' todos ``{\em all}'' os arquivos
abertos pelo comando {\tt vim *.html}. Opcionalmente você pode
combinar ``{\tt :wall}'' e ``{\tt :qall}'' com o comando {\tt :wqall}, que
salva todos os arquivos abertos e em seguida sai do Vim.

\section{Colocando a última busca em um comando }
Observação: lembre-se \verb|Ctrl = ^|

\begin{verbatim}
     :^r/
\end{verbatim}

\section{Inserindo o nome do arquivo no comando }

\begin{verbatim}
     :^r%
\end{verbatim}

\section{Inserindo o último comando }

\begin{verbatim}
     ^r:
\end{verbatim}

Se preceder com ``:'' você repete o comando, equivale a acessar o histórico de
comandos com as setas

\begin{verbatim}
     :^r:
\end{verbatim}

\section{Para repetir exatamente a última inserção }

\begin{verbatim}
     i<c-a>
\end{verbatim}

%%%%%%%%%%%%%%%%%%%%%%%%%%%%%%%%%%%%%%%%%%%%%%%%%%%%%%%%%%%%%%%%%%%%%%%%
% vim:enc=utf-8:ts=5:sw=5:et:ff=unix:
%%%%%%%%%%%%%%%%%%%%%%%%%%%%%%%%%%%%%%%%%%%%%%%%%%%%%%%%%%%%%%%%%%%%%%%%

\chapter{Comandos Externos}
O Vim permite executar comandos externos para processar ou filtrar o
conteúdo de um arquivo. De forma geral, fazemos isso digitando (no
modo normal):

\begin{verbatim}
     :!ls .... visualiza o conteúdo do diretório
\end{verbatim}

Lembrando que anexando um simples ponto, a saída do comando torna-se o 
documento que está sendo editado:

\begin{verbatim}
     :.!ls .... imprime na tela o conteúdo do diretório
\end{verbatim}

A seguir, veja alguns exemplos de utilização:

\section{Ordenando}
Podemos usar o comando {\em sort} que ordena o conteúdo de um arquivo dessa forma:

\begin{verbatim}
     :5,15!sort ..... ordena da linha 5 até a linha 15
\end{verbatim}

O comando acima ordena da linha 5 até a linha 15.

O comando {\em sort} existe tanto no Windows quanto nos sistemas Unix.
Digitando simplesmente {\em sort}, sem argumentos, o comportamento padrão
é de classificar na ordem alfabética (baseando-se na linha inteira).
Para mais informações sobre argumentos do comando {\em sort}, digite:

\begin{verbatim}
     sort --help ou man sort (no Unix) ou
     sort /? (no Windows).
\end{verbatim}

\section{Removendo linhas duplicadas}

\begin{verbatim}
     :%!uniq
\end{verbatim}

O caractere ``\%'' representa a região equivalente ao arquivo atual inteiro.
A versão do Vim 7 em diante tem um comando {\em sort} que permite remover
linhas duplicadas {\em uniq} e ordenar, sem a necessidade de usar comandos
externos, para mais detalhes:

\begin{verbatim}
     :h sort
\end{verbatim}

\section{Ordenando e removendo linhas duplicadas no Vim 7}

\begin{verbatim}
     :sort u
\end{verbatim}

Quando a ordenação envolver números faz-se:

\begin{verbatim}
     :sort n
\end{verbatim}

\section{{\em Beautifiers}}

A maior parte das linguagens de programação possui ferramentas
externas chamadas {\em beautifiers}, que servem para embelezar o código,
através da indentação e espaçamento. Por exemplo, para embelezar um
arquivo HTML é possível usar a ferramenta ``tidy\footnote{http://tidy.sourceforge.net/}'', do W3C:

\begin{verbatim}
     :%!tidy
\end{verbatim}

\section{Editando comandos longos no Linux}
\label{Editando comandos longos no Linux}

É comum no ambiente GNU/Linux a necessidade de digitar comandos longos
no terminal, para facilitar esta tarefa pode-se seguir estes passos:

\begin{enumerate}
     \item Definir o Vim como editor padrão do sistema editando 
           o arquivo `{\tt .bashrc}\footnote{Arquivo de configuração do bash}':
           \begin{verbatim}
               #configura o vim como editor padrão
               export EDITOR=vim
               export VISUAL=vim
            \end{verbatim}
      \item No terminal usar a combinação de teclas `{\tt Ctrl-x-e}'.
            Esta combinação de teclas abre o editor padrão do sistema
            onde se deve digitar o comando longo, ao sair do editor 
            o terminal executa o comando editado.
\end{enumerate}

\section{Compilando e verificando erros}
\vimhelp{cwindow quickfix-window}

Se o seu projeto já possui um {\tt Makefile}, então você pode fazer uso do comando
{\tt :make} para poder compilar seus programas no conforto de seu Vim:

\begin{verbatim}
     :make
\end{verbatim}

A vantagem de fazer isso é poder usar outra ferramenta bastante interessante, a janela
de {\em quickfix}:

\begin{verbatim}
     :cw[indow]
\end{verbatim}

O comando {\tt cwindow} abrirá uma janela em um {\em split} horizontal com a
listagem de erros e {\em warnings}.  Você poderá navegar pela lista usando os
cursores e ir diretamente para o arquivo e linha da ocorrência.

Modificando o compilador, o comando {\tt make} pode mudar sua ação.

\begin{verbatim}
	:compiler javac
	:compiler gcc
	:compiler php
\end{verbatim}

Note que {\em php} não tem um compilador. Logo, quando executado, o {\tt make} irá verificar
por erros de sintaxes.

\begin{verbatim}
	:compiler
\end{verbatim}

O comando acima lista todos os compiladores suportados.

\section{Grep}
\label{sec:Grep}
\vimhelp{grep quickfix-window cw}

Do mesmo jeito que você usa {\tt grep} na sua linha de comando você pode usar
o {\tt grep} interno do Vim. Exatamente do mesmo jeito:

\begin{verbatim}
     :grep <caminho> <padrão> <opções>
\end{verbatim}

Use a janela de {\em quickfix}\footnote{{\tt :cope}} aqui também para exibir os resultados do {\tt
grep} e poder ir diretamente a eles.

\section{Indent}

{\tt Indent}\footnote{http://www.gnu.org/software/indent} 
é um programa que indenta seu código fonte de acordo com os padrões configurados
no seu arquivo {\tt HOME/.indent.pro}. Vou pressupor que você já saiba usar o {\tt indent}
e como fazer as configurações necessárias para ele funcionar, então vamos ao funcionamento 
dele no Vim:

Para indentar um bloco de código, primeiro selecione-o com o modo {\em visual line} (com {\tt V}), 
depois é só entrar com o comando como se fosse qualquer outro comando externo:
\begin{verbatim}
     :!indent
\end{verbatim}

No caso, como foi selecionado um bloco de código, irão aparecer alguns caracteres extras, 
mas o procedimento continua o mesmo:
\begin{verbatim}
     :'<,'>!indent
\end{verbatim}


\section{Calculadora Científica com o Vim}
\label{sec:Calculadora Científica com o Vim}

Para usar a função de Calculadora Científica no Vim usamos uma ferramenta
externa, que pode ser o comando `{\tt bc}' do GNU/Linux, ou uma linguagem de
programação como {\em Python} ou {\em Ruby}, veremos como habilitar a
calculadora usando o {\em Python}. Obviamente esta linguagem de programação
deve estar instalada no sistema em que se deseja usar seus recursos.  Deve-se
testar se a versão do Vim tem suporte ao Python ``\verb+:version+'', em seguida
colocam-se os mapeamentos no ``.vimrc''.

\begin{verbatim}
     :command! -nargs=+ Calc :py print <args>
     :py from math import *
\end{verbatim}

Feito isto pode-se usar o comando ``{\tt :Calc}'' como visto abaixo:

\begin{verbatim}
     :Calc pi
     :Calc cos(30)
     :Calc pow(5,3)
     :Calc 10.0/3
     :Calc sum(xrange(1,101))
     :Calc [x**2 for x in range(10)] 
\end{verbatim}

\section{Editando saídas do Shell}
\label{sec:Editando saídas do Shell}

Muitas vezes, precisamos manipular saídas do shell antes de enviá-las por e-mail, reportar ao chefe ou até mesmo 
salvá-las. Utilizando

\begin{verbatim}
     vim -
     ou
     gvim -
\end{verbatim}

a saída do Shell é redirecionada para o (G)Vim automaticamente, não sendo
necessário redirecioná-la para um arquivo temporário e, logo após, abrí-lo para
editá-lo e modificá-lo.

Outra situação em que se pode combinar o vim com saidas do shell é com o
comando `\verb|grep|'. Usando-se a opção `\verb|-l|' do grep listamos apenas os
arquivos que correspondem a um padrão.

\begin{verbatim}
    grep -irl voyeg3r .
    ./src/img/.svn/entries
    ./src/Makefile
    ./src/vimbook.tex
\end{verbatim}

Pode-se em seguida chamar o vim usando substituição de comandos, como o comando
`\verb|!!|' corresponde ao último comando, e neste caso a saida corresponde a
uma lista de arquivos que contém o padrão a ser editado faz-se:

\begin{verbatim}
    vim `!!`
\end{verbatim}

\section{Log do Subversion}

A variável de ambiente {\em \$SVN\_EDITOR} pode ser usada para se especificar o caminho para o editor de texto de
sua preferência, a fim de usá-lo na hora de dar um {\em commit} usando o {\em subversion}.

\begin{verbatim}
     export SVN_EDITOR=/usr/bin/vim
     svn commit
\end{verbatim}

Será aberto uma sessão no Vim, que depois de salva, será usada para LOG do commit.

\section{Referências}

\begin{itemize}
 \item \url{http://www.dicas-l.com.br/dicas-l/20070119.php}
 \item \url{http://vim.wikia.com/wiki/Scientific_calculator}
 \item \url{http://docs.python.org/library/cmath.html}
 \item \url{http://docs.python.org/library/math.html}
\end{itemize}

%%%%%%%%%%%%%%%%%%%%%%%%%%%%%%%%%%%%%%%%%%%%%%%%%%%%%%%%%%%%%%%%%%%%%%%%%%%%%%%%

%%%%%%%%%%%%%%%%%%%%%%%%%%%%%%%%%%%%%%%%%%%%%%%%%%%%%%%%%%%%%%%%%%%%%%%%
% vim:enc=utf-8:ts=5:sw=5:et:ff=unix:
%%%%%%%%%%%%%%%%%%%%%%%%%%%%%%%%%%%%%%%%%%%%%%%%%%%%%%%%%%%%%%%%%%%%%%%%

\chapter{Verificação Ortográfica}
\label{cha:vero}

\vimhelp{spell}

O Vim possui um recurso nativo de verificação ortográfica ({\em spell}) em
tempo de edição, apontando palavras e expressões desconhecidas---usualmente
erros de grafia---enquanto o usuário as digita. 

Basicamente, para cada palavra digitada o Vim procura por sua grafia em um
dicionário. Não encontrando-a, a palavra é marcada como desconhecida
(sublinhando-a ou alterando sua cor), e fornece ao usuário mecanismos para
{\em corrigi-la} (através de sugestões) ou {\em cadastrá-la} no dicionário
caso esteja de fato grafada corretamente.

\section{Habilitando a verificação ortográfica}
\vimhelp{spell, spelllang}

A verificação ortográfica atua em uma linguagem (dicionário) por vez,
portanto, sua efetiva habilitação depende da especificação desta linguagem.
Por exemplo, para habilitar no arquivo em edição a verificação ortográfica na
língua portuguesa ({\em pt}), assumindo-se a existência do dicionário em
questão:

\begin{verbatim}
     :setlocal spell spelllang=pt
\end{verbatim}

ou de forma abreviada:

\begin{verbatim}
     :setl spell spl=pt
\end{verbatim}


Trocando-se {\tt setlocal} ({\tt setl}) por apenas {\tt set} ({\tt se}) faz
com que o comando tenha efeito global, isto é, todos os arquivos da sessão
corrente do Vim estariam sob efeito da verificação ortográfica e do mesmo
dicionário (no caso o {\tt pt}).

A desabilitação da verificação dá-se digitando:

\begin{verbatim}
     :setlocal nospell
     :set nospell            (efeito global)
\end{verbatim}

Caso queira-se apenas alterar o dicionário de verificação ortográfica, suponha
para a língua inglesa ({\tt en}), basta:

\begin{verbatim}
     :setlocal spelllang=en
     :set spelllang=en       (efeito global)
\end{verbatim}

\subsection{Habilitação automática na inicialização}
\vimhelp{autocmd, Filetype, BufNewFile, BufRead}

Às vezes torna-se cansativo a digitação explícita do comando de habilitação da
verificação ortográfica sempre quando desejada.  Seria conveniente se o Vim
habilitasse automaticamente a verificação para aqueles tipos de arquivos que
comumente fazem uso da verificação ortográfica, como por exemplo arquivos
``texto''. Isto é possível editando-se o arquivo de configuração do Vim {\tt
.vimrc} (veja Cap.~\ref{cha:Como editar preferências no Vim}) e incluindo as
seguintes linhas: 

\begin{verbatim}
     autocmd Filetype text setl spell spl=pt
     autocmd BufNewFile,BufRead *.txt setl spell spl=pt
\end{verbatim}

Assim habilita-se automaticamente a verificação ortográfica usando o
dicionário da língua portuguesa ({\tt pt}) para arquivos do tipo {\tt texto} e
os terminados com a extensão {\tt .txt}. Mais tecnicamente, diz-se ao Vim para
executar o comando \verb|setl spell spl=pt| sempre quando o tipo do arquivo
({\tt Filetype}) for {\tt text} (texto) ou quando um arquivo com extensão {\tt .txt}
for carregado ({\tt BufRead}) ou criado ({\tt BufNewFile}).

\section{O dicionário de termos}

A qualidade da verificação ortográfica do Vim está diretamente ligada à
completude e corretude do dicionário da linguagem em questão. Dicionários
pouco completos são inconvenientes à medida que acusam falso positivos em
demasia; pior, dicionários contendo palavras grafadas incorretamente, além de
acusarem falso positivos, induzem o usuário ao erro ao sugerirem grafias
erradas.

É razoavelmente comum o Vim já vir instalado com dicionários de relativa
qualidade para algumas linguagens (ao menos inglês, habitualmente).
Entretanto, ainda é raro para a maioria das instalações do Vim trazer por {\em
default} um dicionário realmente completo e atualizado da língua portuguesa. A
próxima seção sintetiza, pois, os passos para a instalação de um excelente---e
disponível livremente---dicionário de palavras para a língua portuguesa.

\subsection{Dicionário português segundo o acordo ortográfico}

A equipe do projeto {\tt BrOffice.org}\footnote{\url{http://www.broffice.org}}
e seus colaboradores mantêm e disponibilizam livremente um grandioso dicionário
de palavras da língua portuguesa. Além do expressivo número de termos, o
dicionário contempla as mudanças ortográficas definidas pelo {\em Acordo
Ortográfico}\footnote{\url{http://pt.wikipedia.org/wiki/Acordo_Ortográfico_de_1990}}
que entraram em vigor no início de 2009.

A instalação envolve três passos, são eles: 
\begin{enumerate}
     \item obtenção do dicionário através do site {\tt BrOffice.org}; 
     \item conversão para o formato interno de dicionário do Vim; e 
     \item instalação dos arquivos resultantes.
\end{enumerate}

\subsubsection{Obtenção do dicionário}

O dicionário pode ser obtido através do endereço: 
\begin{itemize}
\item[] \url{http://www.broffice.org/verortografico/baixar}
\end{itemize}

O arquivo baixado encontra-se compactado no formato {\tt Zip}, bastando
portanto descompactá-lo com qualquer utilitário compatível com este formato,
por exemplo, o comando {\tt unzip}.

\subsubsection{Conversão do dicionário}
\vimhelp{mkspell}

Após a descompactação, os arquivos \verb|pt_BR.aff| e \verb|pt_BR.dic|,
encontrados no subdiretório {\tt dictionaries/}, serão usados para a criação
dos dicionários no formato interno do Vim\footnote{O formato interno de dicionário
do Vim assegura melhor desempenho, em termos de agilidade e consumo de
memória, quando a verificação ortográfica do editor encontra-se em operação.}.
A conversão propriamente dita é feita pelo próprio Vim através do comando {\tt
mkspell}:

\begin{enumerate}
\item Carrega-se o Vim a partir do diretório {\tt dictionaries/}
\item O comando {\tt mkspell} é então executado como:
\begin{verbatim}
     :mkspell pt pt_BR
\end{verbatim}
\end{enumerate}

O Vim então gera um arquivo de dicionário da forma
\verb|pt.<codificação>.spl|, dentro de {\tt dictionaries/}, onde
\verb|<codificação>| é a codificação de caracteres do sistema, normalmente
\verb|utf-8| ou \verb|latin1|; caso queira-se um dicionário em uma codificação
diferente da padrão será preciso ajustar a variável {\tt encoding} antes da
invocação do comando {\tt mkspell}:

\begin{verbatim}
     :set encoding=<codificação>
     :mkspell pt pt_BR
\end{verbatim}

\subsubsection{Instalação do(s) dicionário(s) gerado(s)}
\vimhelp{runtimepath}

Finalmente, o dicionário gerado---ou os dicionários, dependendo do uso ou não
de codificações diferentes---deve ser copiado para o subdiretório {\tt spell/}
dentro de qualquer caminho (diretório) que o Vim ``enxergue''. A lista de
caminhos lidos pelo Vim encontra-se na variável {\tt runtimepath}, que pode
ser inspecionada através de:

\begin{verbatim}
     :set runtimepath
\end{verbatim}

É suficiente então copiar o dicionário \verb|pt.<codificação>.spl| para o
subdiretório {\tt spell/} em qualquer um dos caminhos listados através do
comando mostrado. 

\section{Comandos relativos à verificação ortográfica}

\subsection{Encontrando palavras desconhecidas}

Muito embora o verificador ortográfico cheque imediatamente cada palavra
digitada, sinalizando-a ao usuário caso não a reconheça, às vezes é mais
apropriado realizar a verificação ortográfica do documento por inteiro.
O Vim dispõe de comandos específicos para busca e movimentação em palavras
grafadas incorretamente (desconhecidas) no escopo do documento, dentre eles:

\begin{verbatim}
     ]s ..... vai para a próxima palavra desconhecida
     [s ..... como o ]s, mas procura no sentido oposto
\end{verbatim}

Ambos os comandos aceitam um prefixo numérico, que indica a quantidade de
movimentações (buscas). Por exemplo, o comando {\tt 3]s} vai para a {\em
terceira} palavra desconhecida a partir da posição atual.

\subsection{Tratamento de palavras desconhecidas}

Há basicamente duas operações possíveis no tratamento de uma palavra apontada
pelo verificador ortográfico do Vim como desconhecida: 

\begin{enumerate}
\item {\bf corrigi-la} -- identificando o erro com ou sem o auxílio das
sugestões do Vim.
\item {\bf cadastrá-la no dicionário} -- ensinando o Vim a reconhecer sua
grafia.
\end{enumerate}

Assume-se nos comandos descritos nas seções a seguir que o cursor do editor
encontra-se sobre a palavra marcada como desconhecida.

\subsubsection{Correção de palavras grafadas incorretamente}

É possível que na maioria das vezes o usuário perceba qual foi o erro cometido
na grafia, de forma que o próprio possa corrigi-la sem auxílio externo. No
entanto, algumas vezes o erro não é evidente, e sugestões fornecidas pelo Vim
podem ser bastante convenientes. Para listar as sugestões para a palavra
em questão executa-se:

\begin{verbatim}
     z= ..... solicita sugestões ao verificador ortográfico
\end{verbatim}

Se alguma das sugestões é válida---as mais prováveis estão nas primeiras
posições---então basta digitar seu prefixo numérico e pressionar {\tt
<Enter>}. Se nenhuma sugestão for adequada, basta simplesmente pressionar {\tt
<Enter>} e ignorar a correção.

\subsubsection{Cadastramento de novas palavras no dicionário}

Por mais completo que um dicionário seja, eventualmente palavras,
especialmente as de menor abrangência, terão que ser cadastradas a fim de
aprimorar a exatidão da verificação ortográfica. A manutenção do dicionário 
dá-se pelo cadastramento e retirada de palavras:

\begin{verbatim}
     zg ..... adiciona a palavra no dicionário
     zw ..... retira a palavra no dicionário, marcando-a como 
              `desconhecida'
\end{verbatim}

%%%%%%%%%%%%%%%%%%%%%%%%%%%%%%%%%%%%%%%%%%%%%%%%%%%%%%%%%%%%%%%%%%%%%%%%
% vim:enc=utf-8:ts=5:sw=5:et:ff=unix:
%%%%%%%%%%%%%%%%%%%%%%%%%%%%%%%%%%%%%%%%%%%%%%%%%%%%%%%%%%%%%%%%%%%%%%%%

\chapter{Salvando Sessões de Trabalho}
\label{cha:Salvando Sessões de Trabalho}

Suponha a situação em que um usuário está trabalhando em um projeto no qual
vários arquivos são editados simultaneamente; quatro arquivos estão abertos,
algumas macros foram criadas e variáveis que não constam no \verb|vimrc| foram
definidas. Em uma situação normal, se o Vim for fechado a quase totalidade
dessas informações se perde\footnote{Algumas informações, no entanto, são
automaticamente armazenadas no arquivo {\tt viminfo}; veja {\tt :h viminfo} }; 
para evitar isto uma sessão pode ser criada, gerando-se um
``retrato do estado atual'', e então restaurada futuramente pelo
usuário---na prática é como se o usuário não tivesse saído do editor. 

Uma sessão do Vim guarda, portanto, uma série de informações sobre a edição
corrente, de modo a permitir que o usuário possa restaurá-la quando desejar.
Sessões são bastante úteis, por exemplo, para se alternar entre diferentes
projetos, carregando-se rapidamente os arquivos e definições relativas a cada
projeto.

\section{O que uma sessão armazena?}

Uma sessão é composta das seguintes informações:

\begin{itemize}
 \item Mapeamentos globais
 \item Variáveis globais
 \item Arquivos abertos incluindo a lista de {\it buffers}
 \item Diretório corrente ({\tt :h curdir})
 \item Posição e tamanho das janelas (o {\em layout}) 
\end{itemize}

\section{Criando sessões}
\label{sec:Criando sessões}

Sessões são criadas através do comando \verb|:mksession|:

\begin{verbatim}
     :mksession sessao.vim .... cria a sessão e armazena-a em sessao.vim
     :mksession! sessao.vim ... salva a sessão e sobrescreve-a em sessao.vim
\end{verbatim}

\section{Restaurando sessões}

Após gravar sessões, elas podem ser carregadas ao iniciar o Vim:

\begin{verbatim}
     vim -S sessao.vim
\end{verbatim}

ou então de dentro do próprio Vim (no modo de comando):

\begin{verbatim}
     :so sessao.vim
\end{verbatim}

Após restaurar a sessão, o nome da sessão corrente é acessível através de uma
variável interna ``\verb+v:this_session+''; caso queira-se exibir na linha de
comando o nome da sessão ativa (incluindo o caminho), faz-se:

\begin{verbatim}
     :echo v:this_session
\end{verbatim}

Podemos fazer mapeamentos para atualizar a sessão atual e exibir detalhes da
mesma:

\begin{verbatim}
     "mapeamento para gravar sessão
     nmap <F4> :wa<Bar>exe "mksession! " . v:this_session<CR>:so ~/sessions/

     "mapeamento para exibir a sessão ativa
     map <s-F4> <esc>:echo v:this_session<cr>
\end{verbatim}

\section{{\tt Viminfo}}\label{sec:Viminfo}

Se o Vim for fechado e iniciado novamente, normalmente perderá uma porção
considerável de informações. A diretiva {\tt viminfo} pode ser usada para
memorizar estas informações.

\begin{itemize}
\item Histórico da linha de comando
\item Histórico de buscas
\item Histórico de entradas {\em input-line history}
\item Conteúdo de registros não vazios
\item Marcas de vários arquivos
\item Último padrão de busca/substituição 
\item A lista de {\em buffers}
\item Variáveis globais
\end{itemize}

Deve-se colocar no arquivo de configuração algo como:
 
\begin{verbatim}
set viminfo=%,'50,\"100,/100,:100,n
\end{verbatim}

Algumas opões da diretiva {\tt viminfo}:

\begin{description}
\item [!] Quando incluído salva e restaura variáveis globais (variáveis
com letra maiúscula) e que não contém letras em minúsculo como MANTENHAISTO.

\item ["] Número máximo de linhas salvas para cada registro.

\item [\%] Quando incluído salva e restaura a lista de {\em buffers}. Caso o Vim seja 
iniciado com um nome como argumento, a lista de {\em buffers} não é restaurada. {\em Buffers} sem
nome e {\em buffers} de ajuda não são armazenados no {\tt viminfo}.
                                      
\item ['] Número máximo de arquivos recém editados.

\item [/] Máximo de itens do histórico de buscas.

\item [:] Máximo de itens do histórico da linha de comando

\item [<] Número máximo de linhas salvas por cada registro, se zero
os registros não serão salvos. Quando não incluído, todas as linhas são salvas.

\end{description}

Para ver mais opções sobre o arquivo `{\tt viminfo}' leia `{\tt :h viminfo}'.
Pode-se também usar um arquivo de ``Sessão''. A diferença é que `{\tt viminfo}' não 
depende do local de trabalho (escopo).  
Quando o arquivo `{\tt viminfo}' existe e não está vazio, as informações novas
são combinadas com as existentes. A opção `{\tt viminfo}' é uma string contendo
informações sobre o que deve ser armazenado, e contém limites de o quanto 
vai ser armazenado para cada item.

%%%%%%%%%%%%%%%%%%%%%%%%%%%%%%%%%%%%%%%%%%%%%%%%%%%%%%%%%%%%%%%%%%%%%%%%%%%%%%%%

%%%%%%%%%%%%%%%%%%%%%%%%%%%%%%%%%%%%%%%%%%%%%%%%%%%%%%%%%%%%%%%%%%%%%%%%
% vim:enc=utf-8:ts=5:sw=5:et:ff=unix:
%%%%%%%%%%%%%%%%%%%%%%%%%%%%%%%%%%%%%%%%%%%%%%%%%%%%%%%%%%%%%%%%%%%%%%%%

\chapter{Como Editar Preferências no Vim}\label{cha:Como editar preferências no Vim}

O arquivo de preferências do Vim é nomeado {\tt vimrc}, um arquivo oculto que
normalmente encontra-se no diretório de trabalho ({\em home}) do usuário:

\begin{verbatim}
     ~/.vimrc
     /home/usuario/.vimrc
\end{verbatim}

No sistema operacional Windows o arquivo costuma ser:

\begin{verbatim}
     ~\_vimrc
     c:\documents and settings\usuario\_vimrc
\end{verbatim}

\section{Onde colocar {\em plugins} e temas de cor}
\label{Onde colocar plugins e temas de cor}

No Windows deve haver uma pasta chamada `{\tt vimfiles}' (caso não exista
deve-se criá-la), que fica em:

\begin{verbatim}
     c:\documents and settings\usuario\vimfiles
\end{verbatim}

No GNU/Linux a pasta de arquivos do Vim é chamada {\tt .vim}, comumente
armazenada em

\begin{verbatim}
     /home/user/.vim
\end{verbatim}

Tanto em {\tt .vim} como {\tt vimfiles} encontram-se usualmente as seguintes
pastas:

\begin{verbatim}
     vimfiles ou .vim
        |
        +--color
        |
        +--doc
        |
        +--syntax
        |
        +--plugin
\end{verbatim}

Os {\em plugins}, como se pode deduzir, devem ser colocados no diretório
denominado {\tt plugin}. {\Large \ding{45}} Na seção Plugins~\ref{Plugins} (p.
\pageref{Plugins}) estão descritos alguns {\em plugins} para o Vim.

\section{Comentários }
\label{Comentários }

Comentários são linhas que são ignoradas pelo interpretador Vim e servem
normalmente para descrição de comandos e ações, deixando portanto mais legível
e didático o arquivo de configuração. Uma linha é um comentário se seu
primeiro caractere é uma aspa `\verb+"+':

\begin{verbatim}
     " linhas começadas com aspas são comentários
     " e portanto serão ignoradas pelo Vim
\end{verbatim}

Recomenda-se usar comentários ao adicionar ou modificar comandos no arquivo
{\tt vimrc}, pois assim torna-se mais fácil futuras leituras e modificações
neste arquivo.

\section{Efetivação das alterações no {\tt vimrc}}
\label{Efetivação das alterações no vimrc}

As alterações no {\tt vimrc} só serão efetivadas na próxima vez que o Vim for
aberto, a não ser que o recarregamento do arquivo de configuração seja
instruído explicitamente:

\begin{verbatim}
     :source ~/vimrc ....... se estiver no GNU/Linux
     :source ~/_vimrc ...... caso use o Windows
     :so arquivo ........... `so' é uma abreviação de `source'
\end{verbatim}

\section{{\em Set}}
\label{Set}

Os comandos {\tt set}, responsáveis por atribuir valores à variáveis,
 podem ser colocados no \verb|.vimrc|:

\begin{verbatim}
     set nu
\end{verbatim}

ou digitados como comandos:

\begin{verbatim}
     :set nu
\end{verbatim}


%%%%%%%%%%%%%%%%%%%%%%%%%%%%%%%%%%%%%%%%%%%%%%%%%%%%%%%%%%%%%%
% ESTE TRECHO NÃO PODE FICAR INDENTADO POIS SE ASSIM O FOR  %%
% ELE NÃO CABERÁ NA PÁGINA!!!!                              %%
%%%%%%%%%%%%%%%%%%%%%%%%%%%%%%%%%%%%%%%%%%%%%%%%%%%%%%%%%%%%%%
\index{cores!esquemas}
\begin{verbatim}
set number ............... "mostra numeração de linhas
set nu ................... "simplificação de `number'
set showmode ............. "mostra o modo em que estamos
set showcmd .............. "mostra no status os comandos inseridos
set tabstop=4 ............ "tamanho das tabulações
set ts=4 ................. "simplificação de `tabstop'
set shiftwidth=4 ......... "quantidade de espaços de uma tabulação
set sw=4 ................. "simplificação de `shiftwidth'
syntax on ................ "habilita cores
syn on ................... "simplificação de `syntax'
colorscheme tema ......... "esquema de cores `syntax highlight'
autochdir ................ "configura o diretório de trabalho
set hls .................. "destaca com cores os termos procurados
set incsearch ............ "habilita a busca incremental
set ai ................... "auto identação
set aw ................... "salva automaticamente ao trocar de `buffer'
set ignorecase ........... "ignora maiúsculas e minúsculas nas buscas
set ic ................... "simplificação de ignorecase
set smartcase ............ "numa busca em maiúsculo habilita `case'
set scs .................. "sinônimo de `smartcase'
set backup ............... "habilita a criação de arquivos de backup
set bk ................... "simplificação de `backup'
set backupext=.backup .... "especifica a extensão do arquivo de backup
set bex=.backup .......... "simplificação de backupext
set backupdir=~/.backup,./ "diretório(s) para arquivos de backup
set bdir ................. "simplificação de `backupdir'
set nobackup ............. "evita a criação de arquivos de backup
ste nobk ................. "simplificação de `nobackup'
set cursorline ........... "abreviação de cursor line (destaca linha atual)
set cul .................. "simplificação de `cursorline'
set ttyfast .............. "melhora o redraw de janelas.
set columns=88 ........... "deixa a janela com 88 colunas.
set mousemodel=popup ..... "exibe o conteúdo de folders e sugestões spell
set viminfo=%,'50,\"100,/100,:100,n "armazena opções (buffers)
\end{verbatim}

\section{Ajustando o parágrafos em modo normal} % (fold)
\label{sec:Ajustando o parágrafos em modo normal}
\index{parágrafo!ajustar}

O comando `\verb|gqap|' ajusta o parágrafo atual em modo normal. 
{\Large {\ding{45}}} usando a opção `{\tt :set nojoinspaces}' o vim 
colocará dois espaços após o ponto final ao se ajustar os parágrafos.
geralmente usamos `\verb+^I+' para representar uma tabulação        
<Tab>, e `\verb+$+' para indicar o fim de linha. Mas é possível     
customizar essas opções. sintaxe:                          

\begin{verbatim}
    
     set listchars=key:string,key:string                        

      - eol:{char} 
      Define o caracter a ser posto depois do fim da linha  

      - tab:{char1}{char2} 
      O tab é mostrado pelo primeiro caracter {char1} e     
      seguido por {char2}                                   

      - trail:{char}                                            
      Esse caracter representa os espaços em branco.        
                                                                
      - extends:{char}                                          
      Esse caracter representa o início do fim da linha      
      sem quebrá-la                                          
      Está opção funciona com a opção nowrap habilitada       
                                                                
     "exemplo 1:
     "set listchars=tab:>-,trail:.,eol:#,extends:@
     
     "exemplo 2:
     "set listchars=tab:>-
     
     "exemplo 3:
     "set listchars=tab:>-
     
     "exemplo 4:
     set nowrap    "Essa opção desabilita a quebra de linha
     "set listchars=extends:+
     
     Caso esteja usando o gvim pode setar um esquema de cores
     set colo desert
\end{verbatim}

% fim da seção Ajustando o parágrafos em modo normal

\section{Exibindo caracteres invisíveis}
\label{Exibindo caracteres invisíveis}

\begin{verbatim}
     :set list
\end{verbatim}

\section{Definindo macros previamente}
\label{Definindo macros previamente}
Definindo uma macro de nome \verb|s| para ordenar e retirar linhas duplicadas

\begin{verbatim}
     let @s=":sort u"
\end{verbatim}

Para executar a macro \verb|s| definida acima faça:

\begin{verbatim}
     @s
\end{verbatim}

O Vim colocará no comando

\begin{verbatim}
     :sort -u
\end{verbatim}

Bastando pressionar \verb|<Enter>|.
Observação: esta macro prévia pode ficar no {\tt vimrc} ou ser digitada em comando ``:''


\begin{verbatim}
     :5,20sort u
     "da linha 5 até a linha 20 ordene e retire duplicados
     
     :sort n
     " ordene meu documento considerando números
     " isto é útil pois se a primeira coluna contiver
     " números a ordenação pode ficar errada caso não usemos
     " o parâmetro ``n''
\end{verbatim}

\section{Mapeamentos}\label{Mapeamentos}
\vimhelp{key-mapping, mapping}

Mapeamentos permitem criar atalhos de teclas para quase tudo. Tudo depende da
criatividade do usuário e do quanto conhece o Vim, com eles podemos controlar ações
com quaisquer teclas, mas antes temos que saber que para criar mapeamentos,
precisamos conhecer a maneira de representar as teclas e combinações. Alguns
exemplos:

\begin{verbatim}
     tecla ....... tecla mapeada
     <c-x> ....... Ctrl-x
     <left> ...... seta para a esquerda
     <right> ..... seta para a direita
     <c-m-a> ..... Ctrl-Alt-a
     <cr> ........ Enter
     <Esc> ....... Escape
     <leader> .... normalmente \
     <bar> ....... | pipe
     <cword> ..... palavra sob o cursor
     <cfile> ..... arquivo sob o cursor
     <cfile> ..... arquivo sob o cursor sem extensão
     <sfile> ..... conteúdo do arquivo sob o cursor
     <left> ...... salta um caractere para esquerda
     <up> ........ equivale clicar em `seta acima'
     <m-f4> ...... a tecla alt -> m  mais a tecla f4
     <c-f> ....... Ctrl-f
     <bs> ........ backspace
     <space> ..... espaço
     <tab> ....... tab
\end{verbatim}

Para ler mais sobre atalhos de tecla no Vim acesse 

\begin{verbatim}
     :h index
\end{verbatim}

No Vim podemos mapear uma tecla para o modo normal, realizando
determinada operação e a mesma tecla pode desempenhar outra função
qualquer em modo de inserção ou comando, veja:

\begin{verbatim}
     " mostra o nome do arquivo com o caminho
     map <F2> :echo expand("%:p")
     " insere um texto qualquer
     imap <F2> Nome de uma pessoa
\end{verbatim}

A única diferença nos mapeamentos acima é que o mapeamento para modo de
inserção começa com `{\tt i}', assim como para o modo ``comando'' `{\tt :}'
começa com `{\tt c}' no caso `{\tt cmap}'.

\subsection{Recarregando o arquivo de configuração}
\label{sec:Recarregando o arquivo de configuração}

Cada alteração no arquivo de configuração do Vim só terá efeito na próxima vez que você
abrir o Vim a menos que você coloque isto dentro do mesmo

\begin{verbatim}
     " recarregar o vimrc
     " Source the .vimrc or _vimrc file, depending on system
     if &term == "win32" || "pcterm" || has("gui_win32")
        map ,v :e $HOME/_vimrc<CR>
        nmap <F12> :<C-u>source ~/_vimrc <BAR> echo "Vimrc recarregado!"<CR>
     else
        map ,v :e $HOME/.vimrc<CR>
        nmap <F12> :<C-u>source ~/.vimrc <BAR> echo "Vimrc recarregado!"<CR>
     endif
\end{verbatim}

Agora basta pressionar ``\verb|<F12>|'' em modo normal e as alterações passam a valer
instantaneamente, e para chamar o {\tt vimrc} basta usar.

\begin{verbatim}
     ,v
\end{verbatim}

Os mapeamentos abaixo são úteis para quem escreve códigos HTML, permitem
inserir caracteres reservados do HTML usando uma barra invertida para proteger
os mesmos, o Vim substituirá os ``barra alguma coisa'' pelo caractere
correspondente.

\begin{verbatim}
     inoremap \&amp; &amp;amp;
     inoremap \&lt; &amp;lt;
     inoremap \&gt; &amp;gt;
     inoremap \. &amp;middot;
\end{verbatim}

O termo {\em inoremap} significa: em modo de inserção não remapear, ou seja
ele mapeia o atalho e não permite que o mesmo seja remapeado, e o
mapeamento só funciona em modo de inserção, isso significa que um atalho
pode ser mapeado para diferentes modos de operação. \\

Veja este outro mapeamento:

\begin{verbatim}
     map <F11> <Esc>:set nu!<cr>
\end{verbatim}

Permite habilitar ou desabilitar números de linha do arquivo corrente.
A exclamação ao final torna o comando booleano, ou seja, se a
numeração estiver ativa será desabilitada, caso contrário será
ativada. O ``\verb|<cr>|'' ao final representa um {\tt Enter}.

\subsection{Limpando o ``registro'' de buscas}\label{Limpando o ``registro'' de buscas}

A cada busca, se a opção `{\tt hls}'\footnote{hls é uma abreviação de hightlight search}
estiver habilitada o Vim faz um destaque colorido, mas se você quiser limpar pode fazer este
mapeamento

\begin{verbatim}
     nno <S-F11> <Esc>:let @/=""<CR>
\end{verbatim}

É um mapeamento para o modo normal que faz com que a combinação de
teclas \verb|Shift-F11| limpe o ``registro'' de buscas

\subsection{Destacar palavra sob o cursor }
\label{Destacar palavra sob o cursor }

\begin{verbatim}
     nmap <s-f> :let @/=">"<CR>
\end{verbatim}

O atalho acima \verb|s-f| corresponde a \verb|Shift-f|.

\subsection{Remover linhas em branco duplicadas }
\label{Remover linhas em branco duplicadas }

\begin{verbatim}
     map ,d <Esc>:%s/\(^\n\{2,}\)/\r/g<cr>
\end{verbatim}

No mapeamento acima estamos associando o atalho:

\begin{verbatim}
     ,d
\end{verbatim}

\dots~à ação desejada, fazer com que linhas em branco sucessivas sejam
substituídas por uma só linha em branco, vejamos como funciona:

\begin{verbatim}
     map ......... mapear
     ,d .......... atalho que quermos
     <Esc> ....... se estive em modo de inserção sai
     : ........... em modo de comando
     % ........... em todo o arquivo
     s ........... substitua
     \n .......... quebra de linha
     {2,} ........ duas ou mais vezes
     \r .......... trocado por \r Enter
     g ........... globalmente
     <cr> ........ confirmação do comando
\end{verbatim}

As barras invertidas podem não ser usadas se o seu Vim estiver com a opção
{\em magic} habilitada

\begin{verbatim}
     :set magic
\end{verbatim}

Por acaso este é um padrão portanto tente usar assim pra ver se funciona

\begin{verbatim}
     map ,d :%s/\n{2,}/\r/g<cr>
\end{verbatim}

\subsection{Mapeamento para Calcular Expressões}
\label{sub:Mapeamento para Calcular Expressões}

Os mapeamentos abaixo exibem o resultado das quatro operações básicas (soma,
subtração, multiplicação e divisão). O primeiro para o modo normal no qual
posiciona-se o cursor no primeiro caractere da expressão tipo ``{\tt 5*9}'' e em
seguida pressiona-se ``{\tt Shift-F1}'', o segundo para o modo {\em insert} em que, após
digitada a expressão pressiona-se o mesmo atalho.

\begin{verbatim}
    " calculadora
    map <s-f1> <esc>0"myEA=<c-r>=<c-r>m<enter><esc>
    imap <s-f1> <space><esc>"myBEa=<c-r>=<c-r>m<enter><del>
\end{verbatim}

Para efetuar cálculos com maior precisão e também resolver problemas como
potências raízes, logaritmos pode-se mapear comandos externos, como a
biblioteca matemática da linguagem de programação Python.
\href{http://vim.wikia.com/wiki/Scientific\_calculator}{Neste link}~\cite{CientificCalculator} há um
manual que ensina a realizar este procedimento, ou acesse o capítulo \ref{Uma
calculadora diferente} na página~ \pageref{Uma calculadora diferente}.


\subsection{Mapeamentos globais}


Podemos fazer mapeamentos globais ou que funcionam em apenas um modo:

\begin{verbatim}
     map  - funciona em qualquer modo
     nmap - apenas no modo Normal
     imap - apenas no modo de Inserção
\end{verbatim}

Mover linhas com {\tt Ctrl-$\downarrow$} ou {\tt Ctrl-$\uparrow$}:

\begin{verbatim}
     " tem que estar em modo normal!
     nmap <C-Down> ddp
     nmap <C-Up> ddkP
\end{verbatim}

Salvando com uma tecla de função:

\begin{verbatim}
     " salva com F9
     nmap <F9> :w<cr>
     " F10 - sai do Vim
     nmap <F10> <Esc>:q<cr>
\end{verbatim}

\subsection{Convertendo as iniciais de um documento para maiúsculas}
\label{Convertendo as iniciais de um documento para maiúsculas}

\begin{verbatim}
     " MinusculasMaiusculas: converte a primeira letra de cada
     " frase para MAIÚSCULAS
     nmap ,mm :%s/\C\([.!?][])"']*\($\|[ ]\)\_s*\)\(\l\)/\1\U\3/g<CR>
     " Caso queira confirmação coloque uma letra ``c'' no final da 
     " linha acima:
     " (...) \3/gc<CR>
\end{verbatim}

\section{Autocomandos }
\label{Autocomandos }
\vimhelp{autocmd.txt}

Autocomandos habilitam comandos automáticos para situações
específicas. Para executar determinada ação ao
iniciar um novo arquivo o autocomando deverá obedecer este padrão:

\begin{verbatim}
     au BufNewFile tipo ação
\end{verbatim}

Veja um exemplo:

\begin{verbatim}
     au BufNewFile,BufRead *.txt source ~/.vim/syntax/txt.vim
\end{verbatim}

No exemplo acima o Vim aplica autocomandos para arquivos novos
``{\tt BufNewfile}'' ou existentes ``{\tt BufRead}'' terminados em \verb|txt|, e para
estes tipos carrega um arquivo de {\em syntax}\index{syntax}, ou seja, um esquema de cores
específico.

\begin{verbatim}
     " http://aurelio.net/doc/vim/txt.vim    coloque em ~/.vim/syntax
     au BufNewFile,BufRead *.txt source ~/.vim/syntax/txt.vim
\end{verbatim}

Para arquivos do tipo texto `{\tt *.txt}' use um arquivo de {\em syntax} em
particular.

O autocomando abaixo coloca um cabeçalho para {\em scripts} {\tt bash} caso a
linha 1 esteja vazia, observe que os arquivos em questão tem que ter a
extensão `{\tt .s}'.

\begin{verbatim}
     au BufNewFile,BufRead *.sh if getline(1) == "" | normal ,sh
\end{verbatim}

\subsection{Exemplo prático de autocomandos}
\label{sub:Exemplo prático de autocomandos}

Há situações em que é necessária a uniformização de ações, por exemplo, em
códigos Python deve-se manter um padrão para a indentação, ou será com espaços
ou será com tabulações, não se pode misturar os dois pois o interpretador retornaria um erro.
Outra situação em que misturar espaços com tabulações ocasiona erros é em
códigos \LaTeX, ao compilar o documento a formatação não sai como desejado.
Até que se perceba o erro leva um tempo.  Para configurar o vim de forma que
ele detecte este tipo de erro ao entrar no arquivo:

\begin{verbatim}
     au! VimEnter * match ErrorMsg /^\t\+/

     " explicação para o autocomando acima
     au! ............... automaticamente
     VimEnter .......... ao entrar no vim
     * ................. para qualquer tipo de arquivo
     match ErrorMsg .... destaque como erro
     / ................. inicio de um padrão
     ^ ................. começo de linha
     \t ................ tabulação
     \+ ................ uma vez ou mais
     / ................. fim do padrão de buscas
\end{verbatim}

Para evitar que este erro se repita, ou seja, que sejam 
adicionados no começo de linha tabulações no lugar de espaços
adiciona-se ao \textasciitilde/.vimrc

\begin{verbatim}
     set expandtab
\end{verbatim}

É perfeitamente possível um autocomando que faça direto a substituição de
tabulações por espaços, mas neste caso não é recomendado que o autocomando se
aplique a todos os tipos de aquivos.

\section{Funções}
\label{sec:Funções}

\subsection{Fechamento automático de parênteses}
\label{sec:Fechamento automático de parênteses}

\begin{verbatim}
     " --------------------------------------
     " Ativa fechamento automático para parêntese
     " Set automatic expansion of parenthesis/brackets
     inoremap ( ()<Esc>:call BC_AddChar(``)'')<cr>i
     inoremap { {}<Esc>:call BC_AddChar(``}'')<cr>i
     inoremap [ []<Esc>:call BC_AddChar(``]'')<cr>i
     `` '' ``''<Esc>:call BC_AddChar(``''")<cr>i
     "
     " mapeia Ctrl-j para pular fora de parênteses colchetes etc...
     inoremap <C-j> <Esc>:call search(BC_GetChar(), ``W'')<cr>a
     " Function for the above
     function! BC_AddChar(schar)
        if exists(``k'')
            let b:robstack = b:robstack . a:schar
        else
            let b:robstack = a:schar
        endif
     endfunction
     function! BC_GetChar()
        let l:char = b:robstack[strlen(b:robstack)-1]
        let b:robstack = strpart(b:robstack, 0, strlen(b:robstack)-1)
        return l:char
     endfunction
    
    '''Outra opção para fechamento de parênteses'''
    
     " Fechamento automático de parênteses
     imap { {}<left>
     imap ( ()<left>
     imap [ []<left>
    
     " pular fora dos parênteses, colchetes e chaves, mover o cursor
     " no modo de inserção
     imap <c-l> <Esc><right>a
     imap <c-h> <Esc><left>a
\end{verbatim}

\subsection{Função para barra de status}\label{Função para barra de status}

\begin{verbatim}
     set statusline=%F%m%r%h%w\
        [FORMAT=%{&ff}]\
        [TYPE=%Y]\
        [ASCII=\%03.3b]\
        [HEX=\%02.2B]\
        [POS=%04l,%04v][%p%%]\ [LEN=%L]
\end{verbatim}

Caso este código não funcione acesse
\href{http://vim.wikia.com/wiki/Writing\_a\_valid\_statusline}{este
link}~\cite{StatusLine}.


\subsection{Rolar outra janela}\label{Rolar outra janela}

Se você dividir janelas tipo

\begin{verbatim}
     Ctrl-w n
\end{verbatim}

pode colocar esta função no seu \verb|.vimrc|

\begin{verbatim}
     " rola janela alternativa
     fun! ScrollOtherWindow(dir)
     if a:dir == ``n''
        let move = ``>''
     elseif a:dir == ``p''
        let move = ``>''
     endif
     exec ``p'' . move . ``p''
     endfun
     nmap <silent> <M-Down> :call ScrollOtherWindow(``n'')<CR>
     nmap <silent> <M-Up> :call ScrollOtherWindow(``p'')<CR>
\end{verbatim}

Esta função é acionada com o atalho {\tt Alt-$\uparrow$} e {\tt Alt-$\downarrow$}.

\subsection{Função para numerar linhas}\label{Função para numerar linhas}
\index{linhas!numerar}

No site wikia há um código de função para.  link para a dica:
\href{http://vim.wikia.com/wiki/Number\_a\_group\_of\_lines}{numerar linhas}~\cite{NumerarLinhas}

\subsection{Função para trocar o esquema de cores}
\index{cores!esquemas}

\begin{verbatim}
     function! <SID>SwitchColorSchemes()
       if exists(``e'')
        if g:colors_name == 'native'
          colorscheme billw
        elseif g:colors_name == 'billw'
          colorscheme desert
        elseif g:colors_name == 'desert'
          colorscheme navajo-night
        elseif g:colors_name == 'navajo-night'
          colorscheme  zenburn
        elseif g:colors_name == 'zenburn'
          colorscheme bmichaelsen
        elseif g:colors_name == 'bmichaelsen'
          colorscheme wintersday
        elseif g:colors_name == 'wintersday'
          colorscheme summerfruit
        elseif g:colors_name == 'summerfruit'
          colorscheme native
        endif
       endif
     endfunction
     map <silent> <F6> :call <SID>SwitchColorSchemes()<CR>
\end{verbatim}

baixe os esquemas aqui:
\url{http://nanasi.jp/old/colorscheme_0.html}

\subsection{Uma função para inserir cabeçalho de script}
\label{Uma função para inserir cabeçalho de script bash}
para chamar a função
basta pressionar, sh em modo normal

\begin{verbatim}
     " Cria um cabeçalho para scripts bash
     fun! InsertHeadBash()
        normal(1G)
        :set ft=bash
        :set ts=4
        call append(0, ``h'')
        call append(1, ``:'' . strftime("%a %d/%b/%Y hs %H:%M"))
        call append(2, "# ultima modificação:``(''%a %d/%b/%Y hs %H:%M"))
        call append(3, "# NOME DA SUA EMPRESA")
        call append(3, "# Propósito do script")
        normal($)
     endfun
     map ,sh :call InsertHeadBash()<cr>
\end{verbatim}

\subsection{Função para inserir cabeçalhos Python}
\label{Função para inserir cabeçalhos Python}

\begin{verbatim}
     " função para inserir cabeçalhos Python
     fun! BufNewFile_PY()
      normal(1G)
      :set ft=python
      :set ts=2
      call append(0, "#!/usr/bin/env python")
      call append(1, "# # -*- coding: ISO-8859-1 -*-")
      call append(2, ``:'' . strftime("%a %d/%b/%Y hs %H:%M"))
      call append(3, `` '' . strftime("%a %d/%b/%Y hs %H:%M"))
      call append(4, "# Instituicao: <+nome+>")
      call append(5, "# Proposito do script: <+descreva+>")
      call append(6, "# Autor: <+seuNome+>")
      call append(7, "# site: <+seuSite+>")
      normal gg
     endfun
     autocmd BufNewFile *.py call BufNewFile_PY()
     map ,py :call BufNewFile_PY()<cr>A
   
     " Ao editar um arquivo será aberto no último ponto em
     " que foi editado
   
     autocmd BufReadPost *
       \ if line('``\''``('''\``'') <= line(``$'') |
       \   exe ''normal g`\``" |
       \ endif
\end{verbatim}

\begin{verbatim}
     " Permite recarregar o Vim para que modificações no
     " Próprio vimrc seja ativadas com o mesmo sendo editado
     nmap <F12> :<C-u>source $HOME/.vimrc <BAR> echo "Vimrc recarregado!"<CR>
\end{verbatim}

Redimensionar janelas

\begin{verbatim}
     " Redimensionar a janela com
     " Alt-seta à direita e esquerda
     map <M-right> <Esc>:resize +2 <CR>
     map <M-left> <Esc>:resize -2 <CR>
\end{verbatim}

\subsection{Função para pular para uma linha}
\label{Função para pular para uma linha}

\begin{verbatim}
     "ir para linha
     " ir para uma linha específica
     function! GoToLine()
     let ln = inputdialog("ir para a linha...")
     exe ``:'' . ln
     endfunction
     "no meu caso o mapeamento é com Ctrl-l
     "use o que melhor lhe convier
     imap <S-l> <C-o>:call GoToLine()<CR>
     nmap <S-l> :call GoToLine()<CR>
\end{verbatim}

\subsection{Função para gerar backup}
\label{Função para gerar backup}

A função abaixo é útil para ser usada quando você vai editar um arquivo
gerando modificações significativas, assim você poderá restaurar o backup se necessário

\begin{verbatim}
     " A mapping to make a backup of the current file.
     fun! WriteBackup()
        let fname = expand("%:p") . "__" . strftime("%d-%m-%Y--%H.%M.%S")
        silent exe ":w " . fname
        echo "Wrote " . fname
     endfun
     nnoremap <Leader>ba :call WriteBackup()<CR>
\end{verbatim}

{\Large \ding{45}} O atalho ``{\tt <leader>}'' em geral é a barra 
invertida ``$\backslash$'', na dúvida ``{\tt :help <leader>}''.

\section{Como adicionar o Python ao {\em path} do Vim?}
\label{Como adicionar o Python ao path do Vim?}

fonte:
\url{http://vim.wikia.com/wiki/Automatically_add_Python_paths_to_Vim_path}
Coloque o seguinte script em:

\begin{verbatim}
     * ~/.vim/after/ftplugin/python.vim    (on Unix systems)
     %* $HOME/vimfiles/after/ftplugin/python.vim    (on Windows systems)
\end{verbatim}

\begin{verbatim}
     python << EOF
     import os
     import sys
     import vim
     for p in sys.path:
         # Add each directory in sys.path, if it exists.
         if os.path.isdir(p):
             # Command `set' needs backslash before each space.
             vim.command(r``s'' % (p.replace(`` '', r`` '')))
     EOF
\end{verbatim}

Isto lhe permite usar `{\tt gf}' ou {\tt Ctrl-w Ctrl-F} para abrir um arquivo sob o cursor

\section{Criando um menu}
\label{Criando um menu}

Como no Vim podemos ter infinitos comandos fica complicado memorizar tudo
é aí que entram os menus, podemos colocar nossos plugins e atalhos favoritos
em um menu veja este exemplo

\begin{verbatim}
     amenu Ferramentas.ExibirNomeDoTema :echo g:colors_name<cr>
\end{verbatim}

O comando acima diz:

\begin{verbatim}
     amenu ........................ cria um menu
     Ferramentas.ExibirNomeDoTema . Menu plugin submenu ExibirNomeDoTema
     :echo g:colors_name<cr> ...... exibe o nome do tema atual
\end{verbatim}

Caso haja espaços no nome a definir você pode fazer assim

\begin{verbatim}
     amenu Ferramentas.Exibir\ nome\ do\ tema :echo g:colors_name<cr>
\end{verbatim}

\section{Criando menus para um modo específico}
\label{Criando menus para um modo específico}

\begin{verbatim}
     :menu .... Normal, Visual e Operator-pending
     :nmenu ... Modo Normal
     :vmenu ... Modo Visual
     :omenu ... Operator-pending modo
     :menu! ... Insert e Comando
     :imenu ... Modo de inserção
     :cmenu ... Modo de comando
     :amenu ... Todos os modos
\end{verbatim}

\section{Exemplo de menu}
\label{Exemplo de menu}

\begin{verbatim}
     " cores
     menu T&emas.cores.quagmire :colo quagmire<CR>
     menu T&emas.cores.inkpot :colo inkpot<CR>
     menu T&emas.cores.google :colo google<CR>
     menu T&emas.cores.ir_black :colo ir_black<CR>
     menu T&emas.cores.molokai :colo molokai<CR>
     " Fontes
     menu T&emas.fonte.Inconsolata :set gfn=Inconsolata:h10<CR>
     menu T&emas.fonte.Anonymous :set anti gfn=Anonymous:h8<CR>
     menu T&emas.fonte.Envy\ Code :set anti gfn=Envy_Code_R:h10<CR>
     menu T&emas.fonte.Monaco :set gfn=monaco:h9<CR>
     menu T&emas.fonte.Crisp :set anti gfn=Crisp:h12<CR>
     menu T&emas.fonte.Liberation\ Mono :set gfn=Liberation\ Mono:h10<CR>
\end{verbatim}

{\Large \ding{45}} O comando ``{\tt :update}'' Atualiza o menu recém modificado.  Quando o comando
``{\tt :amenu}'' É usado sem nenhum argumento o Vim mostra os menus definidos
atualmente.  Para listar todas as opções de menu para `Plugin' por exemplo
digita-se no modo de comandos ``{\tt :amenu Plugin}''.

\section{Outros mapeamentos}
\label{Outros mapeamentos}

Destaca espaços e tabulações redundantes:

\begin{verbatim}
     highlight RedundantWhitespace ctermbg=red guibg=red
     match RedundantWhitespace /\s\+$\| \+\ze\t/
\end{verbatim}

Explicando com detalhes

\begin{verbatim}
     \s ..... espaço
     \+ ..... uma ou mais vezes
     $ ...... no final da linha
     \| ..... ou
     `` '' .. espaço (veja imagem acima)
     \+ ..... uma ou mais vezes
     \ze .... até o fim
     \t ..... tabulação
\end{verbatim}

Portanto a expressão regular acima localizará espaços ou tabulações no final de linha
e destacará em vermelho.


\begin{verbatim}
     "Remove espaços redundantes no fim das linhas
     map <F7> <Esc>mz:%s/\s\+$//g<cr>`z
\end{verbatim}

Um detalhe importante

\begin{verbatim}
     mz ... marca a posição atual do cursor para retornar no final do comando
     `z ... retorna à marca criada
\end{verbatim}

Se não fosse feito isto o cursor iria ficar na linha da última substituição!

\begin{verbatim}
     "Abre o vim explorer
     map <F6> <Esc>:vne .<cr><bar>:vertical resize -30<cr><bar>:set nonu<cr>
\end{verbatim}

Podemos usar ``Expressões Regulares\footnote{\url{http://guia-er.sourceforge.net}}'' em
buscas do Vim veja um exemplo para retirar todas as tags HTML

\begin{verbatim}
     "mapeamento para retirar tags HTML com Ctrl-Alt-t
     nmap <C-M-t> :%s/<[^>]*>//g <cr>
     " Quebra a linha atual no local do cursor com F2
     nmap <F2> a<CR><Esc>
     " join lines  -- Junta as linhas com Shift-F2
     nmap <S-F2> A<Del><Space>
\end{verbatim}

Para mais detalhes sobre buscas acesse o capítulo``\ref{cha:Buscas e substituições}
na página \pageref{cha:Buscas e substituições}''.

\section{Complementação com ``tab''}\label{Complementação com ``tab''}

\begin{verbatim}
     "Word completion
     "Complementação de palavras
     
     set dictionary+=/usr/dict/words
     set complete=.,w,k
     
     "------ complementação de palavras ----
     "usa o tab em modo de inserção para completar palavras
     
     function! InsertTabWrapper(direction)
        let col = col(``.'') - 1
        if !col || getline(``.'')[col - 1] !~ '\k'
           return ``>''
        elseif ``d'' == a:direction
           return ``>''
        else
           return ``>''
        endif
     endfunction
     
     inoremap <tab> <c-r>=InsertTabWrapper (``d'')<cr>
     inoremap <s-tab> <c-r>=InsertTabWrapper (``d'')<cr>
\end{verbatim}

\section{Abreviações}\label{Abreviações}

Abreviações habilitam auto-texto para o Vim. O seu funcionamento consiste de
três campos, o primeiro é o modo no qual a abreviação funcionará, o segundo é a
palavra que irá disparar a abreviação e o terceiro campo é a abreviação
propriamente dita. Para que em {\em modo de comando `:'} a palavra `salvar'
funcione para salvar os arquivos, adiciona-se a seguinte abreviação ao `{\tt
\textasciitilde/.vimrc}'. 

\begin{verbatim}
    cab salvar w
\end{verbatim}

Abaixo abreviações para o modo de inserção:

\begin{verbatim}
     iab slas Sérgio Luiz Araújo Silva
     iab Linux GNU/Linux
     iab linux GNU/Linux
\end{verbatim}

\section{Evitando arquivos de backup no disco}
\label{Evitando arquivos de backup no disco}

Nota-se em algumas situações que existem alguns arquivos com o mesmo nome dos
arquivos que foram editados, porém com um til (\textasciitilde) no final. Esses
arquivos são {\em backups} que o Vim gera antes de sobrescrever os arquivos, e
podem desde ocupar espaço significativo no disco rígido até representar falha
de segurança, como por exemplo arquivos {\tt .php\textasciitilde} que não são
interpretados pelo servidor web e expõem o código-fonte.

Para que os {\em backups} sejam feitos enquanto os arquivos estejam sendo escritos, porém não 
mantidos após terminar a escrita, utiliza-se no \verb|.vimrc|:

\begin{verbatim}
     set nobackup
     set writebackup
\end{verbatim}

Fonte: \url{http://eustaquiorangel.com/posts/520}

\section{Mantendo apenas um Gvim aberto}
\label{Mantenddo apenas um Gvim aberto}

Essa dica destina-se apenas à versão do Vim que roda no ambiente gráfico, ou
seja, o Gvim, pois ela faz uso de alguns recursos que só funcionam nesse
ambiente. A meta é criar um comando que vai abrir os arquivos indicados em abas
novas sempre na janela já existente. 

Para isso deve-se definir um {\em script} que esteja no seu
{\em path}\footnote{Diretórios nos quais o sistema busca pelos comandos} do sistema
(e que possa ser executado de alguma forma por programas do tipo {\em launcher}
no modo gráfico) que vai ser utilizado sempre que quisermos abrir nossos
arquivos dessa maneira. Para efeito do exemplo, o nome do arquivo será {\tt
tvim} (de {\em tabbed vim}), porém pode ser nomeado com o nome que for
conveniente.

A única necessidade para essa dica funcionar é a versão do Vim ter suporte para
o argumento {\tt --serverlist}, o que deve ser garantido nas versões presentes
na época em que esse documento foi escrito. Para fazer uma simples verificação
se o comando está disponível, deve ser digitado em um terminal:

\begin{verbatim}
     vim --serverlist
     gvim --serverlist
\end{verbatim}

Se ambos os comandos acima resultaram em erro, o procedimento não poderá ser
implementado. Do contrário, deve-se utilizar o comando que teve um retorno
válido ({\tt vim} ou {\tt gvim}) para a criar o {\em script}. Supondo que foi o
comando {\tt gvim} que não retornou um erro, criamos o {\em script} da seguinte
forma:

\begin{verbatim}
     #!/bin/bash
     if [ $# -ne 1 ]
     then
        echo "Sem arquivos para editar."
        exit
     fi
     gvim --servername $(gvim --serverlist | head -1) --remote-tab $1
\end{verbatim}

Desse modo, se for digitado {\tt tvim} sem qualquer argumento, é exibida a
mensagem de erro, do contrário, o arquivo é aberto na cópia corrente do Gvim,
em uma nova aba, por exemplo:

\begin{verbatim}
     tvim .vimrc
\end{verbatim}

Fonte: \url{http://eustaquiorangel.com/posts/477}

\section{Referências}
\label{Referências}
* \url{http://www.dicas-l.com.br/dicas-l/20050118.php}

%%%%%%%%%%%%%%%%%%%%%%%%%%%%%%%%%%%%%%%%%%%%%%%%%%%%%%%%%%%%%%%%%%%%%%%%%%%%%%%%
% vim:enc=utf-8:ts=5:sw=5:et:ff=unix:
%
% Permission is granted to copy, distribute and/or modify this document
% under the terms of the GNU Free Documentation License, Version 1.3 or
% any later version published by the Free Software Foundation.
%	
% A copy of the license is included in the file called "COPYING".
%%%%%%%%%%%%%%%%%%%%%%%%%%%%%%%%%%%%%%%%%%%%%%%%%%%%%%%%%%%%%%%%%%%%%%%%%%%%%%%%
\chapter{Um Wiki para o Vim}\label{cha:Um Wiki para o Vim}

É inegável a facilidade que um Wiki nos traz, os documentos são
indexados e linkados de forma simples. Já pesquisei uma porção de
Wikis e, para uso pessoal recomendo o Potwiki.  O ``link'' do Potwiki é:
\href{http://www.vim.org/scripts/script.php?script\_id=1018}{este}~\cite{PluginPotWiki}.
O Potwiki é um Wiki completo para o Vim, funciona localmente embora
possa ser aberto remotamente via ssh\footnote{Sistema de acesso remoto}.
Para criar um ``link'' no Potwiki basta usar WikiNames, são nomes
iniciados com letra maiúscula e que contenham outra letra em maiúsculo
no meio. \\


Ao baixar o arquivo salve em \verb|~/.vim/plugin|. \\



Mais ou menos na linha 53 do Potwiki \verb|~/.vim/plugin/potwiki.vim| você
define onde ele guardará os arquivos, no meu caso
\verb|/home/docs/textos/wiki|. a linha ficou assim:

\begin{verbatim}
     call s:default('home',"~/.wiki/HomePage")
\end{verbatim}

Outra forma de indicar a página inicial seria colocar no seu .virmc

\begin{verbatim}
     let potwiki_home = "$HOME/.wiki/HomePage"
\end{verbatim}

\section{Como usar}
\label{Como usar}

O Potwiki trabalha com WikiWords, ou seja, palavras iniciadas com
letras em maiúsculo e que tenham outra letra em maiúsculo no meio (sem
espaços). Para iniciar o Potwiki abra o Vim e pressione \verb|\ww|.

\begin{verbatim}
     <Leader> é igual a \   - veja :help leader
     \ww  .... abra a sua HomePage
     \wi  .... abre o Wiki index
     \wf  .... segue uma WikiWords (can be used in any buffer!)
     \we  .... edite um arquivo Wiki
     \\   .... Fecha o arquivo
     <CR> .... segue WikiWords embaixo do cursor <CR> é igual a Enter
     <Tab>.... move para a próxima WikiWords
     <BS> .... move para os WikiWords anteriores (mesma página)
     \wr  .... recarrega WikiWords
\end{verbatim}

\section{Salvamento automático para o Wiki }
\label{Salvamento automático para o Wiki }
Procure por uma seção {\em autowrite} no manual do Potwiki

\begin{verbatim}
     :help potwiki
\end{verbatim}

O valor que está em zero deverá ficar em 1

\begin{verbatim}
     call s:default(`autowrite',0)
\end{verbatim}

\section{Dicas}
\label{Dicas}
Como eu mantenho o meu Wiki oculto ``.wiki'' criei um ``link'' para a pasta de textos

\begin{verbatim}
     ln -s ~/.wiki /home/sergio/docs/textos/wiki
\end{verbatim}

Vez por outra entro na pasta \verb|~/docs/textos/wiki| e crio um
pacote {\tt tar.gz} e mando para ``web'' como forma de manter um ``backup''.

\section{Problemas com codificação de caracteres}
\label{Problemas com codificação de caracteres}

Atualmente uso o Ubuntu em casa e ele já usa utf-8. Ao restaurar meu
``backup'' do Wiki no Kurumin os caracteres ficaram meio estranhos,
daí fiz:

\begin{verbatim}
     baixei o pacote [recode]
     # apt-get install recode
     
     para recodificar caracteres de utf-8 para isso faça:
     recode -d u8..l1 arquivo
\end{verbatim}

%%%%%%%%%%%%%%%%%%%%%%%%%%%%%%%%%%%%%%%%%%%%%%%%%%%%%%%%%%%%%%%%%%%%%%%%
% vim:enc=utf-8:ts=5:sw=5:et:ff=unix:
%%%%%%%%%%%%%%%%%%%%%%%%%%%%%%%%%%%%%%%%%%%%%%%%%%%%%%%%%%%%%%%%%%%%%%%%

\chapter{Hábitos para Edição Efetiva}
\label{cha:Hábitos para edição efetiva}

Um dos grandes problemas relacionados com os softwares é sua subutilização. Por
inércia o usuário tende a aprender o mínimo para a utilização de um programa e
deixa de lado recursos que poderiam lhe ser de grande valia. O mantenedor do
Vim, Bram Moolenaar\footnote{http://www.moolenaar.net}, recentemente publicou
vídeos e manuais sobre os ``7 hábitos para edição efetiva de
textos''\footnote{http://br-linux.org/linux/7-habitos-da-edicao-de-texto-efetiva},
este capítulo pretende resumir alguns conceitos mostrados por Bram Moolenaar em
seu artigo.

\section{Mova-se rapidamente no texto}
\label{sec:Mova-se rapidamente no texto}

O capítulo \ref{cha:Movendo-se no Documento}, ``Movendo-se no Documento'', 
página~\pageref{cha:Movendo-se no Documento} mostra uma série de comandos para
agilizar a navegação no texto. Memorizando estes comandos ganha-se tempo
considerável, um exemplo simples em que o usuário está na linha 345 de um arquivo
decide ver o conteúdo da linha 1 e em seguida voltar à linha 345:

\index{movendo-se!entre linhas}
\begin{verbatim}
     gg ....... vai para a linha 1
     '' ....... retorna ao último ponto em que estava
\end{verbatim}

Fica claro portanto que a navegação rápida é um dos requisitos para edição
efetiva de documentos.


\section{Use marcas}
veja a seção \ref{sec:UsandoMarcas} na página \pageref{sec:UsandoMarcas}.

\index{marcas}
\begin{verbatim}
     ma ..... em modo normal cria uma marca `a'
     'a ..... move o cursor até a marca `a'
     d'a .... deleta até a marca `a'
     y'a .... copia até a marca `a'
\end{verbatim}



\begin{verbatim}
     gg ... vai para a linha 1 do arquivo
     G .... vai para a última linha do arquivo
     0 .... vai para o início da linha
     $ .... vai para o fim da linha
     fx ... pula até a próxima ocorrência de ``x''
     dfx .. deleta até a próxima ocorrência de ``x''
     g, ... avança na lista de alterações
     g; ... retrocede na lista de alterações
     p .... cola o que foi deletado/copiado abaixo
     P .... cola o que foi deletado/copiado acima
     H .... posiciona o cursor no primeiro caractere da tela
     M .... posiciona o cursor no meio da tela
     L .... posiciona o cursor na última linha da tela
\end{verbatim}

\begin{verbatim}
     * ........ localiza a palavra sob o cursor
     % ........ localiza fechamentos de chaves, parênteses etc.
     g* ....... localiza palavra parcialmente
\end{verbatim}

\begin{verbatim}
     '.  apostrofo + ponto retorna ao último local editado
     '' retorna ao local do ultimo salto
\end{verbatim}

Suponha que você está procurando a palavra `argc':

\begin{verbatim}
     /argc
\end{verbatim}

Digita `n' para buscar a próxima ocorrência

\begin{verbatim}
     n
\end{verbatim}

Um jeito mais fácil seria:

\begin{verbatim}
     "coloque a linha abaixo no seu vimrc
     :set hlsearch
\end{verbatim}

Agora use asterisco para destacar todas as ocorrências do padrão desejado
e use a letra `n' para pular entre ocorrências, caso deseje seguir o caminho
inverso use `N'.

\section{Use quantificadores}
\label{Use quantificadores}
\index{qunatificadores}
Em modo normal você pode fazer

\begin{verbatim}
     10j ..... desce 10 linhas
     5dd ..... apaga as próximas 5 linhas
     :50 ..... vai para a linha 50
     50gg .... vai para a linha 50
\end{verbatim}


\section{Edite vários arquivos de uma só vez }
\label{Edite vários arquivos de uma só vez }
\index{editando!vários arquivos}

O Vim pode abrir vários arquivos que contenham um determinado padrão.
Um exemplo seria abrir dezenas de arquivos HTML e trocar a ocorrência
\verb+bgcolor="ffffff"+ Para \verb+bgcolor="eeeeee"+ Usaríamos a seguinte 
sequência de comandos:

\begin{verbatim}
     vim *.html  .............................. abre os arquivos
     :bufdo :%s/bgcolor=`ffffff'/bgcolor=`eeeeee'/g   substituição
     :wall .................................... salva todos
     :qall .................................... fecha todos
\end{verbatim}

Ainda com relação à edição de vários arquivos poderíamos abrir alguns
arquivos {\tt txt} e mudar de um para o outro assim:

\begin{verbatim}
     :wn
\end{verbatim}

O `{\tt w}' significa gravar e o `{\tt n}' significa {\em next}, ou seja,
gravaríamos o que foi modificado no arquivo atual e mudaríamos para o próximo.

{\Large {\ding{45}}} Veja também ``Movendo-se no documento'',
capítulo~\ref{cha:Movendo-se no Documento} página~\pageref{cha:Movendo-se no
Documento}

\section{Não digite duas vezes}
\label{Não digite duas vezes}

\begin{itemize}
\item O Vim complementa com `{\tt tab}'. Veja mais na seção \ref{Complementação com ``tab''} na página \pageref{Complementação com ``tab''}.
\item Use macros. Detalhes na seção \ref{sec:Gravando comandos}
página \pageref{sec:Gravando comandos}.
\item Use abreviações coloque abreviações como abaixo em seu \verb|~/.vimrc|. Veja mais na seção \ref{Abreviações}.
\item As abreviações fazem o mesmo que auto-correção e auto-texto em outros editores
\end{itemize}

\begin{verbatim}
     iab tambem também
     iab linux GNU/Linux
\end{verbatim}

No modo de inserção você pode usar:

\begin{verbatim}
     Ctrl-y  ........ copia caractere a caractere a linha acima
     Ctrl-e  ........ copia caractere a caractere a linha abaixo
     Ctrl-x Ctrl-l .. completa linhas inteiras
\end{verbatim}

Para um trecho muito copiado coloque o seu conteúdo em um registrador:

\begin{verbatim}
     "ayy ... copia a linha atual para o registrador `a'
     "ap  ... cola o registrador `a'
\end{verbatim}

Crie abreviações para erros comuns no seu arquivo de configuração (~/.vimrc):

\begin{verbatim}
     iabbrev teh the
     syntax keyword WordError teh
\end{verbatim}

As linhas acima criam uma abreviação para erro de digitação da palavra `the'
e destaca textos que você abrir que contenham este erro.

\section{Use dobras}\label{sec:Use folders}
\index{folders}

O Vim pode ocultar partes do texto que não estão sendo utilizadas permitindo
uma melhor visualização do conteúdo. Mais detalhes no capítulo
\ref{cha:Folders} página \pageref{cha:Folders}.

\section{Use autocomandos}
\label{Use autocomandos}
\vimhelp{autocmd.txt}
\index{autocomandos}

No arquivo de configuração do Vim \verb|~/.vimrc| pode-se pode criar comandos
automáticos que serão executados diante de uma determinada circunstância. 
O comando abaixo será executado em qualquer arquivo existente, ao abrir o mesmo, 
posicionando o cursor no último local editado:

\begin{verbatim}
     "autocmd BufEnter * lcd %:p:h
     autocmd BufReadPost *
       \ if line("'\"") > 0 && line("'\"") <= line("$") |
       \   exe "normal g`\"" |
       \ endif
\end{verbatim}


Grupo de comandos para arquivos do tipo `{\tt html}'. Observe que o autocomando
carrega um arquivo de configuração do Vim exclusivo para o tipo {\tt html/htm}
e no caso de arquivos novos `{\tt BufNewFile}' ele já cria um esqueleto puxando
do endereço indicado:

\begin{verbatim}
augroup html
 au! <--> Remove all html autocommands
  au!
  au BufNewFile,BufRead *.html,*.shtml,*.htm set ft=html
  au BufNewFile,BufRead,BufEnter  *.html,*.shtml,*.htm so ~/docs/vim/.vimrc-html
  au BufNewFile *.html 0r ~/docs/vim/skel.html
  au BufNewFile *.html*.shtml,*.htm /body/+  " coloca o cursor após o corpo <body>
  au BufNewFile,BufRead *.html,*.shtml,*.htm set noautoindent
augroup end
\end{verbatim}


\section{Use o File Explorer}\label{Use o file explorer}
\index{fileExplorer}

O Vim pode navegar em pastas assim:

\begin{verbatim}
     vim .
\end{verbatim}

Você pode usar `{\tt j}' e `{\tt k}' para navegar e {\tt Enter} para editar o arquivo
selecionado:

\section{Torne as boas práticas um hábito }
\label{Torne as boas práticas um hábito }

Para cada prática produtiva procure adquirir um hábito e mantenha-se
atento ao que pode ser melhorado. Imagine tarefas complexas, procure
um meio melhor de fazer e torne um hábito.

% Isso deve ser deslocado para outro lugar
\section{Referências}
%\label{Referências} % label repetida
\begin{itemize}
   \item \url{http://www.moolenaar.net/habits\_2007.pdf} por Bram Moolenaar
   \item \url{http://vim.wikia.com/wiki/Did\_you\_know}
\end{itemize}

\chapter{Plugins}\label{Plugins}

``Plugins'' são um meio de estender as funcionalidades do Vim, há
``plugins'' para diversas tarefas, desde wikis para o Vim até
ferramentas de auxílio a navegação em arquivos com é o caso do
``plugin'' \url{http://www.vim.org/scripts/script.php?script\_id=1658}
NerdTree, que divide uma janela que permite navegar pelos diretórios
do sistema a fim de abrir arquivos a serem editados.

\section{Como testar um plugin sem instalá-lo?}
\label{Como testar um plugin sem instala-lo?}

\begin{verbatim}
     :source <path>/<plugin>
\end{verbatim}

Caso o plugin atenda suas necessidades você pode instala-lo. Este
procedimento também funciona para temas de cor!



No GNU/Linux
\begin{verbatim}
     ~/.vim/plugin/
\end{verbatim}

No Windows

\begin{verbatim}
     ~/vimfiles/plugin/
\end{verbatim}

Obs: Caso não exista a pasta você pode criá-la!

Exemplo no GNU/Linux

\begin{verbatim}
     + /HOME/USER
           |
           |
            + .VIM
                |
                |
                + PLUGIN
\end{verbatim}

Obs: Alguns plugins dependem da versão do Vim, para saber qual
a que está atualmente instalada:

\begin{verbatim}
     :ver
\end{verbatim}

\section{Plugin para \LaTeX}
\label{Plugin para LaTeX}
Um plugin completo para \LaTeX está acessível aqui: \url{http://vim-latex.sourceforge.net/}
Uma vez adicionado o plugin você pode inserir seus {\em templates}
em:

\begin{verbatim}
     ~/.vim/ftplugin/latex-suite/templates
\end{verbatim}


\section{Criando folders para arquivos \LaTeX}
\label{Criando folders para arquivos LaTeX}

\begin{verbatim}
     set foldmarker=\\begin,\\end
     set foldmethod=marker
\end{verbatim}

Adicionar marcadores ({\em labels}) às seções de um documento \LaTeX
\begin{verbatim}
     :.s/^\(\\section\)\({.*}\)/\1\2\r\\label\2
     
     : ........... comando
     / ........... inicia padrão de busca
     ^ ........... começo de linha
     \(palavra\) . agrupa um trecho
     \(\\section\) agrupa `\section'
     \\ .......... torna \ literal
     { ........... chave literal
     .* .......... qualquer caractere em qualquer quantidade
     } ........... chave literal
     / ........... finaliza parão de busca
     \1 .......... repeter o grupo 1 \(\\section\) 
     \2 .......... repete o grupo 2 \({.*\}\)
     \r .......... insere quebra de linha
     \\ .......... insere uma barra invertida
     \2 .......... repete o nome da seção
\end{verbatim}

\section{Criando seções \LaTeX}\label{Criando seções latex}
o comando abaixo substitui

\begin{verbatim}
     ==seção==
\end{verbatim}

   por

\begin{verbatim}
     \section{seção}
\end{verbatim}

\begin{verbatim}
     :.s/^==\s\?\([^=]*\)\s\?==/\\section{\1}/g
     
     : ......... comando
     . ......... linha atual
     s ......... substitua
     ^ ......... começo de linha
     == ........ dois sinais de igual
     \s\? ...... seguido ou não de espaço
     [^=] ...... não pode haver = (^ dentro de [] é negação)
     * ......... diz que o que vem antes pode vir zero ou mais vezes
     \s\? ...... seguido ou não de espaço
     \\ ........ insere uma barra invertida
     \1 ........ repete o primeiro trecho entre ()
\end{verbatim}

\section{Plugin para manipular arquivos}
\url{http://www.vim.org/scripts/script.php?script_id=2337#0.1.9}
Para entender este plugin acesse este vídeo:
 \url{http://www.screencast.com/t/P6nJkJ0DE}


\section{Complementação de códigos}
\label{Complementação de códigos}

O ``plugin'' snippetsEmu é um misto entre complementação de códigos e
os chamados modelos ou {\em templates}. Insere um trecho de código pronto,
mas vai além disso, permitindo saltar para trechos do modelo inserido
através de um atalho configurável de modo a agilizar o trabalho do
programador. \url{http://www.vim.org/scripts/script.php?script\_id=1318}

\section{Instalação}
\label{Instalação}

Um artigo ensinando como instalar o ``plugin'' snippetsEmu pode ser lido aqui:
 \url{http://vivaotux.blogspot.com/2008/03/instalando-o-plugin-snippetsemu-no-vim.html}

\section{Um wiki para o Vim}
\label{sec:Um wiki para o Vim}

O ``plugin'' wikipot implementa um wiki para o Vim no qual você define
um ``link'' com a notação WikiWord, onde um ``link'' é uma palavra que
começa com uma letra maiúscula e tem outra letra maiúscula no meio
Obtendo o plugin neste link: \url{http://www.vim.org/scripts/script.php?script\_id=1018}.

\section{Acessando documentação do python no Vim}\label{Acessando documentação do python no Vim}

 \url{http://www.vim.org/scripts/script.php?script\_id=910}

\section{Formatando textos planos com syntax}
\label{Formatando textos planos com syntax}
\url{http://www.vim.org/scripts/script.php?script\_id=2208&rating=helpful#1.3}

Veja como instalar o este plugin no capítulo \ref{sec:Um wiki para o Vim}.

\section{Movimentando em camel case}
\label{Movimentando em camel case}

O {\em plugin} \href{http://www.vim.org/scripts/script.php?script_id=1905}{{\tt CamelCaseMotion}}
auxilia a navegação em palavras em \href{http://en.wikipedia.org/wiki/Camel_case}{{\em camel case}}
ou separadas por sublinhados, através de mapeamentos similares aos que fazem a movimentação normal entre 
strings, e é um recurso de grande ajuda quando o editor é utilizado para programação. 

Após instalado o plugin, os seguintes atalhos ficam disponíveis:
\begin{description}
 \item [,w] Movimenta para a próxima posição {\em camel} dentro da string
 \item [,b] Movimenta para a posição {\em camel} anterior dentro da string
 \item [,e] Movimenta para o caracter anterior à proxima posição {\em camel} dentro da string
\end{description}

Fonte: \url{http://eustaquiorangel.com/posts/522}

\section{Plugin FuzzyFinder}
\label{sec:Plugin FuzzyFinder}
                                                                       
Este plugin é a implementação de um recurso do editor 
Texmate\footnote{Editor de textos da Apple com muitos recursos}.
Sua proposta é acessar de forma rápida:


\begin{enumerate}
\item Arquivos \verb|:FuzzyFinderFile|
\item Arquivos recem editados \verb|:FuzzyFinderMruFile|
\item Comandos recem utilizados \verb|:FuzzyFinderMruCmd|
\item Favoritos \verb|:FuzzyFinderAddBookmark, :FuzzyFinderBookmarks|
\item Navegação por diretórios \verb|:FuzzyFinderDir|
\item Tags {\tt :FuzzyFinderTag}
\end{enumerate}

Para ver o plugin em ação acesse este link: 
\url{http://vimeo.com/2938498}.

\subsection{Obtendo e instalando o FuzzyFinder}
O plugin pode ser obtido no seguinte endereço: 
\url{http://www.vim.org/scripts/script.php?script_id=1984},
para instala-lo basta copiar para o diretorio 
{\tt ~/.vim/plugin}.

\section{sec:O plugin EasyGrep}
\label{sec:O plugin EasyGrep}

Usuários de sistemas {\em Unix Like}\footnote{Sistemas da família Unix tipo o Linux}, já
conhecem o poder do comando {\tt grep}, usando este comando procuramos palavras dentro 
de arquivos, este plugin simplifica esta tarefa, além de permitir a utilização
da versão do grep nativa do Vim {\tt vimgrep}, assim usuários do Windows também podem usar 
este recurso. Um comando grep funciona mais ou menos assim:

\begin{verbatim}
     grep [opções] "padrão" /caminho
\end{verbatim}

Mas no caso do plugin {\em EasyGrep} fica assim:

\begin{verbatim}
     :Grep foo  ........ procura pela palavra 'foo'
     :GrepOptions ...... exibe as opções de uso do plugin
\end{verbatim}

\subsection{Obtendo e instalando o plugin {\em EasyGrep}?}
O plugin pode ser obtido no seguinte endereço: 
``\url{http://www.vim.org/scripts/script.php?script_id=2438#0.9}'', já sua instalação
é simples, basta copiar o arquivo obtido no link acima para a pasta:
   
\begin{verbatim}
     ~/.vim/plugin .......... no caso do linux
     ~/vimfiles/plugin ...... no caso do windows
\end{verbatim}

Um vídeo de exemplo (na verdade uma animação gif)
\url{http://downloads.veryspeedy.net/vim/EasyGrep.gif}

\section{sec:O plugin {\em SearchComplete}}
\label{sec:O plugin {\em SearchComplete}}

Para que o vim complete opções de busca ``com a tecla <tab>'', digita-se uma
palavra parcialmente e o plugin atua, exibindo palavras que tem 
o mesmo início, por exemplo:

\begin{verbatim}
     /merca<tab>
     /mercado
     /mercantil
     /mercadológico
\end{verbatim}

Cada vez que se pressiona a tecla {\tt <tab>} o cursor saltará para 
a próxima ocorrência daquele fragmento de palavra.

\subsection{Como instalar o plugin {\em SearchComplete}}
Pode-se obte-lo no seguinte endereço: ``\url{http://www.vim.org/scripts/script.php?script_id=474}'', 
e para instala-lo basta copia-lo para a pasta apropriada:
    
\begin{verbatim}
     ~/vimfiles/plugin .......... no windows
     ~/.vim/plugin .............. no Gnu/Linux
\end{verbatim}

%%%%%%%%%%%%%%%%%%%%%%%%%%%%%%%%%%%%%%%%%%%%%%%%%%%%%%%%%%%%%%%%%%%%%%%%
% vim:enc=utf-8:ts=5:sw=5:et:ff=unix:
%%%%%%%%%%%%%%%%%%%%%%%%%%%%%%%%%%%%%%%%%%%%%%%%%%%%%%%%%%%%%%%%%%%%%%%%

\chapter{Referências}
\begin{itemize}
\item \url{http://www.vivaolinux.com.br/artigos/impressora.php?codigo=2914} VIM avançado (parte 1)]
\item \url{http://www.rayninfo.co.uk/vimtips.html}
\item \url{http://www.geocities.com/yegappan/vim\_faq.txt}
\item \url{http://br.geocities.com/cesarakg/vim-cook-ptBR.html}
\item \url{http://larc.ee.nthu.edu.tw/~cthuang/vim/files/vim-regex/vim-regex.htm}
\item \url{http://aurelio.net/vim/vimrc-ivan.txt}
\item \url{http://vivaotux.blogspot.com/search/label/vim}
\item \url{http://www.tug.dk/FontCatalogue/seriffonts.html}
\end{itemize}


%\nocite{*} % Se descomentado, todas as referências do arquivo 
            % de bibliografia serão adicionadas ao texto, mesmo 
            % se não forem citados.
% Redefinindo o estilo de topo de página
\fancyhf{} 
\fancyhead[LE,RO]{\bfseries\thepage}
\fancyhead[LO]{\bfseries Referências Bibliográficas}
\fancyhead[RE]{\bfseries Referências Bibliográficas}
% Estilo de Blibliografia
% coppe-unsrt é o estilo COPPE de referência baseado
% em código numérico.
% Uma mistura de ABNT com algo indefinido.
\bibliographystyle{coppe-unsrt}
\bibliography{bibliografia}

% Incluindo os apêndices
% \backmatter % se descomentado, remove a expressão "Apêndice LETRA" 
% colocando de volta o estilo de topo de página.
% \fancyhf{} 
% \fancyhead[LE,RO]{\bfseries\thepage}
% \fancyhead[LO]{\bfseries\rightmark}
% \fancyhead[RE]{\bfseries\leftmark}
% \appendix
% \include{sem_anexos_reais_no_momento}

% Anexo, que não é anexo, é uma página dos finalmentes.
% Da forma que está abaixo, fica uma homenagem mais agradável
\clearpage
% Estilo de topo
\fancyhf{} 
% É visualmente mais agradável sem
% \fancyhead[LE,RO]{\bfseries\thepage}
% \fancyhead[LO]{\bfseries Colaboradores}
% \fancyhead[RE]{\bfseries Colaboradores}
% Inclui a lista de colaboradores, conscientes e inconscientes. 
%%%%%%%%%%%%%%%%%%%%%%%%%%%%%%%%%%%%%%%%%%%%%%%%%%%%%%%%%%%%%%%%%%%%%%%%
% vim:enc=utf-8:ts=5:sw=5:et:ff=unix:
%%%%%%%%%%%%%%%%%%%%%%%%%%%%%%%%%%%%%%%%%%%%%%%%%%%%%%%%%%%%%%%%%%%%%%%%

\begin{center}
\begin{Huge}~\\[10mm]\textbf{{Colaboradores}}\\[20mm]\end{Huge}
\phantomsection
\addcontentsline{toc}{chapter}{Colaboradores}
\end{center}

Seção dedicada aos colaboradores do projeto \href{http://code.google.com/p/vimbook}{vimbook}
% Não era hora de alguém usar essa página ?




% índice
\fancyhf{} 
\fancyhead[LE,RO]{\bfseries\thepage}
\fancyhead[LO]{\bfseries Índice}
\fancyhead[RE]{\bfseries Índice}
% é a última coisa do livro
\printindex
\end{document}
